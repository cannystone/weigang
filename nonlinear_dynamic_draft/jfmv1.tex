% This is file JFM2esam.tex
% first release v1.0, 20th October 1996
%       release v1.01, 29th October 1996
%       release v1.1, 25th June 1997
%       release v2.0, 27th July 2004
%       release v3.0, 16th July 2014
%   (based on JFMsampl.tex v1.3 for LaTeX2.09)
% Copyright (C) 1996, 1997, 2014 Cambridge University Press

%%%%%%%%%%%%%%%%%%%%%%%%%%%%%%%%%%%%%%%% package include 


\documentclass{jfm}
\usepackage{graphicx}

\usepackage{epstopdf, epsfig}
\usepackage{color,soul} %highlight with color
\graphicspath{{./figures_num/}} % Specifies the directory where pictures are stored
\newtheorem{lemma}{Lemma}
\newtheorem{corollary}{Corollary}
\usepackage{amssymb}
\usepackage{amsmath}
\usepackage{natbib}
\usepackage{subfloat}
\usepackage{subcaption}
\usepackage{bm}
\usepackage{multirow}
\usepackage{xcolor}
\usepackage{tikz}
%\usepackage{pgfplots}
%\usepackage{amsbsy}
%\usepackage{amsfonts}
%\usepackage{booktabs,array,dcolumn} 
%\usepackage{ulem}

%%======================New commands============================================
\newcommand{\todo}[1]{\textcolor{magenta}{#1}}
\newcommand{\changes}[1]{\textcolor{magenta}{#1}}     %%defined a new function to trace changes by modfiy font color
\def\vec#1{\mbox{\boldmath $#1$}}
\newcommand{\bu}{\mathbf{u}}
\newcommand{\bw}{\mathbf{w}}
\newcommand{\bn}{\mathbf{n}}
\newcommand{\bnx}{\mathbf{n_x}}
\newcommand{\bny}{\mathbf{n_y}}
\newcommand{\bs}{\boldsymbol{\sigma}}
\newcommand{\disnum}[1]{\texttt{#1}}
\def\Otf{\Omega^\mathrm{f}(t)}
\def\Ots{\Omega^\mathrm{s}(t)}
\def\Of{\Omega^\mathrm{f}}
\def\Os{\Omega^\mathrm{s}}
\def\bz{\mathbf z}
\newcommand{\xx}{\mbox{$\mathbf{x}^{\mathrm{f}}$}}
\def\strain{\vec \epsilon}
\def\stress{{\vec \sigma}}
\def\div{\vec \nabla}
\def\G{\Gamma}
\def\vphi{\vec{\varphi}^\mathrm{s}}
\newcommand{\bchi}{\mathbf{\chi}}

\usepackage{environ}
\NewEnviron{myequation}{%
\begin{equation}
\scalebox{0.7}{$\BODY$}
\end{equation}
}
%\newcommand{\hwplotA}{\raisebox{2pt}{\tikz{\draw[thick,dash dot] (0,1) -- (5,1);}}}
\newcommand{\reddashdot}{\raisebox{2pt}{\tikz{\draw[red,dashdotted,line width=1.2pt](0,0) -- (5mm,0);}}}
\newcommand{\bluedashdot}{\raisebox{2pt}{\tikz{\draw[blue,dashdotted,line width=1.2pt](0,0) -- (5mm,0);}}}
\newcommand{\greendashdot}{\raisebox{2pt}{\tikz{\draw[green,dashdotted,line width=1.2pt](0,0) -- (5mm,0);}}}
\newcommand{\greendash}{\raisebox{2pt}{\tikz{\draw[green,dashed,line width=1.2pt](0,0) -- (5mm,0);}}}
\newcommand{\greensolid}{\raisebox{2pt}{\tikz{\draw[green,solid,line width=1.2pt](0,0) -- (5mm,0);}}}
\newcommand{\reddash}{\raisebox{2pt}{\tikz{\draw[red,dashed,line width=1.2pt](0,0) -- (5mm,0);}}}
\newcommand{\reddot}{\raisebox{2pt}{\tikz{\draw[red,dotted,line width=1.2pt](0,0) -- (5mm,0);}}}
%\usepackage{placeins}
%\usepackage{tabularx}
%\usepackage{epstopdf}
%\usepackage{afterpage}
%%=======================================================================================

% For multiletter symbols
%\newcommand\Real{\mbox{Re}} % cf plain TeX's \Re and Reynolds number
%\newcommand\Imag{\mbox{Im}} % cf plain TeX's \Im
%\newcommand\Rey{\mbox{\textit{Re}}}  % Reynolds number
%\newcommand\Pran{\mbox{\textit{Pr}}} % Prandtl number, cf TeX's \Pr product
%\newcommand\Pen{\mbox{\textit{Pe}}}  % Peclet number
%%%%%%%%%%%%%%%%%%%%%%%%%%%%%%%%%%%%%%%%

\shorttitle{Model Reduction and Mechanism of VIV}
\shortauthor{W. Yao and R. K. Jaiman}

\title{Model Reduction and Mechanism for the Vortex-Induced Vibrations of Bluff Bodies}

\author{W. Yao
  \and  R. K. Jaiman
   \corresp{\email{mperkj@nus.edu.sg}}
  }
 
\affiliation{Department of Mechanical Engineering, National University Singapore, Singapore 119077}

%%%%%%%%%%%%%%%%%%%%%%%%%%%%%%%%%%%%%%%%

\begin{document}

\maketitle

\begin{abstract}
We present an effective reduced-order model (ROM) technique to couple 
an incompressible flow with a transversely vibrating bluff body in a state-space format.
%
The ROM of unsteady wake flow is based on the Navier-Stokes equations 
and is constructed by means of eigensystem realization algorithm (ERA).
%
We investigate the  underlying mechanism of vortex-induced vibration (VIV) 
of a circular cylinder at low Reynolds number via linear stability analysis.
%
To understand the frequency lock-in mechanism and self-sustained VIV phenomenon, 
a systematic analysis is performed by examining the eigenvalue trajectories  
of ERA-based ROM for a range of reduced oscillation frequency $(F_s)$, 
while maintaining fixed values of Reynolds number ($Re$) and mass ratio ($m^*$).
%
The effects of Reynolds number $Re$, the mass ratio $m^*$ and the rounding of square cylinder 
are examined to generalize the proposed ERA-based ROM for the VIV lock-in analysis.  
The considered cylinder configurations are the basic square with sharp corners, 
circle and three intermediate rounded squares, which are created 
by varying a single rounding parameter. 
%
Results show that the two frequency lock-in regimes, the so-called 
\emph{resonance} and \emph{flutter},  
only exist when certain conditions are satisfied and the regimes 
have a strong dependence on the shape of bluff body, Reynolds number and the mass ratio. 
%
In addition, the frequency lock-in during VIV of square cylinder 
is found to be dominated by the resonance regime without any coupled mode 
flutter at low Reynolds number. 
%
To further discern the influence of geometry on the VIV lock-in mechanism, we consider a smooth curve 
geometry of ellipse and two sharp corner geometries of forward triangle and diamond-shaped 
bluff bodies.
%
While the ellipse and diamond geometries exhibit the 
flutter and mixed resonance-flutter regimes, the forward triangle undergoes only 
the flutter-induced lock-in for $30 \le Re \le 100 $ at $m^*=10$.
% galloping 
In the case of forward triangle configuration, the ERA-based ROM accurately predicts 
the low-frequency galloping instability. We observe a kink in the amplitude 
response associated with 1:3 synchronization whereby 
the forward triangular body oscillates at a single dominant frequency but the lift force 
has the frequency component at three times of the body oscillation frequency. 
%
Finally, we present a stability phase diagram to summarize 
the VIV lock-in regimes of the five smooth curve and sharp-cornered based bluff bodies.
%
These findings attempt to generalize our understanding of VIV lock-in mechanism 
for bluff bodies at low Reynolds number.
%
The proposed ERA-based ROM is found to be accurate, efficient 
and ease-of-use for the linear stability analysis of VIV and it can have a profound 
impact on the development of control strategies for nonlinear vortex shedding and VIV.
\end{abstract}

\begin{keywords}
vortex-induced vibration, low-dimensional models, unstable wake flow
\end{keywords}

\section{Introduction}\label{sec:intro}


\subsection{Vortex-induced vibration}
%% definition of VIV 
Vortex shedding from a bluff body and the vortex-induced vibration (VIV) are ubiquitous 
and have a broad range of applications in numerous fields such 
as offshore, wind and aerospace engineering. Apart from their great practical 
importance, these phenomena in fluid mechanics have a fundamental value 
due to the vast richness of their vorticity dynamics and coupled nonlinear physics.
Asymmetric vortex shedding shed from a bluff 
body causes a large unsteady transverse load, which in turn may lead to structural 
vibrations when the structure is free 
to vibrate in the transverse direction (\cite{sarpkaya2004,williamson2004,bearman2011}). 
These large vortex-induced vibrations can lead to damage and potential risk to the structures, 
in particular for ocean structures such as marine risers, subsea pipelines and cables.
%
When the natural frequency of the structure is close the vortex shedding frequency, 
the phenomenon of VIV results into a complex evolution of the shedding frequency, which 
deviates from the Strouhal relation of stationary counterpart. In this 
frequency lock-in regime, the vortex formation locks on to the natural frequency of body 
within a range of the Strouhal frequency and there exists a strong coupling 
between the fluid and the structure \citep{sarpkaya2004}. Such frequency lock-in 
phenomenon leads to high amplitude and self-sustained vibrations, thus there is a need 
to understand the origin and different regimes during the lock-in process. 
%
The lock-in process is self-excited and is characterized by the matching of the frequency 
of periodic vortex shedding and the oscillation frequency of the body \citep{khalak1999}.
%
%

%% VIV research 
The flow over a single elastically  mounted two-dimensional bluff body has served as a 
generic VIV model for both numerical and experimental investigations.
%
In this canonical configuration, it is often convenient to consider the elastically mounted 
cylinder as two coupled oscillators whereby one system is the oscillating body and the other one is 
the wake.
%
Numerous studies have been conducted to understand the frequency lock-in phenomenon 
for this simplified fluid-structure system. This VIV model problem 
manifests a complex dynamical behavior, which is still subject of active research 
over the past decade \citep{williamson2004,bearman2011}.
%
Apart from the fundamental physics 
of a single cylinder VIV \citep{Blackburn1999,Shiels2001,Singh2005,Leontini2006}, 
the topics for numerical investigations range from the development of coupling procedures for 
the Navier-Stokes and the structural equations 
(\cite{He2012}, \cite{Jaiman2015}, \cite{Jaiman2016a}, \cite{Jaiman2016b}), 
to the modeling of near wall proximity effects \citep{Tham2015}, 
multiple-cylinder arrangements \citep{Mysa2016,Liubin2016}, 
and suppression devices \citep{Yu2015,Law2017}. 
High-fidelity computational fluid dynamics (CFD) can reveal a vast amount of 
physical insight in terms of vorticity distribution, the force dynamics, 
the frequency characteristics and phase relations, 
and the shape of VIV trajectory. Despite improved algorithms and powerful supercomputers,
the state-of-art CFD-based VIV simulation 
is less attractive with regard to parametric optimization 
and the development of control strategies.
%
The primary motivation behind the present work is: 
(i) to develop an efficient low-order model for the VIV lock-in of
a circular-shaped bluff body, and (ii) to generalize the eigenvalue analysis of 
VIV lock-in mechanism for other two-dimensional bluff bodies.

\subsection{VIV mechanism and control}
%% VIV mechanism 
A simple interpretation of frequency lock-in during VIV is attributed 
to the classical resonance or synchronization with a well-defined frequency. 
Structural response amplitude gradually should grow as the structure 
natural frequency $f_{N}$ approaches 
the alternate vortex shedding frequency $f_{vs}$, and should attain its 
maximum value when $f_N/f_{vs} \approx 1$. 
However, VIV simulations (\cite{Singh2005,Tham2015}) at $Re=100$ reveal that 
the circular cylinder acquires the maximum amplitude at $f_N/ f_{vs} \approx 1.3$ or in the vicinity 
of VIV lock-in onset, which is not consistent with the simple resonance interpretation. 
Therefore, the classical resonance is not adequate to interpret the underlying VIV lock-in mechanism 
and the large amplitude during the lock-in process. 
%
Through a linear global stability analysis of the flow past an elastically-mounted 
cylinder, \cite{cossu2000} identified two modes in the fluid-structure 
system, namely nearly structural mode and the von Karman wake mode.
% 
\cite{DeLangre2006} investigated the mechanism of lock-in to a coupled mode 
flutter by using a simple linear wake-oscillator model 
for a transversely vibrating circular cylinder. 
%
The VIV analysis in \cite{DeLangre2006} was performed 
by considering an empirical wake oscillator model while 
neglecting nonlinear and viscous terms. 
%
Analogous to plunging and pitching instability of airfoil in the classical aeroelasticity, 
\cite{DeLangre2006} attributed the root cause of VIV lock-in to the mode coupling between 
the transverse periodic motion and the continuous rotation of the separation point 
along the smooth contour of circular cylinder. 


Using a standard asymptotic analysis, \cite{meliga2011} confirmed the existence of 
the two modes identified by \cite{cossu2000} and termed them as the wake mode (WM) 
and structure mode (SM). For weak fluid-structure interaction in the limit 
of large solid-to-fluid mass ratio, the eigenvalue of wake mode was found to be similar 
to the leading eigenvalue computed for the flow past a fixed cylinder whereas
the eigenvalue of the SM approached to the natural eigenvalue
of the cylinder-only system.
%It is common practice to utilize linear reduced order 
%model (ROM) for aeroelastic system stability analysis (\cite{Silva2005}). 
Inspired by the semi-analytical finding of \cite{DeLangre2006},
\cite{Zhang2015} recently employed 
a linear ROM-based CFD method 
to provide further evidence of the frequency lock-in phenomenon 
of circular cylinder at $Re=60$, and two regimes have been confirmed 
in the VIV response, namely \emph{resonance-induced lock-in} and \emph{flutter-induced lock-in}. The 
resonance regime is related to the vorticity dynamics of wake flow, whereas the 
flutter regime may be interpreted as an inertial coupling between 
the structure and global wake flow.
% Mittal paper 
In another recent work of \cite{mittal2016}, the lock-in phenomenon has been 
investigated via a linear stability and direct time integration and two leading 
eigenmodes referred to as the fluid mode and the elastic mode were classified 
for a transversely vibrating circular cylinder. These two leading 
modes were found to have a strong coupling for low mass ratios and a clear 
demarcation of the fluid (wake) mode or elastic (structure) mode was found to be 
non-trivial. As opposed to the decoupled modes (WM and SM) for high mass ratios, 
these modes were termed as coupled modes for low mass ratios \citep{mittal2016}.


Owing to the complexity of VIV with regard to fluid-structure interaction, 
a unified description of the frequency lock-in still remains unclear 
for arbitrary shaped bluff bodies and general physical conditions.
%
Of particular interest of this study is to understand some elementary aspects of the self-sustained VIV oscillations  by considering a linear aspect of the lock-in process. 
The linear instability plays a key role in the origin of self-sustain VIV oscillation arising 
from coupled fluid-structure system. Once the fluid-structure system rises to  
a high-amplitude VIV response, 
the nonlinearity begins to dominate and the system transforms 
into a fully developed (self-limiting) limit-cycle state.
%
Some key questions with regard to the generality of 
VIV lock-in process have remained unexplained, such as:
How does the geometry of bluff body influence the frequency lock-in in VIV? Why the VIV behavior of square cylinder is different from its circular counterpart? 
Do the resonance and flutter regimes exist always or the regimes are actually 
influenced by the Reynolds number and the geometry of bluff body? 
In this article, we attempt to answer these questions 
and understand more general aspects of linear VIV mechanism 
via our proposed ERA-based ROM procedure.     
%

An understanding of VIV mechanism can help in developing flow control techniques 
based on both passive and active control schemes, 
whereby the passive schemes require no energy input and  the active schemes 
rely on continuous energy input. 
%
Owing to the complexity of VIV, the control schemes are generally 
ad-hoc and a good understanding of the dynamical behavior 
with respect to the flow and structure parameters is required.
%
%% linear stability analysis 
%As discussed earlier, the VIV is a nonlinear unsteady phenomena and is characterized by alternate vortex shedding. 
Although a high-fidelity CFD model is able to resolve physical feature of interest, a linear model based on the model reduction provides a way to perform stability analysis for the flow past a bluff body and to design active control strategies (\cite{Marquet2008}, \cite{Thompson2014}, \cite{Mettot2014}, and \cite{Flinois2016}). 
Two ways exist to derive a linear model of original nonlinear system. 
While the first one is to derive a linear governing equation and then discretize the system of equations, the second approach is to discretize the nonlinear model first and 
then to obtain the linear model from it. 
The latter method is widely used in the aeroelastic research community to construct the linear model by automatic differencing method. 
However, both types of methods are expensive and are not attractive 
for parametric study and the development of VIV control strategies. 
%
A low-order model based on minimal state-space dimension has a potential to 
become a practical alternative to understand the VIV mechanism and 
to design a proper control procedure. A model-based control design can help to 
regulate and stabilize alternate vortex formation and the near-wake dynamics.
%
Such model relies on the smallest state-space dimension of realized systems that 
have the similar input-output relations within a specified degree of precision. 
As shown in \cite{HoKalman1966} that the minimum problem represents  
the problem of identifying the sequence of real matrices, also known as the Markov parameters, 
based on the impulse response of a dynamic system.
%

\subsection{Model order reduction}
% Model reduction 
Model order reduction (MOR) technique is to approximate original full order (high dimensional) 
system with a low order model, which retains the significant dynamics of 
the original system and provides order of magnitude efficiency improvement 
to construct the essential dynamics of the system. 
%
As discussed in \cite{Flinois2016}, we can categorize the previous studies on the linear model 
order reduction into two main approaches. 
%
The first ROM construction approach is based on Galerkin projection of full 
order system onto a small subspace spanned by mode vectors. 
The mode vector can be obtained by proper orthogonal decomposition (POD), balanced 
truncation (\cite{Moore1981}), or dynamic mode decomposition 
(DMD)(\cite{rowley2009,schmid2010}).
%
One of the drawbacks of conventional POD/Galerkin models is that while they capture 
the most energetic modes based on a user-defined energy norm, low-energy features may be
crucial to the dynamics of underlying problem.
%
As compared to the POD method, which only extracts modes 
from the snapshots of the primary system, the balance truncation method derives the modes 
by collecting the snapshots of both primary and adjoint systems. This feature of 
the balance truncation method allows to identify the modes which are dynamically important. 
Based on the work of \cite{Moore1981}, \cite{Willcox2002} and \cite{Rowley2005} further extended the balance truncation concept 
to a large system by approximating the system observable and controllable Gramians 
via two sets of snapshots from a linearized forward simulation 
and a companion adjoint simulation. 
This algorithm is usually referred to as the balanced proper orthogonal decomposition (BPOD) 
and provides two sets of modes, namely primal and adjoint modes. 
%The BPOD retains the most dynamically relevant observable and controllable modes and usually is considered superior over the POD method
%{\color{red} as far as linear model is considered. }

%A linear representation of flow equations is typically 
%necessary to apply Galerkin projection, which can be quite cumbersome 
%to obtain from the existing flow solver.
%
The second approach is based on the system identification method, which  only requires 
input and output information and considers the original system as a black box via input-output 
dynamical relationship. From a time-domain formulation and the realization of a state-space 
model, a ROM of dynamic system can be constructed on the basis of input-output data.
%
One of widely used system identification methods is the eigensystem realization algorithm (ERA)
introduced by \cite{Juang1985} for the model reduction using a 
Hankel matrix based decomposition.
ERA essentially extends the well-known algorithm of \cite{HoKalman1966} in control theory  
and creates a minimal realization that follows the evolution of system output 
when it is subjected to an impulse input.
%
In a recent theoretical study, \cite{Ma2011} proves that 
the ERA constructs a ROM which is mathematically equivalent to the BPOD method. 
With regard to recent fluid dynamics applications, the ERA has been considered for unsteady problems by \cite{Flinois2016,Yao2015}.

The aforementioned methods are originally developed for stable linear systems. 
%
Extensions  have been made to circumvent this restriction 
of the model reduction for unstable systems by either 
partitioning the unstable and stable subspace or inverting the large linear system 
\citep{barbagallo2009,Ahuja2010,Dergham2011}.
%
In one of the recent work by \cite{Flinois2015}, 
a theoretical analysis was presented to show 
that the unmodified balance truncation (designed for stable systems)
method can be applied to an unstable system.
%
Following this analysis and the work of \cite{Ma2011}, the ERA is recently employed 
for the active control of unstable wake behind a bluff body (\cite{Flinois2016}). 
Compared to the ROM method used in \cite{Zhang2015}, 
which lacks a mathematical rigor and is highly sensitive to training trajectory, 
the ERA has a theoretical foundation for unstable linear systems generated by 
the unsteady wake dynamics and vortex-induced vibrations. 
Therefore, following \cite{Flinois2015} and the finding of \cite{Ma2011}, 
the ERA is adopted in this paper to construct the low-order fluid model.


\subsection{Contributions and organization}
In this work, we present a physical insight and the 
underlying mechanism of vortex-induced vibration 
by exploiting a unified description of frequency lock-in during elastically mounted cylinders. 
We introduce the ERA-based ROM  to capture just enough physics 
to extract the stability properties of the fluid-structure system
of two-dimensional bluff bodies consisting of sharp corners and smooth curves. 
%
Of particular interest is to provide a generalized description 
of these frequency lock-in regimes at low Reynolds numbers via model reduction technique. 
Unlike the wake-oscillator model, the present technique does not rely on any empirical formulation and captures naturally the physical effects related to the 
added mass and damping forces through the solution of the Navier-Stokes equations.
%based on input-output dynamics.
%%
%We construct the low-order model for fluid-structure interaction based on the ERA (\cite{Juang1985}).
We first employ the ERA-based ROM for unstable wake flow over a stationary circular cylinder 
and predict the critical Reynolds number $Re_{cr}$ of vortex shedding. 
%%
We then perform the stability analysis of fluid-structure system via the ERA-based ROM to analyze
the effects of Reynolds number $Re$, the mass ratio $m^*$ and the rounding of square cylinder.
To examine the accuracy and reliability of the low-order model, we assess the ROM results 
against the full-order simulations 
%based on the body-conforming Lagrangian-Eulerian coupling for fluid-structure interaction. 
%The full-order simulations are 
performed by the variationally coupled Navier-Stokes and rigid body equations.

%
We will show in this paper that the two frequency lock-in regimes associated 
with \emph{resonance} and \emph{flutter} characteristics 
only exist when certain conditions are satisfied. These regimes 
have a strong dependence on the shape of the bluff body, the Reynolds number and the mass ratio.
The presence of sharp corners on a square cylinder largely alters the VIV lock-in
characteristics as compared to the circular counterpart with smooth curves.
%
We report that the frequency lock-in of the square cylinder 
is found to be dominated by the \emph{resonance} regime without any coupled mode 
flutter at low Reynolds number ($Re \le 80$). This indicates that the previous theoretical finding by \cite{DeLangre2006} on the root cause of frequency lock-in due to
the coupled flutter does not hold for a transversely vibrating sharp-cornered square cylinder.
%
Apart from the frequency lock-in regimes, we qualitatively 
visualize the spatio-temporal evolution of vortex shedding and leading eigenmodes 
to link the lock-in process with the intrinsic wake dynamics.
%%
To understand the influence of geometry on the frequency lock-in regimes,
we present a stability phase diagram for five two-dimensional bluff bodies 
namely, circle, square, ellipse, forward triangle and diamond.
%
Compared to the circular cylinder, we show that the flutter mode is pronounced in the elliptical cylinder  
while the lock-in/synchronization is galloping-dominated for the forward triangle configuration.
%
The proposed ERA-based ROM is general and efficient for 
fluid-structure systems without the need for a linearized flow or an adjoint solver, 
which allows the method to be even applicable for physical experiments. 

%
The paper is structured as follows: Section 2 introduces the full order model, the 
state-space formulation for the model reduction and 
the eigensystem realization algorithm for the wake flow and vortex-induced vibration.
Section 3 describes the VIV problem set-up and presents the convergence and  
the numerical verification of the ERA-based ROM model. 
The systematic analysis of frequency lock-in mechanism 
as a function of Reynolds number and the effects of rounding and geometry are provided in Section 4. 
Concluding remarks are presented in Section 5. 

\section{Numerical methodology}\label{sec:method}
For the sake of completeness, we first summarize the formulation for high-dimensional 
full order model (FOM) 
and describe the implementation of the numerical schemes used for the coupled variational fluid-structure solver. 
Later we present the ERA 
for the construction of reduced-order model (ROM) using the Navier-Stokes 
and rigid-body equations.
\subsection{Full order model formulation}
To study the interaction of elastically mounted cylinder with the fluid, 
we consider a variational fluid formulation based on 
the arbitrary Lagrangian-Eulerian (ALE) description 
and the semi-discrete time stepping \citep{Jaiman2015,Liu2014}. 
%
Consider the fluid domain $\Otf$ with the spatial and temporal coordinates 
denoted by $\xx$ and $t$, respectively.
%
The Navier-Stokes (NS) equations governing an incompressible flow  
in the ALE reference frame are
%
\begin{align}
\rho^\mathrm{f} \left( \frac{\partial \bu^\mathrm{f}}{\partial t} \bigg\rvert_{\bchi} + \left(\bu^\mathrm{f}-\bw\right)\cdot \boldsymbol{\nabla} \bu^
\mathrm{f} \right )= \boldsymbol{\nabla} \cdot \boldsymbol{\sigma}^\mathrm{f} + \mathbf{b}^\mathrm{f} \mbox{ on } \Otf \label{eq:N-S}, \\
\boldsymbol{\nabla} \cdot\bu^\mathrm{f} = 0 \mbox{ on } \Otf \label{eq:continuity},
\end{align}
where $\rho^\mathrm{f}$, $\bu^\mathrm{f}$, 
$\bw$, $\boldsymbol{\sigma}^\mathrm{f}$, and $\mathbf{b}^\mathrm{f}$ 
are the fluid density, the fluid velocity, the ALE mesh velocity, the 
Cauchy stress tensor and the body force per unit mass, respectively. 
%
For the partial time derivative in Eq. (\ref{eq:N-S}), 
the ALE referential coordinate $\bchi$ is held fixed and for a Newtonian fluid
$\boldsymbol{\sigma}^\mathrm{f}$ is defined as
\begin{equation}
\boldsymbol{\sigma}^\mathrm{f} = -p\mathbf{I} + \mu^\mathrm{f}\left(\boldsymbol{\nabla} \bu^\mathrm{f} + \left(\boldsymbol{\nabla} \bu^\mathrm{f}\right)^T
\right),
\label{eq:cauchyStress}
\end{equation}
%
where $p$, $\mu^\mathrm{f}$ and $ \mathbf{I} $ are the  pressure, the dynamic viscosity of the fluid and an identity tensor, respectively.
A rigid-body structure submerged in the fluid experiences unsteady fluid forces
and consequently may undergo flow-induced vibrations if the body is mounted elastically. 
To simulate translational motion of a two-dimensional rigid body about its center of mass, the Lagrangian motion along the Cartesian axes is given by:
\begin{equation}
\mathbf{m} \cdot\frac{d {\bu}^\mathrm{s}}{d t}+ \mathbf{c} \cdot{\bu}^\mathrm{s}+\mathbf{k} \cdot\left(\boldsymbol{\varphi}^\mathrm{s}
\left(t\right)-\boldsymbol{\varphi}^\mathrm{s}
\left(0\right)\right)=\mathbf{F}^\mathrm{s},
\label{eq:rigid}
\end{equation}
\begin{align}
{\bu}^\mathrm{s}\left(t\right) = 
\frac{\partial \vec{\varphi}^\mathrm{s}}{\partial t}, 
\end{align}
where ${\bu}^\mathrm{s}\left(t\right)$ represents the velocity of immersed rigid body in the fluid domain, $\vec{\varphi}^\mathrm{s}\left(t\right)$ denotes
the position of the center 
of the rigid body at time $t$, and
$\mathbf{m},\ \mathbf{c}$, $ \mathbf{k}$ are mass, damping and stiffness coefficient matrices for the translational motions. Here,
 $\mathbf{F}^\mathrm{s}$ is 
 fluid force on the rigid body.  
 %and $\Ots$ represents the domain occupied by the rigid body.

%The fluid and the structural equations are coupled by the continuity of velocity
%and traction along the fluid-structure interface.
%The new position of rigid body is updated through a position vector $\boldsymbol{\varphi}^\mathrm{s}$, which maps the initial position $\mathrm{\mathbf{z}}_0
%$ of the rigid body to its new position at time $t$. 
%Let $\Gamma=\partial\Of(0) \cap \partial\Os(0)$ be the fluid-structure
%interface at $t=0$ and
Let $\G(t)$ is the fluid-structure interface at time $t$,
the coupled system requires to satisfy the continuity of velocity and 
the force equilibrium at the fluid-body interface $\Gamma$ as follows
\begin{align}
{\bu}^\mathrm{f}\left(t\right) = {\bu}^\mathrm{s}\left(t\right) , 
\label{eq:bcsVelocity} \\
\int_{\G(t)} \boldsymbol{\sigma}^\mathrm{f}\left(\xx,t\right)\cdot\boldsymbol{\mathrm{n}} \mathrm{d} \Gamma
+ \vec{F}^\mathrm{s} =0 ,
\label{eq:bcsTraction}
\end{align}
where $\boldsymbol{\mathrm{n}}$  is the outer normal to the fluid-body interface.
%$\gamma$ is any part of the fluid-body interface $\Gamma$ in the reference configuration, 
%$\mathrm{d} \Gamma$ denotes a differential surface
%area and ${\boldsymbol{\varphi}^\mathrm{s}(\gamma,t)}$ is the corresponding fluid part at time $t$. 
In Eq. (\ref{eq:bcsTraction}), the first term represents 
the force exerted by the fluid  
and the second term is the solid load vector applied in Eq.~(\ref{eq:rigid}).
%
The ALE mesh nodes on the fluid domain $\Omega^f(\xx,t)$ can be updated by solving a linear steady pseudo-elastic material model
\begin{align}
\label{MeshEq}
\div \cdot \stress^\mathrm{m} = \vec{0}, \qquad \stress^\mathrm{m} = (1+k_\mathrm{m})\left[\left(\div \vec{\eta}^\mathrm{f}+\left(\div
\vec{\eta}^\mathrm{f}\right)^T\right)+\left(\div \cdot \vec{\eta}^\mathrm{f}\right)\vec{\mathrm{I}} \right], 
\end{align} 
where $\stress^\mathrm{m}$ is the stress experienced by the ALE mesh due 
to the strain induced by the rigid-body movement, 
$\vec{\eta}^\mathrm{f}$ represents the ALE mesh node displacement and 
$k_\mathrm{m}$ is a mesh stiffness variable chosen as a function of the element area to 
limit the distortion of small elements located in the immediate vicinity 
of the fluid-body interface.

%
The weak variational form of Eq.~(\ref{eq:N-S}) is discretized in space 
using $\mathbb{P}_{n}/\mathbb{P}_{n-1}$ iso-parametric finite elements 
for the fluid velocity and pressure, where 
$\mathbb{P}_{n}$ denotes the standard $n^\mathrm{th}$
order Lagrange finite element space on the discretized fluid domain. 
%
To satisfy the inf-sup condition, $\mathbb{P}_{2}/\mathbb{P}_{1}$ finite element 
mesh is adopted and the second-order backward scheme is used for the time discretization 
of the Navier-Stokes system \citep{Liu2014}. 
In the present study, a partitioned staggered scheme is considered for the 
full-order simulations of fluid-structure interaction \citep{Jaiman2011}.
The motion of structure is driven by the traction forces exerted 
by the flowing fluid at the fluid-structure interface $\Gamma$, whereby 
the structural motion predicts the new interface position and the geometry changes 
for the ALE fluid domain at each time step. The movement of
the internal ALE fluid nodes is accommodated such that the mesh quality 
does not deteriorate as the motion of solid structure becomes large.
%
The above coupled variational formulation completes the presentation 
of full order model for the direct numerical simulation of fluid-structure interaction. 
We next present a state-space formulation of the model reduction using 
a system identification technique based on the input-output dynamics.
%
\subsection{Basic state-space formulation and model reduction}
The linear time-invariant (LTI) and multiple-input multiple-output (MIMO) model represented in a state-space form 
at discrete times $t=k\Delta t$, 
$k=0,1,2,...,$ with a constant sampling time $\Delta t$ reads 
\begin{equation}
\left. \begin{array}{ll}

\displaystyle \mathsfbi{x}_{r}(k+1)=\mathsfbi{A}_{r}\mathsfbi{x}_{r}(k)+\mathsfbi{B}_{r}\mathsfbi{u}(k)  \\[8pt]

\displaystyle \mathsfbi{y}_{r}(k)=\mathsfbi{C}_{r}\mathsfbi{x}_{r}(k)+\mathsfbi{D}_{r}\mathsfbi{u}(k) 
\end{array}\right\},
 \label{eq:ERA_ROM}
\end{equation}  
where $\mathsfbi{x}_r$ is an $n_r$-dimensional state vector, 
$\mathsfbi{u}$ denotes a $q$-dimensional input vector 
and $\mathsfbi{y}_r$ is a $p$-dimensional output vector.  
The integer $k$ is a sample index for the time stepping.
%
The system matrices are $\left(\mathsfbi{A}_r,\mathsfbi{B}_r,\mathsfbi{C}_r,\mathsfbi{D}_r\right)$, 
whereby the transition matrix $\mathsfbi{A}_r$ characterizes the dynamics of the system.
Here, $\mathsfbi{B}_r$, 
$\mathsfbi{C}_r$ and $\mathsfbi{D}_r$ denote the input, output and 
feed-through matrices, respectively. 
For the given output vector $\mathsfbi{y}_r$, the statement of system realization 
is to construct the system matrices $\left(\mathsfbi{A}_r,\mathsfbi{B}_r,\mathsfbi{C}_r,\mathsfbi{D}_r \right)$,
such that the vector $\mathsfbi{y}_r$ is reproduced by the state-space model.
In a discrete-time setting,
the state-space realization matrices $\left(\mathsfbi{A}_{r},\mathsfbi{B}_{r},\mathsfbi{C}_{r},\mathsfbi{D}_{r} \right)$ 
of the dynamical system are constructed by the ERA, 
in which only the impulse response function (IRF) of the original full order system 
is required for the system realization. The impulse response of full-order linear system is 
first defined as 
$\mathsfbi{y}=\left[\mathsfbi{y}_1,\mathsfbi{y}_2,\mathsfbi{y}_3,...,\mathsfbi{y}_{n_i}\right]$, 
where $n_i$ represents the length of the impulse response 
and $\mathsfbi{y}_i$ denotes IRF with the dimension $p \times q$ . 
Based on the impulse response, the generalized block Hankel matrices $r\times s$  
can be constructed as
%%%%%%%%%
\begin{equation}
\mathsfbi{H}(k-1) = 
\left[ {\begin{array}{*{20}c}
    \mathsfbi{y}_{k+1}&     \mathsfbi{y}_{k+2}& ...& \mathsfbi{y}_{k+s}        \\    
    \mathsfbi{y}_{k+2}& \mathsfbi{y}_{k+3}& ...& \mathsfbi{y}_{k+s+1}    \\
    \vdots& \vdots& \ddots& \vdots        \\
    \mathsfbi{y}_{k+r}& \mathsfbi{y}_{k+r+1}& ...&\mathsfbi{y}_{k+(s+r-1)}
 \end{array} } \right].
\label{eq:Hankel}
\end{equation}  
From the partitioned SVD of the Hankel matrix $\mathsfbi{H}(0)$, 
we can have
\begin{equation}
\mathsfbi{H}(0) = \mathsfbi{U}\mathsfbi{\Sigma}\mathsfbi{V}^*=
\left[\mathsfbi{U}_1 \quad \mathsfbi{U}_2\right]
\left[{\begin{array}{*{20}c}
   \mathsfbi{\Sigma}_1 \quad \mathsfbi{0} \\
   \mathsfbi{0} \quad \mathsfbi{\Sigma}_2
 \end{array} }
\right]
\left[{\begin{array}{*{20}c}
  \mathsfbi{V}_1^* \\
  \mathsfbi{V}_2^*
 \end{array} }
\right],
\end{equation}
where the diagonal matrix $\Sigma$ are the Hankel singular values (HSVs) $\sigma_{i}$, 
which represents the dynamical significance through sorting 
such that $\sigma_{1} \geq \ldots \geq \sigma_{n} \geq 0$. 
The block matrix $\Sigma_2$ contains the zeros or 
negligible elements. By truncating the dynamically less significant states, we estimate 
$\mathsfbi{H}(0) \approx \mathsfbi{U}_1\mathsfbi{\Sigma}_1\mathsfbi{V}_1^*$.  
The reduced system matrices
$\left(\mathsfbi{A}_{r},\mathsfbi{B}_{r},\mathsfbi{C}_{r},\mathsfbi{D}_{r}\right)$ are then defined as
\begin{equation}
\left. \begin{array}{ll}

\displaystyle \mathsfbi{A}_{r}=\Sigma^{-1/2}_{1}\mathsfbi{U}^{*}_{1}\mathsfbi{H}(1)\mathsfbi{V}_{1}\Sigma^{-1/2}_{1} \\
\displaystyle \mathsfbi{B}_{r}=\Sigma^{1/2}_{1}\mathsfbi{V}^{*}_{1}\mathsfbi{E_m}   \\
\displaystyle \mathsfbi{C}_{r}=\mathsfbi{E_t}^{*}\mathsfbi{U}_{1}\Sigma^{1/2}_{1}   \\    
\displaystyle \mathsfbi{D}_{r}=\mathsfbi{y}_{1}
\end{array}\right\}.
 \label{eq:ERA_ROM_matrices}
\end{equation}  
%
Here, $\mathsfbi{E_m}^*=\left[\mathsfbi{I}_q \quad \mathsfbi{0} \right]_{q\times N}$,
$\mathsfbi{E_t}^{*}=\left[\mathsfbi{I}_p \quad \mathsfbi{0} \right]_{p \times M}$, where
$N =s \times q$, $M =r \times p$, $\mathsfbi{I}_{p}$ and $\mathsfbi{I}_{q}$ are the identity matrices. 
%
We next present the ERA to construct the fluid ROM, which relies 
on the incompressible NS equations to represent the dynamics of a
small amplitude perturbation around the equilibrium base flow. 


\subsection{ERA-based model reduction for VIV}
In the present work, we only consider the transverse motion of cylinder in a flowing stream for the sake of simplicity.
However, the formulation of ERA-based ROM is general for any fluid-structure system.
%The fluid and the structure are coupled by the fact that the fluid velocity must 
%be equal to the velocity of the cylinder at its surface. 
%
The cylinder is mounted on a spring system in the cross-flow direction, 
which allows the cylinder to vibrate through unsteady lift comprising 
of the pressure and shear stresses of the fluid. Owing to the direct solution of 
the Navier-Stokes equations, the effects of added-mass and added damping forces are 
implicitly captured in the present model.
%
To perform linear stability analysis, the fluid ROM constructed by ERA is coupled with 
the linear structural model. 
The nondimensional structural equation for a transversely vibrating cylinder with 
one-degree-of-freedom can be written as  
\begin{equation}
{\ddot{Y} + 4\zeta\pi F_{s}\dot{Y} + (2\pi F_{s})^2 Y}=\frac{a_{s}}{m^{*}} C_{l},
\label{eq:structure1}
\end{equation}
where $Y$ is the transverse displacement; $C_{l}$ is the lift coefficient, 
$m^{*}$ and $\zeta$ are the ratio of the mass of the vibrating structure to the mass of the displaced fluid and the 
damping coefficient, respectively; $F_{s}$ is the reduced natural frequency of the structure 
defined as $F_{s}=f_{N}D/U = 1/U_r$, where $U_r$ is the reduced velocity which
is alternative parameter to describe the frequency lock-in phenomenon. 
Mass ratio $m^*$ is a key parameter for VIV lock-in and it is 
defined as the ratio of vibrating structure  to the mass of displaced fluid.
%
The characteristics length scale factor  $a_{s}$ 
is related to the geometry of body. For example, the values are 
$a_{s}=\frac{2}{\pi}$ for a circular cylinder, and  $a_{s}=0.5$ 
for a square cylinder. There exists a complex dynamical relation between 
the transverse amplitude $Y$ and the lift force $C_{l}$. 
%The influence of cylinder amplitude 
%on the wake force and the added-mass force is non-trivial and highly nonlinear.
%
One of the main objectives of this work is to construct a state-space relationship between 
the transverse force and the amplitude directly from the 
NS equations subject to an impulse. We next proceed to the model reduction
of fluid-structure system.

The non-dimensional structural Eq. \mbox{(\ref{eq:structure1}) }
can be cast into a state-space formulation as
\begin{equation}
 \dot{\mathsfbi{x_{s}}}=\mathsfbi{A_{s}}\mathsfbi{x_{s}}+\mathsfbi{B_{s}}C_l,
\label{eq:state}
\end{equation} 
where the state matrices and vectors are
\[ 
	\mathsfbi{A_{s}}=\left[ \begin{array}{cc}
	0 & 1\\
	 -(2\pi F_{s})^2 &-4\zeta\pi F_{s}      
 	\end{array} \right], \quad
 	\mathsfbi{B_{s}}=\left[ \begin{array}{c}
 	0 \\
 	\frac{a_{s}}{m^{*}}
	\end{array}  \right], \quad
	\mathsfbi{x_{s}}=\left[ \begin{array}{c}
 	Y \\
 	\dot{Y}
	\end{array}  \right].
 \] 
It is straightforward to solve Eq. (\ref{eq:state}) using a standard time-integrator (\cite{Yao2016_JFS}). In the present work, an exact state-space discretization of Eq. (\ref{eq:state}) is considered as follows:
\begin{equation}
\left. \begin{array}{ll}
\displaystyle \mathsfbi{x_{s}}(k+1)=\mathsfbi{A_{sd}}\mathsfbi{x_{s}}(k)+\mathsfbi{B_{sd}}C_l(k),  \\[8pt]
\displaystyle Y(k)=\mathsfbi{C_{sd}}\mathsfbi{x_{s}}(k)
\end{array}\right\},
\label{eq:discrete_structure}
\end{equation}
where the state matrices are $\mathsfbi{A_{sd}}=e^{\mathsfbi{A_{s}}\Delta t}$, $\mathsfbi{B_{sd}}={\mathsfbi{A_{s}}}^{-1}(e^{\mathsfbi{A_{s}}\Delta t}-\vec{\mathrm{I}})\mathsfbi{B_s}$ 
at discrete times $t=k\Delta t$, 
$k=0,1,2,...,$ with a constant sampling time $\Delta t$, $\vec{\mathrm{I}}$ is 
an identity matrix with the same size of $\mathsfbi{A_{s}}$, and 
$\mathsfbi{C_{sd}}=[1 \quad 0]$. 
%
Through the input-output dynamics, the fluid ROM is derived by the ERA method 
as described in Eq. (\ref{eq:ERA_ROM}). 
The input for the ROM  is the transverse displacement $Y$, and the output 
is the lift coefficient $C_l$. The ERA-based ROM with the single-input and single-output (SISO) can reformulated as:
\begin{equation}
\left. \begin{array}{ll}

\displaystyle \mathsfbi{x}_{r}(k+1)=\mathsfbi{A}_{r}\mathsfbi{x}_{r}(k)+\mathsfbi{B}_{r}Y(k),  \\[8pt]

\displaystyle C_l(k)=\mathsfbi{C}_{r}\mathsfbi{x}_{r}(k)+\mathsfbi{D}_{r}Y(k).
\end{array}\right\}
 \label{eq:ERA_ROM2}
\end{equation}  
%
Substituting Eq. (\ref{eq:ERA_ROM2}) to Eq. (\ref{eq:discrete_structure}), the resultant ROM can be expressed as
\begin{equation}
 \mathsfbi{x_{fs}}(k+1)=\left[ \begin{array}{cc}
 \mathsfbi{A_{sd}}+\mathsfbi{B_{sd}}\mathsfbi{D_{r}}\mathsfbi{C_{sd}} & \mathsfbi{B_{sd}}\mathsfbi{C_{r}} \\
 \mathsfbi{B_{r}}\mathsfbi{C_{sd}} & \mathsfbi{A_{r}}
 
\end{array}  \right]\mathsfbi{x_{fs}}(k) =\mathsfbi{A_{fs}}\mathsfbi{x_{fs}}(k),
\label{eq:vivrom}
\end{equation} 
where $\left(\mathsfbi{A}_{r},\mathsfbi{B}_{r},\mathsfbi{C}_{r},\mathsfbi{D}_{r}\right)$ 
are the ROM matrices defined by the ERA method as given in Eq. (\ref{eq:ERA_ROM}) 
and $\mathsfbi{A_{fs}}$ denotes the coupled fluid-structure matrix in the discrete state-space form,
and $\mathsfbi{x_{fs}}=[\mathsfbi{x_{s}} \quad \mathsfbi{x}_r]^T$.
%

The present ERA-based ROM reproduces the 
the input-output dynamics of the full order system.
%
The linear stability analysis of VIV system can be 
expressed into an eigenvalue problem of Eq. (\ref{eq:vivrom}). 
The eigenvalue distribution of coupled fluid-structure 
matrix $\mathsfbi{A_{fs}}$ characterizes the stability of VIV system.   
($\lambda, \widehat{\mathsfbi{x}}$) correspond to continuous-time 
eigenvalue/eigenvectors 
of $\mathsfbi{A_{fs}}$, whereby the spatial structure is characterized by the complex 
vector field $\widehat{\mathsfbi{x}}$ and their temporal behavior by the complex scalar $\lambda$.
%
The stability analysis can be easily accomplished by tracing the trajectory of complex eigenvalue 
$\lambda$ in the complex plane, whereby  $\widehat{\mathsfbi{x}}$ provides the spatial global modes 
of the ROM.
%
Based on the leading global mode or least damped eigenvalue of ERA-based ROM, 
we define growth (amplification) rate $\sigma$=Re$(\lambda)$ and frequency $f$=Im$(\lambda/2\pi)$.
%
The construction of the above ERA-based ROM model is computationally efficient as it only relies on 
the impulse response of the FOM.
While the aforementioned formulation is presented for the transverse-only vibration of 
the structure, it is general for any coupled fluid-structure system.
%
After reviewing the mathematical formulation and ERA-based ROM technique, 
we next present the numerical set-up and verification of our solver.

%%%%%%%% CONVERGENCE STUDY

\section{Numerical set-up and verification}\label{sec:convergence}

\subsection{Problem definition}\label{sec:prob}
Figure \ref{fig:schematic} shows a schematic diagram of the setup used in our simulation study 
for an elastically mounted bluff body with various cross-sections in a flowing stream. 
The coordinate origin is located at the geometric center of the bluff body.
The streamwise and transverse directions are denoted $x$ and $y$, respectively.
%
A stream of incompressible fluid enters into the domain from an 
inlet boundary $\mathrm{\Gamma_{in}}$ at a horizontal velocity $(u,v)=(U,0)$, where $u$ and $v$ 
denote the streamwise and transverse velocities, respectively. 
The bluff body with mass $m$ and characteristic diameter $D$ is mounted on a linear spring 
in the transverse direction. The damping coefficient $\zeta$ is set to zero in the present work.
%
The computational domain and the boundary conditions are also illustrated 
in figure \ref{fig:schematic}. 
No-slip wall condition is implemented on the surfaces of the bluff body, and a traction-free
boundary condition is implemented along the outlet $\mathrm{\Gamma_{out}}$
while the slip wall condition is implemented on the top $\mathrm{\Gamma_{top}}$
and bottom $\mathrm{\Gamma_{bottom}}$ boundaries.
%
The numerical domain extends from $-10D$ at the inlet to $30D$ at the outlet, 
and from $-15D$ to $15D$ in the transverse direction.
%
Except stated otherwise, all positions and length scales are normalized by 
the characteristic dimension $D$, velocities
with the free stream velocity $U$, and frequencies with $U/D$.
%
The Reynolds number $Re$ of flow is based on the
characteristic dimension $D$, kinematic viscosity of fluid 
and free-stream speed $U$.

%%%%%%%%%%%%%%%%%%%%%%%%%%%%%%%%%%%%%%%%%%%%%%%%%%%%%%%%%%%%%%%%%%%%%
\begin{figure}
	 \centering
	 \includegraphics[scale=0.8]{fig1}
     \caption{Schematic diagram of a representative bluff body of 
     elastically mounted cylinder in uniform horizontal flow. 
     Computational domain and boundary conditions are shown.}
\label{fig:schematic}
\end{figure}
%%%%%%%%%%%%%%%%%%%%%%%%%%%%%%%%%%%%%%%%%%%%%%%%%%%%%%%%%%%%%%%%%%%%%


%%%%%%%%%%%%%%%%%%%%%%%%%%%%%%%%%%%%%%%%%%%%%%%%%%%%%%%%%%%%%
\begin{figure}
\centering
\begin{subfigure}{0.495\textwidth}
    \includegraphics[scale=0.35]{fig2a}
    \caption{}
    \label{fig:mesh_left}
    \end{subfigure} 
\begin{subfigure}{0.495\textwidth} 
\centering
 \includegraphics[scale=0.35]{fig2b}
	\caption{}
	\label{fig:mesh_right}
	\end{subfigure}	
        \caption{A representative finite element mesh with
        $\mathbb{P}_{2}/\mathbb{P}_{1}$ discretization: 
        (a) full domain discretization and
        (b) close-up view of finite element mesh in the vicinity of the cylinder. 
        All other meshes for different bluff body geometries are similarly created. }
	\label{fig:mesh}
\end{figure}
%%%%%%%%%%%%%%%%%%%%%%%%%%%%%%%%%%%%%

\subsection{Linear stability analysis}\label{sec:linear}
We verify the validity of our ERA-based ROM by investigating 
the stability of laminar wake behind a two dimensional circular cylinder.
%
To study the mesh convergence for our simulation study, we consider 
three representative discretizations $M1$, $M2$ and $M3$ 
consisting of 9517, 22004, and 41132 $\mathbb{P}_{2}/\mathbb{P}_{1}$ 
iso-parametric elements. 
%
A representative M2 mesh is shown in figure \ref{fig:mesh} and 
the corresponding central square represents the fine mesh region around the cylinder body. 
The mesh in the cylinder wake is appropriately refined to resolve 
the alternate vortex shedding. 
%
The quality of mesh convergence is determined by the prediction of
growth rate $\sigma$ and the frequency $f$ of fluid ROM for 
the flow past a circular cylinder at $Re=60$. 
The nondimensional time step size is $\Delta t=0.05$.  
Based on the procedure described in the previous section, we next briefly describe 
the process of extracting ROMs from the incompressible NS equations.

%%%%%%%% Table for convergence 
\begin{table}
  \begin{center}
  \begin{tabular}{l c c c c}
       Mesh   & Nodes & Elements &  $\sigma $ &  $f$\\
       $M1$   & 4834 & 9514 & 0.0479 &  0.1207\\
      $M2$   & 11119 & 22004 & 0.0484 &  0.1207\\
      $M3$   & 20731 & 41132 & 0.0483 &  0.1207
  \end{tabular}
  \caption{Mesh convergence study: comparison of growth rate and frequency for meshes M1, M2 and M3 
  for the flow past a stationary circular cylinder at $Re=60$.}
  \label{tab:mesh_conv}
  \end{center}
\end{table}
 

%%%%%%%% figure for base flow
\begin{figure}
	 \centering
	 \includegraphics[scale=0.45]{fig3}
     \caption{Base flow of a stationary circular cylinder at $Re=60$; 
     streamwise velocity contours are shown.
     Contour levels are from -0.1 to 1.2 in the increment of 0.1 and the flow is from left to right.}
\label{fig:base}
\end{figure}

The ERA-based ROM is constructed in the neighborhood of the base flow, 
which is computed via fixed-point iteration 
without the time dependent term in  Eq. (\ref{eq:N-S}).
At $Re=60$ with $M2$ mesh, figure \ref{fig:base} shows the streamwise velocity 
contours of the base flow  with a symmetric recirculation bubble.
The bubble length (measured from the center of the cylinder) 
is $L_w = 4.1$, which agrees reasonably well with the literature 
($L_w = 4.2$, \cite{Luchini2007}; $L_w = 4.1$, \cite{Mao2014}).  
% ERA construction 
The Hankel matrix shown in Eq. (\ref{eq:Hankel}) is obtained 
from the output lift signal ($C_l$) subject to the impulse input of transverse displacement $Y$.
A sufficiently small amplitude ($\delta=10^{-4}$) is considered
such that the flow evolves linearly for a relatively large time.
To ensure that the unstable modes start to dominate the essential dynamics 
of the input-output relationship, an adequate number of cycles is required 
to capture the linear dynamics of the system. However, 
the excessively simulation time should be avoided before 
the nonlinearity appears via exponential growth of the unstable modes. 

The linearity of the unstable system is 
confirmed by comparing the response subject to the two impulse inputs with $\delta=10^{-4}$ and 
$\delta=10^{-3}$. A set of $1000$ impulse response outputs ($C_l$) is then stacked at each time step $\Delta t=0.05$, resulting the final simulation time $tU/D=50$. 
The adequate simulation cycles are then determined by examining the convergence of the 
fluid unstable eigenvalues computed from the Hankel matrices with the dimensions of
$500 \times 20$, $500 \times 50$, $500 \times 150$, and $500 \times 200$. 
It was found that the Hankel matrix $500 \times 150$ is sufficient, which corresponds to 
32.5 convective time units.  
The order of ERA-based ROM  is then determined by examining 
the singular values (HSVs) of the Hankel $500 \times 150$ matrix.
%
As shown in figure \ref{fig:signal}, the output lift signal $C_l$ gradually 
diverges as a function 
of convective time $tU/D$ at $Re=60$. This asymptotic divergence behavior 
indicates that the wake flow begins to lose its stability via a Hopf bifurcation, which 
breaks the symmetry of flow field and gives rise to the periodic vortex street. 
%
The first $30$ singular values are shown in figure \ref{fig:HVS}. 
The fast decaying singular values indicates that a 
low order ROM can be constructed. 
To further confirm the accuracy, the ERA-based ROM with the number of modes $n_r =25$  
is compared with the FOM result in figure \ref{fig:signal}.
A good match between the impulse response of the ROM and FOM 
can be seen in the figure. 
%
%By using the impulse response for the Hankel matrices, the ERA-based ROM can be 
%constructed from Eq. (\ref{eq:ERA_ROM_matrices}).
%%
%
It is worth noting that the Hankel matrix is not necessarily to be 
a square matrix for the suitability of ERA-based ROM \citep{Juang1985}. 
As pointed by \cite{Juang1985}, further consideration is 
required to determine optimal $r$ and $s$ in Eq. (\ref{eq:Hankel}). Therefore, the Hankel matrix can be 
tall ($r > s$), wide ($r < s$)  or square ($r = s$). 
In the present study, the dimension $r$ is fixed while $s$ is tuned 
to obtain a resonably converged unstable eigenvalue for a good 
match between the impulse response of FOM and ROM.

Table \ref{tab:mesh_conv} summarizes the comparison of the growth rate and frequency, 
which shows that the difference between $M2$ and $M3$ is less than $1\%$. 
Therefore, the mesh $M2$ is adequate for the stability analysis of VIV.
This study also confirms the convergence property of our ERA-based ROM procedure 
for unstable wake flow. 
%%%%%%%% figure for signal 
\begin{figure}
\centering 
\begin{subfigure}{0.495\textwidth}
\centering
  \includegraphics[scale=0.4]{fig4a}
    \caption{}
    \label{fig:signal}
    \end{subfigure} 
\begin{subfigure}{0.495\textwidth}
\centering
  \includegraphics[scale=0.4]{fig4b}
    \caption{}
    \label{fig:HVS}
    \end{subfigure} 
  \caption{ERA-based ROM for the unstable wake behind a stationary circular 
  cylinder at $Re=60$: (a) lift history of full order model and the ROM based on
  linearized dynamics subject to the impulse response, and 
  (b) Hankel singular values (HSVs) distribution corresponding to $500 \times 150$ Hankel matrix.}

\end{figure}
%%% Prediction of critical Re
For further verification, we next show that the ERA-based ROM 
is able to accurately predict the onset of the unstable wake state due to a
Hopf bifurcation.
The onset of unstable wake is commonly determined by the linearized 
NS equations in the literature (\cite{Luchini2007,Marquet2008,Mettot2014}). 
The growth rate $\sigma$ and 
the frequency $f$ are plotted as a function of Reynolds number in figure \ref{fig:growth_f}. 
The instability of the base flow occurs when the growth rate crosses $\sigma=0$
line at the critical Reynolds number $Re_{cr} \approx 46.8$, which is in a good agreement 
with the numerical prediction of \cite{Luchini2007,Marquet2008} and the experimental 
measurement of \cite{Williamson1996}. 
%
The frequency predicted by the ERA-based ROM at this critical Reynolds number 
is $f=0.119$, which matches quite well with the results 
of \cite{Luchini2007,Marquet2008}. However, it is worth noting that 
the frequency given by the linear model is only valid in the vicinity of critical Reynolds 
number and fails to capture the frequency in the region far away from the critical point, 
where nonlinear effects start to dominate the wake dynamics. 

%%% Global physical modes
To extract the most energetic structures via POD method, the snapshots of the flow 
solutions are stored during the process of the ROM construction i.e., the flow solution
is recorded at each physical time step subjected to the impulse response. 
For the unstable wake case at $Re_{cr} \approx 46.8$, 
the first POD mode corresponding to the 
streamwise and cross-flow velocity is shown 
in figure \ref{fig:cylinder_mode}. The spatial structures are antisymmetric and 
they travel downstream with the formation of Kelvin-Helmholtz instabilities.

%%%%%%%% figure for growth rate 
\begin{figure}
\centering
\begin{subfigure}{0.495\textwidth}
  \includegraphics[scale=0.3]{fig5a}
    \caption{}
    \label{fig:growth}
    \end{subfigure} 
\begin{subfigure}{0.495\textwidth} 
\centering
  \includegraphics[scale=0.3]{fig5b}
	\caption{}
	\label{fig:freq}
	\end{subfigure}	
        \caption{Prediction of critical Reynolds number via ERA-based ROM for the 
                flow past a stationary circular cylinder: 
                (a) growth rate $\sigma$ and
                (b) frequency $f$.  The cylinder wake becomes unstable when 
                the growth rate crosses $\sigma=0$ line at the critical 
                $Re_{cr} \approx 46.8$ and the vortex shedding emanates.  }
        \label{fig:growth_f}
\end{figure}


%%%%%%%% leading mode 

\begin{figure}
\centering
\begin{subfigure}{0.495\textwidth}
	 \includegraphics[scale=0.32]{fig6a}
    \caption{}
    \label{fig:mode1_ux}
    \end{subfigure} 
\begin{subfigure}{0.495\textwidth} 
\centering
     \includegraphics[scale=0.32]{fig6b}
	\caption{}
	\label{fig:mode1_uy}
	\end{subfigure}	
     \caption{First POD mode at 
     $Re_{cr} \approx 46.8$: (a) streamwise velocity and 
     (b) cross-stream velocity.
      The flow is from left to right. }
\label{fig:cylinder_mode}
\end{figure}

%%%%%%%% 

\subsection{Unstable flow past a stationary cylinder}
As discussed in Section \ref{sec:linear}, the wake flow becomes unstable 
through a Hopf bifurcation when $Re > Re_{cr}$ 
and  the vortex shedding appears behind a stationary cylinder 
at the frequency $f_{vs}$. 
The unstable flow finally reaches to a fully nonlinear state 
with the alternate time-periodic vortex shedding. 
%
The flow field in the whole domain behaves like a global oscillator, which 
causes unsteady lift and drag forces on the immersed body.
%
To further establish the validity of the numerical method 
and the desired accuracy for the VIV simulation, 
the dimensionless vortex shedding frequency $St= f_{vs}D/U$ and the root mean square (rms) value of  lift coefficient $C_l$ 
are compared with the works of \cite{williamson1989} and \cite{Zhang1995} 
for the Reynolds number $Re < 180$. 
The results are summarized in Table \ref{tab:cl_st}, which demonstrates a good agreement 
with the literature. This further confirms that 
the numerical methodology and the mesh discretization are adequate to 
capture the vortex dynamics and the stability characteristics of VIV response.  

\begin{table}
\begin{center}
\begin{tabular}{ l c c c c l }

Reynolds number $(Re)$   & $60$ & $80$ &  $100$ &  $120$   \\
\multirow{2}{*}{Lift coefficient rms $(C_l)$}  
& 0.1      & 0.17    &  0.24      & 0.29    & Present \\
& 0.1      & 0.16    &  0.25      & 0.31    & \cite{Zhang1995}\\
 \\
\multirow{3}{*}{Strouhal number ($St$)} 
& 0.137     & 0.156     &  0.166      & 0.176  & Present\\
& 0.136     & 0.152     &  0.164      & 0.172  & \cite{williamson1989} \\
& 0.142     & 0.159     &  0.172      & 0.182  & \cite{Zhang1995} 
\end{tabular}
\caption{Comparison of the rms value of lift coefficient ($C_l$) and Strouhal number $(St)$ obtained 
with previous studies for a range of Reynolds numbers. 
A constant time step size $\Delta t=0.05$ is employed for the present computations.}
\label{tab:cl_st}
\end{center}
\end{table}


%%%%%%%% ERA VIV%%%%%%%%%%%%%%%%%%%%%%%%%%%%%%%%%%%%%%%%%%%%%%%%%%%%%%%%%%%%%%%%%%%%%%%%%%%%%%%%%

\section{Results and discussion}\label{sec:VIV}


\subsection{Assessment of ERA-based ROM}\label{sec:ROM_val}

In this section,
the constructed ERA-based ROM is first applied to analyze the stability properties of the 
transversely vibrating circular cylinder at $(Re,m^*)=(60,10)$. 
Consistent with the previous literature of \cite{meliga2011,Zhang2015}, 
we use the terminology of structure mode (SM) and 
the wake mode (WM) to classify the distinct eigenvalue trajectories of the linear 
fluid-structure system governed by Eq. (\ref{eq:vivrom}).
%
When the eigenvalue of fluid-structure system approaches to that of the stationary 
cylinder, the resulting mode is defined as WM. The fluid-structure 
mode is considered as SM if the eigenvalue comes close to the natural frequency 
of the cylinder-alone system. 
%In the limit of weak coupling (e.g., large mass ratio), the wake mode and the structure mode are decoupled.

%
As discussed in \cite{Zhang2015}, VIV lock-in may result either from the instability 
of WM alone or via simultaneous existence of unstable SM and WM. 
In the event of first scenario, the lock-in occurs due to the closeness of 
the frequencies of WM and SM.
This type of VIV branch is termed as the resonance-induced lock-in.
For the second scenario, the instability to sustain the VIV lock-in occurs via 
combined mode instability of SM and WM, which is referred to as 
the flutter-induced lock-in or the coupled-mode flutter (\cite{DeLangre2006}). 
%
In the present study, 
the wake mode is also considered as the leading mode of unstable fluid system. 

We consider the continuous-time eigenvalues in the context of 
linear stability analysis via the ERA-based ROM. 
Using the fluid-structure matrix Eq. (\ref{eq:vivrom}), the eigenvalue 
is obtained by $\lambda=$log(eig($\mathsfbi{A_{fs}}))/\Delta t$,
where $\Delta t=0.05$ is the time step.
%
To graphically depict the dynamical behavior, we study the eigenvalue distribution of the system 
in the complex plane via root locus. The positions of the eigenvalues provide 
the information about the stability of fluid-structure system. For example, roots in the right 
half plane depict the unstable modes of the system, whereas the roots on the real 
axis characterize the asymptotic (non-oscillatory) behavior. Roots that are closest 
to the right half plane are lightly-damped oscillatory modes.

 %%%%%%% figure for m10_eig
\begin{figure}
\centering
\begin{subfigure}{0.495\textwidth}
\centering
	 \includegraphics[scale=0.35]{fig7a}
    \caption{}
    \label{fig:m10_re60_eig1}
    \end{subfigure} 
\begin{subfigure}{0.495\textwidth} 
\centering
     \includegraphics[scale=0.35]{fig7b}
	\caption{}
	\label{fig:m10_re60_eig23}
	\end{subfigure}	
     \caption{Eigenspectrum of the ERA-based ROM for a circular cylinder at $(Re,m^*)=(60,10)$: 
     (a) root loci as a function of the reduced natural frequency $F_s$, 
     where the unstable right-half (Re$(\lambda) > 0$) plane is shaded in grey color,
     and the hollow arrow indicates increasing $F_s$, 
     (b) real and imaginary parts of the root loci, whereby the lock-in region 
     is shaded in grey color. Two branches of lock-in namely resonance and flutter can 
     be seen in (b).
     }
\label{fig:m10_eig}
\end{figure}

%Fs sweeping 
Figure \ref{fig:m10_re60_eig1} shows the eigenvalue trajectory of 
the fluid-structure system as a function of the reduced natural frequency 
$F_{s}$ with $0.05 \le F_{s} \le 0.25$ and
the increment is $\Delta F_{s}=0.025$.
%
%WM branch 
In the figure, the WM branch originates in the vicinity of eigenvalue of stationary cylinder (uncoupled WM) 
and loops back as the reduced natural frequency $F_s$ increases. It is expected 
that the WM finally recovers to the eigenvalue of stationary cylinder as $F_s$ 
approaches infinity or the cylinder becomes stationary (i.e.,
without elastically mounted). It is also evident that WM  
remains unstable ($\sigma > 0$) throughout the sweeping as $Re = 60 > Re_{cr}$. 
%SM branch 
On the other hand, the SM branch arises from the bottom of the complex plane (low frequency regime) to the upper complex plane (high frequency regime).
As elucidated in figure \ref{fig:m10_re60_eig23}, 
the SM becomes unstable only when $0.147 < F_{s} \le 0.179$, which is determined by
the real part of eigenvalue.
% flutter mode 
As mentioned earlier, the unstable SM phenomenon can be considered  
as the coupled-mode flutter or the \emph{combined mode instability}. 
As shown in figure \ref{fig:m10_re60_eig23}, the imaginary part of eigenvalue as a 
function of $F_s$ reveals that the two distinct frequencies (WM and SM frequencies) 
co-exist in the combined mode instability. 
% resonance mode 
In addition, figure \ref{fig:m10_re60_eig23} also indicates the frequencies 
of WM and SM come closer when $0.11 \le F_{s} \le 0.147$, which is recognized 
as the resonance mode. 
Note that the lower left boundary of the resonance mode 
cannot be pinpointed precisely from the ROM due to the overlapping of SM and WM trajectories. 
Thus we determine the frequency at the lower left boundary from the FOM simulation, 
which is found to be $F_s=0.11$.

To further verify the stability results predicted by the ERA-based ROM, 
the VIV response is computed by direct numerical simulation using FOM. 
%
As shown in figure \ref{fig:m10_re60_fs}, the vortex shedding frequency 
begins to synchronize with the structure natural frequency at $F_{s}=0.11$ and recovers to 
the vortex shedding frequency at $F_{s}=0.179$.
%
As the nonlinearity starts to dominate the VIV response, 
the two distinct frequencies of WM and SM corresponding to the flutter mode
are eventually synchronized with the structure natural frequency $F_s$. 
%%
Figure \ref{fig:m10_re60_cly} suggests that the cylinder rises to the 
peak VIV amplitude at $F_{s}=0.179$ (lock-in onset $U_r \approx 5.59$), 
which compares accurately with the upper boundary of flutter mode predicted by the 
present ERA-based ROM.

It is worth mentioning that the cylinder acquires the maximum amplitude
at $F_{s}=0.179$ or $F_s/St=1.31$, which is not at $F_s/St \approx 1$, as expected from 
the classical resonance interpretation of VIV lock-in. This phenomenon suggests that the 
onset of VIV lock-in results from the amplification of energy input as a consequence of unstable SM, 
in which the structure is able to optimally absorb energy from the surrounding fluid system. 
It is analogous to the pitch and plunge flutter observed in 
the aeroelastic airfoil configuration. 
The flutter mode of VIV lock-in results 
from the coupling of periodic vortex shedding and the structural transverse displacement. 
%We argue that the lock-in in the flutter region is caused by the competition of SM and WM. 
In the flutter regime ($1.07 < F_s/St \le 1.31$), 
the unstable SM and WM jointly sustain VIV lock-in, 
whereas the wake mode dominates the 
resonance regime ($0.8 \le F_s/St \le 1.07$) until the VIV goes into the lock-out region. 

%%%%%%% figure for m10_FOM
\begin{figure}
\centering
\begin{subfigure}{0.495\textwidth}
\centering
	 \includegraphics[scale=0.35]{fig8a}
    \caption{}
    \label{fig:m10_re60_fs}
    \end{subfigure} 
\begin{subfigure}{0.495\textwidth} 
\centering
     \includegraphics[scale=0.35]{fig8b}
	\caption{}
	\label{fig:m10_re60_cly}
	\end{subfigure}	\\
%\begin{subfigure}{0.495\textwidth} 
%\centering
%     \includegraphics[scale=0.3]{m10_fem_cl}
%	\caption{}
%	\label{fig:m10_fem_cl}
%	\end{subfigure}	
%\begin{subfigure}{0.495\textwidth} 
%\centering
%     \includegraphics[scale=0.3]{m10_fem_phi}
%	\caption{}
%	\label{fig:m10_fem_phi}
%	\end{subfigure}	
     \caption{VIV results as a function of reduced natural frequency $F_s$
     using FOM at $(Re,m^*)=(60,10)$: variation of 
      (a) normalized vortex shedding frequency $f$ and 
      (b) rms value of lift coefficient ($C_l$) and normalized maximum amplitude ($Y_{max}$).
      The lock-in is shaded in grey color.}
     \label{fig:m10_re60_fom}
\end{figure}

More quantitative insights into the VIV lock-in mechanism can be obtained by 
figure \ref{fig:m10_re60_Phi}, which shows the phase angle $\phi$ estimated 
by the ERA-based ROM (see Appendix A). The phase angle $\phi$ of the ROM is function of ($F_s,\lambda$) and its
sign is determined by the real part of eigenvalues. 
The instantaneous phase angle of FOM is obtained 
by the Hilbert transformation of lift and displacement, as described in \cite{Tham2015}. 
%
In figure \ref{fig:m10_re60_Phi}, we present the phase angles of the FOM and 
ROM as functions of $F_s$ at $(Re,m^*)=(60,10)$.
%
As compared to the FOM result, 
the WM trajectory yields a continuous transition from $0^0$ to $180^0$ as $F_s$ decreases in the lock-in region.
%where the frequencies of WM and SM synchronize. 
In contrast, the phase angle of SM remains positive only within 
the flutter mode ($0.147 < F_{s} \le 0.179$). 
%
%As presented in figure \ref{fig:m10_re60_eig1}, 
%the structure mode remains stable and the resonance-induced instability is mainly
%sustained by the wake mode.

\begin{figure}
\centering
\centering
\begin{subfigure}{0.495\textwidth}
\centering
	 \includegraphics[scale=0.35]{fig9a}
    \caption{}
    \label{fig:m10_re60_Phi}
    \end{subfigure} 
\begin{subfigure}{0.495\textwidth} 
\centering
     \includegraphics[scale=0.35]{fig9b}
	\caption{}
	\label{fig:m10_response}
	\end{subfigure}	
	\caption{VIV results as a function of reduced natural frequency $F_s$ 
        at $(Re,m^*)=(60,10)$: (a) comparison of 
        phase angle difference $\phi$ between the ERA-based ROM and FOM, where 
        the lock-in is shaded in grey color and 
	 (b) lift $C_l$ history at two representative 
         frequencies $F_s=(0.14,0.177)$.
         In (a), ({\protect\bluedashdot}) represents $F_s=0.13$. }
\end{figure}

It is also interesting to note in figure \ref{fig:m10_re60_cly} that the lift coefficient 
is significantly amplified in the vicinity of VIV lock-in onset reduced velocity ($U_r \approx 5.59$).
A gradual decrease and eventually recovery to the value of stationary cylinder counterpart 
as $U_r$ increases ($F_s$ decreases) can be seen in the figure. 
%the resonance region. 
To further assess the behavior of lift coefficient in the 
flutter and resonance regimes, the lift histories for two 
representative reduced frequencies $F_{s}=0.177$ and 0.140
are compared in figure \ref{fig:m10_response}. 
% lift reduction associate with phase jump 
The minimum rms value of lift coefficient $C_l$ occurs 
at $F_s \approx 0.13$ or $F_s / St \approx 0.95$, which coincides with the phase angle jump, as shown in figure \ref{fig:m10_re60_Phi}. 
Therefore, we can infer that 
the reduction in the rms lift coefficient has a direct relation 
with the phase angle. In Section \ref{sec:ReEffect}, 
we further confirm that the lift reduction is not associated with the resonance mode.

%% mass ratio effect 
Geometric and physical parameters such as cross-sectional shape and 
mass ratio play an important role with regard to the coupling strength 
of fluid-structure interaction. 
A classification of the fluid-structure eigenmodes as 
WM and SM is suitable for weak fluid-structure interaction 
(e.g., very large mass ratio), which is elucidated in figure \ref{fig:m10_eig} by two distinct 
eigenvalue branches of WM and SM. Owing to weaker fluid-structure coupling 
at large $m^*$ (i.e., in the limit of $m^* \to \infty$),  the eigenfrequency 
of WM approaches to that of the stationary cylinder wake for all values of $F_s$ and 
the frequency of SM comes close to the natural frequency of the cylinder-only system.
%
However, the root loci of WM and SM can coalesce   
and form coupled modes at certain conditions, such as in the limit of 
low mass ratio \citep{meliga2011,mittal2016} and for different geometrical shapes, 
as demonstrated in Section \ref{sec:topology}.
These coupled modes do not offer a distinct characteristics of the WM and SM, 
since both branches exchange their roles when the coalescence of eigenvalue occurs.
Similar to \cite{mittal2016}, we term these mixed or coupled modes 
as the wake-structure mode I (WSMI) and the wake-structure mode II (WSMII).
%
For higher value of $F_s$,  WSMI behaves as WM and WSMII recovers to SM.
On the other hand, for smaller $F_s$ range, WSMI and WSMII  represent the SM and
WM, respectively.
%
A characteristic anticrossing with a frequency splitting can be also observed at 
low mass ratio, which is one of the trait of strongly coupled harmonic oscillators 
\citep{novotny2010}.
%
To demonstrate the effect of the mass ratio,
further details of the coupled modes WSMI and WSMII are discussed in Appendix B.

In this section, the ERA-based ROM is successfully employed to perform 
the linear stability analysis of the vortex-induced vibration of transversely vibrating 
circular cylinder. Consistent with the previous study of \cite{Zhang2015} 
on the mechanism of VIV, we clearly observe two distinct lock-in patterns 
of the flutter and the resonance from our eigenmode analysis.
However, the regime classification in \cite{Zhang2015} was only based on 
the VIV linear analysis at $Re=60$. 
%
In the next section, we extend the existence and dependence of the two distinct lock-in modes 
for a larger parameter space of Reynolds number in the laminar flow regime.
%ccccccccccccccccccccccc

\subsection{Effect of Reynolds number}\label{sec:ReEffect}
As shown in figure \ref{fig:growth}, the growth rate amplifies as $Re$ increases, 
which indicates that the coupling between the WM and SM tends to become stronger for higher $Re$. 
To further investigate the effect of Reynolds number, 
the VIV ROMs for $Re=(70,100)$ and $m^*=10$ are constructed and the 
stability analysis similar to $Re=60$ is carried out. 
The root loci as a function of natural frequency $F_s$ are shown in 
figure \ref{fig:dre_eig23}. As compared to the root loci at $Re=60$, figure \ref{fig:dre_eig23} 
shows that the range of unstable SM or flutter mode slightly increases 
to $0.106 \le F_s \le 0.187$ or $0.71 \le F_s/St \le 1.25$ 
and covers the entire lock-in region. 
%
It is also evident by the FOM results, as shown in figure \ref{fig:re70_fem}, 
where the lock-in region 
is  $0.11 \le F_s \le 0.192$ or $0.73 \le F_s/St \le 0.128$. 
This new finding from the present work 
suggests that the extents of flutter and resonance modes are highly sensitive 
to Reynolds number. 
%
This can be further elucidated by looking into the stability region 
where the magnitude of velocity leading eigenmode
is large \citep{Luchini2007}. 
The complex modal velocity components, 
the streamwise velocity $\widehat{u}$ 
and the transverse velocity $\widehat{v}$, 
are computed from the linearized NS equations around the base flow. 
To visualize the magnitude of the leading modal velocity, 
we first compute the amplitude of the complex modal velocity components ($|\widehat{u}|$ and $|\widehat{v}|$)
and then evaluate the pointwise modal velocity magnitude 
as $|\widehat{U}|=\sqrt{|\widehat{u}|^2+|\widehat{v}|^2}$.
As shown in figure \ref{fig:dre_mode}, the stability
region shifts gradually to the bluff body, which indicates the coupling 
effect between the unstable wake and the bluff body is enhanced 
as $Re$ increases.

%%%%%%% figure for eig ROM RE=70
 
 \begin{figure}
\centering
\begin{subfigure}{0.495\textwidth}
\centering
    \includegraphics[scale=0.35]{fig10a}
    \caption{}
    \label{fig:dre_eig1}
    \end{subfigure} 
\begin{subfigure}{0.495\textwidth} 
\centering
 \includegraphics[scale=0.35]{fig10b}
	\caption{}
	\label{fig:dre_eig23}
	\end{subfigure}	
\caption{Effect of Reynolds number on the eigenspectrum of ERA-based ROM at $Re=(60,70,100)$ and $m^*=10$:  
        (a)root loci as a function of the reduced natural frequency $F_s$,
        where the unstable right-half (Re$(\lambda) > 0$) plane is shaded in grey color,
        and hollow arrow indicates increasing $F_s$, 
       (b) real and imaginary parts of the root loci.
       SM data points are denoted by the filled symbol with the same shape as WM in (a,b).
       The boundary of $Re(\lambda)>0$ for SM at $Re=70$ is defined at $F_s=0.106$ 
        ({\protect\greendash}), and $F_s=0.187$ ({\protect\greendashdot}), 
       in the real parts of root loci in (b), respectively. 
       }

	\label{fig:dre_eig}
\end{figure}
 
 
 %%%%%%% figure for VIV FEM
 
 \begin{figure}
\centering
\begin{subfigure}{0.495\textwidth}
\centering
    \includegraphics[scale=0.35]{fig11a}
    \caption{}
    \label{fig:m10_re70_fs}
    \end{subfigure} 
\begin{subfigure}{0.495\textwidth} 
\centering
 \includegraphics[scale=0.35]{fig11b}
	\caption{}
	\label{fig:m10_re70_cly}
	\end{subfigure}	
%\begin{subfigure}{0.495\textwidth} 
%\centering
% \includegraphics[scale=0.3]{m10_re70_cl}
%	\caption{}
%	\label{fig:m10_re70_cl}
%	\end{subfigure}	
\begin{subfigure}{0.495\textwidth} 
\centering
 \includegraphics[scale=0.35]{fig11c}
	\caption{}
	\label{fig:m10_re70_phi}
	\end{subfigure}	
        \caption{VIV results  as a function of reduced natural frequency $F_s$
 by FOM at $(Re,m^*)=(70,10)$: variation of (a) normalized vortex shedding frequency $f$, 
      (b) rms value of lift coefficient ($C_l$) and maximum amplitude ($Y_{max}$), and
      (c) phase angle ($\phi$).
      The lock-in is shaded in grey color.}
	\label{fig:re70_fem}
\end{figure}

%The root loci as a function of $F_s$ with Reynolds number $Re = 60,70,100$ are plotted 
%in figure \ref{fig:difms_eig1}, which 

Figure \ref{fig:dre_eig} shows that the unstable SM branch is gradually pronounced and 
covers the lock-in region as the Reynolds number increases, whereas the size of WM loop becomes smaller. The threshold Reynolds number is approximately $Re = 70$ 
when the resonance-mode ceases to exist for $m^*=10$. 
This result suggests that the frequency lock-in VIV is pure flutter mode 
instability for $Re \ge 70$, which is consistent with the theoretical analysis 
of \cite{DeLangre2006} using the wake oscillator model. However, 
due to the simplification in the wake oscillator model and 
without nonlinear and dissipative terms, a general statement on VIV lock-in
as a  coupled flutter mode may not be valid for all flow conditions. 
On the other hand, the numerical study of \cite{Zhang2015} 
is only valid for $Re_{cr} < Re < 70$. 
Table \ref{tab:Re_VIVpattern} summarizes the existence of flutter 
and resonance modes at different Reynolds numbers for a vibrating circular cylinder. 

\begin{figure}
\centering
\begin{subfigure}{0.495\textwidth}
\centering
    \includegraphics[scale=0.3]{fig12a}
    \caption{}
    \label{fig:re60_mode1}
    \end{subfigure} 
\begin{subfigure}{0.495\textwidth} 
\centering
 \includegraphics[scale=0.3]{fig12b}
	\caption{}
	\label{fig:re70_mode1}
	\end{subfigure}	
        \caption{Influence of Reynolds number on the stability regions defined 
        by the pointwise modal velocity magnitude $|\widehat{U}|$
        of leading eigenmodes for $Re=$: (a) 60, (b) 70.
        The flow is from left to right.
        }
	\label{fig:dre_mode}
\end{figure}


 %%%%%%%%
 
 \begin{table}
  \begin{center}
  \begin{tabular}{l c c}
       VIV regimes   & flutter & resonance \\
       $Re \le Re_{cr}$      & $\checkmark$ & \\
       $Re_{cr} < Re < 70$   & $\checkmark$ & $\checkmark$ \\
       $Re \ge 70$           & $\checkmark$ &  
  \end{tabular}
  \caption{VIV lock-in regimes of circular cylinder for Reynolds number 
  in the range $ 30 \le Re \le 100$ 
  and mass ratio $m^*=10$. For $Re > Re_{cr}$, flutter regime comprises 
    both unstable eigenvalues (Re$(\lambda) > 0$) 
     of WM and SM, whereas resonance regime has only unstable WM.}
  \label{tab:Re_VIVpattern}
  \end{center}
\end{table}

 %%%%%%% figure for VIV vorticity contour at different Fs
 
\begin{figure}
\begin{subfigure}{0.495\textwidth}
\centering
    \includegraphics[scale=0.3]{fig13a}
    \caption{}
    \label{fig:m10_re70_vorfs013}
    \end{subfigure} 
\begin{subfigure}{0.495\textwidth} 
\centering
 \includegraphics[scale=0.3]{fig13b}
	\caption{}
	\label{fig:m10_re70_vorfs015}
	\end{subfigure}	
\begin{subfigure}{0.495\textwidth} 
\centering
 \includegraphics[scale=0.3]{fig13c}
	\caption{}
	\label{fig:m10_re70_vorfs017}
	\end{subfigure}	
\begin{subfigure}{0.495\textwidth} 
\centering
 \includegraphics[scale=0.3]{fig13d}
	\caption{}
	\label{fig:m10_re70_vorfs019}
	\end{subfigure}	
        \caption{Full order results for the circular cylinder at  $(Re, m^*)=(70, 10)$:
          Instantaneous vorticity contours at $F_{s} = $ (a) 0.13, (b) 0.15, (c) 0.17, (d) 0.19. 
        Contour levels are from -0.5 to 0.5 in increment of 0.077 and the flow is from left to right.}
	\label{fig:vor_fs_re70}
\end{figure}
 

It is also interesting to note in figure \ref{fig:m10_re70_cly}
that the reduction in the rms value of lift coefficient 
also appears, although the resonance regime does not exist at $Re = 70$. 
%
This observation suggests that the reduction of rms lift force does 
not interlink with the flutter or resonance regime, which is different from the 
conclusion by \cite{Zhang2015} that the rms lift coefficient is 
attenuated in the resonance regime 
but amplified in the flutter region. As shown in figures \ref{fig:m10_re70_cly} and \ref{fig:m10_re70_phi}, the
minimum lift coefficient appears at $F_s \approx 0.135$ or $F_s / St \approx 0.92$, where the phase angle changes from
$0^0$ to $180^0$. It further confirms that the lift reduction phenomenon is explicitly linked with the phase angle.

%
Furthermore, figure \ref{fig:vor_fs_re70} shows the instantaneous patterns of vortex shedding are investigated for a broad range of reduced natural frequencies.
We adopt the classical 
terminology of \cite{williamson1988vortex}  to identify the vortex shedding patterns.
% definition of 2S and C(2S) 
In the 2S mode, a single
vortex is alternately shed from each side of the cylinder per shedding cycle, 
whereas the vortices start to coalesce in the far wake in the C(2S) mode.
%
As reported by \cite{Zhang2015}, the 2S mode is observed in the resonance regime, 
whereas C(2S) mode appears in the flutter region. However, we argue that the vorticity topology 
changes gradually from the C(2S) to the 2S mode as $F_s$ decreasing from 0.19 to 0.13, which indicates the topology variation associates 
with the vibration amplitude. The C(2S) mode starts to appear at VIV lock-in onset $U_r\approx 5.21$ ($F_{s}=0.192$) 
and gradually transits to the 2S mode as the amplitude decreases. 
%
To further generalize our ERA-based ROM 
for the VIV lock-in regime, we next investigate the influence of rounding in the 
VIV lock-in mechanism of a square cylinder.


%ccccccccccccccccccccccc

%% rounding effects 

\subsection{Effect of rounding}\label{sec:square}
In the previous section, the effects of Reynolds number and mass ratio have 
been considered for the transverse VIV of circular cylinder. 
It is interesting to explore whether the aforementioned VIV lock-in regimes
of circular cylinder is still applied to an elastically mounted square cylinder. 
As it is known that the presence of sharp corners on a square cylinder vastly 
alters the flow dynamics as compared to the ones with circular/elliptical section 
having smooth curves. Besides the angle of incidence, the sharp corners are 
important contributing factor in the geometry of bluff body, as they 
affect the flow separation points which in turn dictates the wake dynamics.
By gradual removal of sharp corners of square cylinder, 
a circular cross-section can be recovered.
%
%
%
As reported in \cite{Jaiman2015}, the VIV response of square cylinder is dramatically different
in comparison to its circular cylinder counterpart.
For example, the lock-in region of square cylinder is narrower and the amplitude 
is smaller for the similar VIV operational parameters ($Re, m^*, \zeta$), 
as extensively discussed in \cite{Jaiman2016a,Jaiman2016b}. 
%
Recently, a rounded square is also numerically studied in 
terms of primary and secondary instabilities
\citep{Park2016}, which shows that the sharp corner alters 
the flow topology significantly,
subsequently changing the stability properties of wake dynamics. 
%
It is therefore important to consider the effect of rounding the corners of a square 
cylinder for the VIV mechanism. 
The VIV stability properties of five different 
cross-sections including circle and square are explored in the context of eigenvalue 
distributions.
\begin{figure}
\centering
    \includegraphics[scale=0.55]{fig14}
    \label{fig:square_schematic}
     \caption{Square-section bluff body with projected width $D$ and
     rounding radius $r_{s}$ in uniform flow. 
      Rounding is introduced by inscribing a quarter circle with $r_{s}$ at each edge of 
      the square geometry.
      The square and circular cylinders correspond 
to the rounding radius of $r_{s}=0$ and $r_{s}=0.5D$, respectively. 
     }
	\label{fig:square}
\end{figure}

Figure \ref{fig:square} schematically depicts the 
square cylinder with a rounding radius $r_s$. The edge length of square 
with sharp corner is denoted by $D$. 
{Rounding is introduced by inscribing a quarter circle with $r_{s}$ at each edge of 
the square geometry, whereby $r_{s}=0.5D$  corresponds to a circular shape 
and $r_{s}=0$ recovers to the basic square shape.}
%
The characteristic dimension $D$ of all the cross-sections is identical,  
where $D$ is the maximum dimension of the cylinder normal to the flow.
%
The origin of the Cartesian coordinate system 
coincides with the center of the square. 
%%%%%%% figure for ROM different topology
 
\begin{figure}
\centering
\begin{subfigure}{0.495\textwidth}
\centering
    \includegraphics[scale=0.35]{fig15a}
    \caption{}
    \label{fig:re60_square_eig1}
    \end{subfigure} 
\begin{subfigure}{0.495\textwidth} 
\centering
 \includegraphics[scale=0.35]{fig15b}
	\caption{}
	\label{fig:re60_square_eig23}
	\end{subfigure}	
        \caption{Effect of rounding $r_s$ on eigenspectrum of ERA-based ROM at $(Re,m^*)=(60,10)$: 
        (a) root loci as a function of the reduced natural frequency $F_s$, 
        whereby the unstable right-half (Re$(\lambda) > 0$) plane is shaded in grey and the 
        hollow arrow indicates increasing $F_s$ and 
        (b) real and imaginary parts of root loci. 
        In (a), the unstable eigenvalues of stationary square cylinder (uncoupled WM) 
        with different rounding values are connected by black curve with solid arrow indicating increasing $r_s$.
        SM is denoted by the filled symbol with the same shape as WM in (a,b).
        }
	\label{fig:square_eig}
\end{figure}

%%%%%%%% figure for square fem 

 \begin{figure}
\centering
\begin{subfigure}{0.495\textwidth}
\centering
    \includegraphics[scale=0.35]{fig16a}
    \caption{}
    \label{fig:m10_re60_square_fs}
    \end{subfigure} 
\begin{subfigure}{0.495\textwidth} 
\centering
 \includegraphics[scale=0.35]{fig16b}
	\caption{}
	\label{fig:m10_re60_square_cly}
	\end{subfigure}	
%\begin{subfigure}{0.495\textwidth} 
%\centering
% \includegraphics[scale=0.3]{square_fem_cl}
%	\caption{}
%	\label{fig:square_fem_cl}
%	\end{subfigure}	
\begin{subfigure}{0.495\textwidth} 
\centering
 \includegraphics[scale=0.35]{fig16c}
	\caption{}
	\label{fig:m10_re60_square_phi}
	\end{subfigure}	
        \caption{VIV results of square cylinder with sharp corners 
        using FOM at $(Re,m^*)=(60,10)$: 
       variation of (a) normalized vortex shedding frequency $f$ and  
      (b) rms value of lift coefficient ($C_l$) and maximum amplitude ($Y_{max}$), and 
      (c) phase angle ($\phi$).
      The lock-in is shaded in grey color.}
	\label{fig:square_fem}
\end{figure}

To begin with, the VIV linear analysis is conducted 
for a square cylinder with sharp corners at ($Re, m^*$) = (60, 10). 
It is evident from figure \ref{fig:re60_square_eig1} that the SM is stable, 
which suggests the lock-in is entirely dominated by the resonance mode. Due to the absence of lock-in 
via flutter mode, the onset reduced velocity ($U_r$) for a square cylinder may not be clearly recognized as compared to the circular cylinder counterpart. 
Furthermore, the region of WM loop for the square cylinder is smaller than the circular cylinder counterpart,
implying the coupling between the fluid and structure is reduced due to the sharp corners.
%
In figure \ref{fig:square_eig}, the root loci for the fluid-structure system of square cylinder 
provide an explanation that the lock-in only consists of resonance mode 
and the flutter state disappears 
due to the presence of sharp corners for $(Re,m^*)=(60,10)$. 
As expected, the sharp corners suppress the continuous movement 
of separation points, whereby the smooth circular cylinder promotes the movement of separation points.
%The lock-in region in this case can be predicted by the FOM simulation. 
Figure \ref{fig:square_fem} shows the frequency,  the VIV response and 
the lift coefficient from the FOM simulation for the vibrating square cylinder. 
The extent of lock-in 
region can be observed as $0.11 \le F_{s} \le 0.154$ or $0.87 \le F_{s}/St \le 1.21$.  
The results illustrate that the maximum amplitude 
also acquires at the lock-in onset ($U_r \approx 6.49$) approximately even no flutter regime 
exists. 
%This finding suggests that the lock-in is indeed due to the mode competition effects 
%between SM and WM.
%
It is notable that the maximum 
amplitude is smaller and the lock-in region is narrower than its circular cylinder counterpart, 
which is observed earlier for the square cylinder \citep{Jaiman2016a,Zhao2013b}. 
%%

Similar to its circular counterpart, the lift coefficient for a square cylinder 
also experiences an amplification 
in the vicinity of lock-in onset $U_r \approx 6.49$ 
, and gradually recovers to the stationary counterpart as $F_s$ decreases ($U_r$ increases). 
The maximum value of rms lift coefficient is $C_l=0.314$, as shown in figure \ref{fig:m10_re60_square_cly},  
which is approximately 3.1 times larger 
than the stationary square cylinder value ($C_l=0.1$).
For the VIV of circular cylinder, on the other hand, 
the amplification of the rms lift coefficient is approximately $5.9$ times than the stationary 
circular cylinder, as shown in figure \ref{fig:m10_re60_cly}.
%
To understand the direction of energy transfer between the fluid and the square cylinder, 
the phase angle is shown in figure \ref{fig:m10_re60_square_phi}. 
Similar to the circular cylinder, there is a sudden jump from $0^0$ to $180^0$ during the lock-in region around $F_s \approx 0.125$ or $F_s/St \approx 0.98$, where the rms lift coefficient 
acquires the minimum value. 
%
%The root loci of square VIV ROM in figure \ref{fig:square_eig} provide an explanation that 
%the lock-in only consists of resonance mode and the flutter state disappears 
%due to the presence of sharp corners for $(Re,m^*)=(60,10)$. 
%As expected, the sharp corners suppress the continuous movement 
%of separation points, whereby the smooth circular cylinder promotes the movement of separation points.


%%%%%%% figure for  different vorticity contour
 
  \begin{figure}
\begin{subfigure}{0.495\textwidth}
\centering
    \includegraphics[scale=0.3]{fig17a}
    \caption{}
    \label{fig:m10_re60_square_vorfs120}
    \end{subfigure} 
\begin{subfigure}{0.495\textwidth} 
\centering
 \includegraphics[scale=0.3]{fig17b}
	\caption{}
	\label{fig:m10_re60_square_vorfs130}
	\end{subfigure}	
\begin{subfigure}{0.495\textwidth} 
\centering
 \includegraphics[scale=0.3]{fig17c}
	\caption{}
	\label{fig:m10_re60_square_vorfs140}
	\end{subfigure}	
\begin{subfigure}{0.495\textwidth} 
\centering
 \includegraphics[scale=0.3]{fig17d}
	\caption{}
	\label{fig:m10_re60_square_vorfs150}
	\end{subfigure}	
        \caption{Full order results for the square cylinder at  $(Re, m^*)=(60, 10)$:
         Instantaneous vorticity contours at $F_{s}=$ (a) 0.12, 
        (b) 0.13, (c) 0.14, (d) 0.15. Contour levels are from -0.5 to 0.5 in increment of 0.077 
        and the flow is from left to right.}
	\label{fig:vor_fs_square}
\end{figure}


%
For the range of $F_s$ considered, two counter-rotating vortices (2S wake mode) are released 
every oscillation cycle from the rear of each cylinder, as shown in figure \ref{fig:vor_fs_square}. 
From the vorticity fields, separations along the front corner of lateral edges  
for the square with sharp corners can be observed for the considered cases.
% Weigang: Check this. should be C(2S) for circular cylinder
While the C(2S) mode is observed in the vicinity of lock-in onset for the circular cylinder, the 
classic 2S wake mode is dominated for the square cylinder. 
This is consistent with our previous hypothesis that 
the wake pattern is closely related to the vibration amplitude.
%
We next elucidate the influence of rounding 
on the distribution of eigenspectrum and the unstable global 
modes.


%%%%%%% figure for  leading mode
\begin{figure}
\centering
\begin{subfigure}{0.495\textwidth}
\centering
	 \includegraphics[scale=0.3]{fig18a}
    \caption{}
    \label{fig:square_r00}
    \end{subfigure} 
\begin{subfigure}{0.495\textwidth} 
\centering
     \includegraphics[scale=0.3]{fig18b}
	\caption{}
	\label{fig:square_r01}
	\end{subfigure}	
\begin{subfigure}{0.495\textwidth} 
\centering
     \includegraphics[scale=0.3]{fig18c}
	\caption{}
	\label{fig:square_r02}
	\end{subfigure}	
\begin{subfigure}{0.495\textwidth} 
\centering
     \includegraphics[scale=0.3]{fig18d}
	\caption{}
	\label{fig:square_r04}
	\end{subfigure}	
     \caption{Stability regions shown by the spatial distribution of pointwise 
     modal velocity amplitude $|\widehat{U}|$ at $Re=60$ 
     for different rounding parameters $r_s=$
     (a) 0.0, (b) $0.1D$, (c) $0.2D$, and (d) $0.4D$. 
     The flow is from left to right.}
\label{fig:square_mode}
\end{figure}
 
%The root loci is plotted in figure \ref{fig:square_eig_drs} for the square cylinder 
%with different values of rounding radius $r_{s}$. 
As shown in figure \ref{fig:re60_square_eig1}, 
the SM trajectory moves gradually towards the positive real axis (Re$(\lambda)> 0$) 
as the rounding radius  $r_s$ increases. 
Meanwhile, the flutter region starts to appear and achieves the maximum 
extent for the rounding radius $r_{s}=0.5D$. 
%
As expected, the rounding of the 
corners delays the onset of separation hence, the rounding aids in reducing the bluffness of square cylinder. 
%
It is also worth noting that unstable SM and WM loop are both more 
pronounced indicating that the coupling effect between the fluid and structure becomes stronger.
%
However, the growth rate of uncoupled WM does not increase monotonically 
from the square ($r_s=0$) to the circular ($r_s=0.5D$) configuration, 
as shown by a black curve with solid arrow in figure \ref{fig:re60_square_eig1}.  
%
The growth rate first decreases, 
then increases and eventually recovers to the growth rate of the circular cylinder.

By examining the real part of root loci in figure \ref{fig:re60_square_eig23}, 
the onset of lock-in starts to move
towards the low reduced velocity (high $F_s$) as the rounding radius $r_s$ increases. 
%
In figure \ref{fig:square_mode}, from the comparison of stability regions through the contours 
of $|\widehat{U}|$, we observe that the rounding has significant 
effects on the wake topology, 
subsequently alerting the stability properties. 
The separation point can move widely as the rounding radius increases, indicating the flutter 
mode is more pronounced. 
%
It also shows that the stability region gradually moves close to the rear of the bluff body as 
$r_s$ increases, which is similar to the effect of $Re$, as discussed in 
Section \ref{sec:ReEffect}. Consequently, the level of 
fluid and structure interaction is enhanced.
%
This trend is monotonic as compared to the growth rate, suggesting
that the flutter mode is pronounced gradually. It explains why the SM trajectory moves towards 
the right half plane monotonically.
%
The WM branch, on the other hand, moves toward imaginary axis positive or 
higher frequency direction, monotonically.
%
For the similar VIV operating parameters, 
the circular cylinder is much easier to perturb as compared to its square counterpart 
at the same Reynolds number. 

% square less stable for fixed but stable with transverse degree of freedom 
While the rounding generally stabilizes 
the wake flow for a stationary square cylinder,  
it promotes the movement of separation points along the smooth rounded surface 
of the vibrating cylinder.
Analogous to the aeroelastic flutter with 
plunge-torsion mode coupling for an airfoil configuration, as pointed by \cite{DeLangre2006}, 
the transverse periodic displacement and the movement of separation points can form a 
similar dynamics for a transversely vibrating circular cylinder.
The square cylinder with sharp corners restricts the free motion of separation points 
and is relatively stable in the sense that the lock-in onset reduced velocity $U_r$ 
is greater than its circular counterpart. 
Moreover, as compared to the circular cylinder at $(Re,m^*)=(60,10)$, 
the lock-in range of square cylinder is narrower and only resonance regime exists. 
%
To further generalize our findings, 
we next examine the lock-in regimes from the eigenvalue distributions for additional bluff bodies 
of a smooth curve geometry of elliptic cylinder and two sharp corner shapes of 
forward triangle and diamond cylinders.
%
%% topology effects

%ccccccccccccccccccccccc
\subsection{Effect of geometry}\label{sec:topology}

In this section, a set of three representative two-dimensional geometries is assessed to elucidate 
the frequency lock-in regimes and to demonstrate the ability of the developed ERA-based ROM. 
The three additional geometries namely ellipse as a smooth curve 
and forward triangle and diamond 
with sharp corners are shown in figure \ref{fig:topology}. 
The major axis with length $D$ of elliptic cylinder is placed normal to the flow direction and 
the aspect ratio $AR=0.5$ is defined by the ratio of minor $D/2$ to major axis length. 
The forward triangle is equilateral and 
has the edge with length $D$ normal to the flow and the peak corner is in the leeward side.
Similar to \cite{Zhao2013b}, 
the diamond geometry is considered as a square cylinder with $45^0$ flow incidence and 
the Reynolds number is defined by the edge length $D$ as the characteristic length scale.
%
Similar to the circular and square cylinders, the new geometries of ellipse, 
the forward triangle and the diamond cylinders 
undergo the unsteady wake transition via a Hopf bifurcation at a critical 
Reynolds number $Re_{cr}$, which is responsible for the onset of the time-periodic 
vortex-shedding phenomenon. 
%
Table \ref{tab:Recr_topology} shows 
the critical Reynolds number  $Re_{cr}$ 
of different geometries computed by our ERA-based ROM, 
which matches reasonably well with previous studies.
%
It is worth noting that the forward triangle has the lowest critical Reynolds number 
for the initial wake transition from steady to unsteady flow. 
%
The unsteady transition of the ellipse with the aspect ratio $AR=0.5$ and the diamond 
with sharp corners occurs lower than their circular and square counterparts.
%
Due to unsteady lift and drag forces, the three geometries can undergo 
flow-induced vibration if mounted on the elastic supports. 
%% topology 
\begin{figure}
\centering
    \includegraphics[scale=0.7]{fig19}
     \caption{Schematics of bluff-body geometries with relevant dimensions. 
     Representative elliptical cylinder (left) has aspect ratio $AR=0.5$, forward 
     triangle (middle) is equilateral with angle $60^0$, 
     and diamond (right) represents a square 
     cylinder with sharp corners at $45^0$ flow incidence.}
      \label{fig:topology}
\end{figure}

%% critical Reynolds number

\begin{table}
  \begin{center}
  \begin{tabular}{l c c c c c}
                        &ellipse &circular &square &forward triangle  &diamond\\
       Present $Re_{cr}$               & 38.0 & 46.8 & 44.7 & 35.5 & 38.9 \\
       \cite{Thompson2014}   & 38.8 & 47.2 & --   & --   & --   \\
       \cite{Park2016}       & --   & 46.7 & 44.7 & --   & --   \\
       Lock-in onset $U_r$   & 4.18 & 5.59 & 6.49 & 5.43 & 4.63
  \end{tabular}
  \caption{Comparison of critical Reynolds numbers  $Re_{cr}$ between 
  the available literature and the
  predicted values by ERA-based ROM for different topologies of bluff bodies.
  The onset reduced velocity $U_r$ of VIV lock-in from the present study 
  is also outlined in the last row.
  }
  \label{tab:Recr_topology}
  \end{center}
\end{table}


To further elucidate the lock-in mechanism, we plot  
the root loci and the real and imaginary parts of eigenvalues 
for the additional three bluff bodies in figure \ref{fig:topology1}. 
The figure clearly shows that the geometry of bluff body has a significant impact 
on the eigenvalue trajectory. The elliptical configuration
has the lowest lock-in onset $U_r$ followed by the diamond 
and the forward triangle configurations as shown in table \ref{tab:Recr_topology}.
%
%Figure \ref{fig:topology1_eig23} shows the amplification rate  
%and the frequency  for the three geometries. 
%
Compared to the circular cylinder at the identical condition 
of $(Re,m^*)=(60,10)$, the root loci of SM and WM coalesce 
for the diamond, ellipse and forward triangle configurations.
Similar to the low mass ratio effect during the VIV of circular cylinder, 
both the branches exchange their roles 
and no distinction can be made between the WM and SM for the three geometries.
Therefore, we consider the  coupled modes WSMI and WSMII 
to classify the stability characteristics for these geometries.
% Growth curves
When the intersection of growth rate curves corresponding the WSMI and WSMII 
occurs in figure \ref{fig:topology1_eig23} (top), 
the stability roles of WSMI and WSMII switch at a specific value of $F_s$ for the 
three geometries.
% Frequency
As shown in figure \ref{fig:topology1_eig23} (bottom),
the two curves of Im$(\lambda)$ for the three geometries  no longer cross, 
in comparison to the circular cylinder counterpart at the identical conditions.
Owing to stronger coupling, there is 
a characteristic anticrossing with a frequency splitting between the WSMI and WSMII
for the three geometries.  
%whereby the frequency splitting can be quantified as 
%$\Delta f = [\mathrm{Im}(\lambda/2\pi)_\mathrm{WM} 
%-\mathrm{Im}(\lambda/2\pi)_\mathrm{SM}]$.
%The elliptical cylinder has a larger anticrossing, which suggests that it has a 
%greater coupling strength as compared to other configurations 
%at the identical operational parameters.
In addition, the forward triangle branch departs further away from the line $f=2\pi F_s$ 
as compared to the diamond and elliptical cylinders.

% Diamond and ellipse
In contrast to the square cylinder ($0^0$ degree flow incidence) at $(Re,m^*)=(60,10)$, the diamond configuration 
has a flutter-dominated VIV lock-in. This difference in the lock-in behavior can be attributed 
to the boundary layer movement over the front edges of the diamond cylinder, whereby 
the square cylinder has flow separations at the upstream corners 
and inhibits the co-existence of flutter and resonance regimes.

%% eigenvalue trajectory 
% circular, equilateral triangle 
 
\begin{figure}
\centering
\begin{subfigure}{0.495\textwidth}
\centering
    \includegraphics[scale=0.35]{fig20a}
    \caption{}
    \label{fig:topology1_eig1}
    \end{subfigure} 
\begin{subfigure}{0.495\textwidth} 
\centering
 \includegraphics[scale=0.35]{fig20b}
	\caption{}
	\label{fig:topology1_eig23}
	\end{subfigure}	
        \caption{Effect of geometry on the eigenspectrum of ERA-based ROM at $(Re,m^*)=(60,10)$: 
        (a) root loci as a function of the reduced natural frequency $F_s$, 
         where the unstable right-half (Re$(\lambda) > 0$) plane is shaded in grey color  
         and the hollow arrow indicates increasing $F_s$, 
         (b) real and imaginary parts of root loci. 
          WSMI data are denoted by the filled symbol with the same shape as WSMII in (a,b).  
          The onset $U_r$ is computed on $F_s=0.184$ (forward triangle {\protect\bluedashdot}),
           $F_s=0.216$ (diamond {\protect\reddashdot}), and $F_s=0.239$ (ellipse {\protect\greendashdot}), in the real parts of the 
           root loci in (b), respectively.}
	\label{fig:topology1}
\end{figure}


 %%%%%%% figure for VIV FEM
 
 \begin{figure}
\centering
\begin{subfigure}{0.495\textwidth}
\centering
    \includegraphics[scale=0.35]{fig21a}
    \caption{}
    \label{fig:train1_y}
    \end{subfigure} 
\begin{subfigure}{0.495\textwidth} 
\centering
 \includegraphics[scale=0.35]{fig21b}
	\caption{}
	\label{fig:train1_f}
	\end{subfigure}	
%\begin{subfigure}{0.495\textwidth} 
%\centering
% \includegraphics[scale=0.3]{trian1_cl}
%	\caption{}
%	\label{fig:trian1_cl}
%	\end{subfigure}	
%\begin{subfigure}{0.495\textwidth} 
%\centering
% \includegraphics[scale=0.3]{trian1_phi}
%	\caption{}
%	\label{fig:trian1_phi}
%	\end{subfigure}
        \caption{VIV results of the forward triangle configuration using 
        FOM at $(Re,m^*)=(60,10)$: variation of 
      (a) normalized vortex shedding frequency $f$ and 
      (b) rms value of lift coefficient ($C_l$) and maximum amplitude ($Y_{max}$).
      The lock-in is shaded in grey color.}
	\label{fig:trian1_re60_fem}
\end{figure}

For the forward triangle configuration, it is notable that the WSMI and WSMII remain unstable 
for $F_s \le 0.184$ or $U_r \ge 5.43$, as predicted by the ERA-based ROM, 
which indicates that flutter-dominated-VIV persists.
Therefore, the forward triangle is of particular interest for the present study.
%
These linear stability results have been confirmed by the FOM simulations, as shown 
in figure \ref{fig:trian1_re60_fem}. The amplitude grows continually for $F_s \le 0.184$ or $U_r \ge 5.43$, 
and the lift coefficient reaches to the maximum value at the lock-in onset $U_r$, which is similar to the circular 
and square cylinders. The vortex shedding frequency starts to synchronize with 
the structure natural frequency and there exists 1:1 frequency synchronization whereby 
the body is synchronized with the vortex shedding frequency. 
The transverse amplitude grows continually as $F_s$ decreases ($U_r$ increases), 
which is referred to as the galloping-dominated flow-induced vibration. 
This galloping regime is characterized by a low frequency and a high amplitude vibration, whereas a circular cross-section is not susceptible to the galloping.
%
For $F_s < 0.085$ ($U_r > 11.76$), the amplitude experiences a small but 
distinct increase in the amplitude, and the 1:3 synchronization gradually appears 
as shown in figure \ref{fig:3harmonic}.
%after the energy switch. 
In this regime, there is net energy transfer from the base flow with the frequency component at three times 
the body oscillation frequency. During this energy transfer, the fluid force performs 
work on the body, which stores the energy in the form of kinetic energy as well as the potential energy in the spring. 
%
To further elucidate the high harmonic response for the forward triangle, figure \ref{fig:3harmonic} 
depicts motion and lift force traces with their corresponding spectra. 
In the figure, there is a  clear third-harmonic frequency in the lift force whereby the body oscillates with a dominant frequency.
Figure \ref{fig:vor_backward} shows the instantaneous vorticity contours at different values 
of reduced natural frequency $F_s$.
The wake mode is 2S for $F_{s}=0.17$, which remains 2S for $F_{s}=0.15$ with somewhat 
increased spacing between the vortices shed alternately from each side of the cylinder.
By further decreasing $F_s$ to 0.1, the strong SM and WM interactions result into the 
larger vibration amplitude and the flow diverges to a wide vortex street.
%%
\begin{figure}
\centering
\begin{subfigure}{0.495\textwidth}
\centering
    \includegraphics[scale=0.35]{fig22a}
    \caption{}
    \label{fig:Fs_05_y}
    \end{subfigure} 
\begin{subfigure}{0.495\textwidth} 
\centering
 \includegraphics[scale=0.35]{fig22b}
	\caption{}
	\label{fig:Fs_05_y_f}
	\end{subfigure}	
\begin{subfigure}{0.495\textwidth} 
\centering
 \includegraphics[scale=0.35]{fig22c}
	\caption{}
	\label{fig:Fs_05_cl}
	\end{subfigure}	
\begin{subfigure}{0.495\textwidth} 
\centering
 \includegraphics[scale=0.35]{fig22d}
	\caption{}
	\label{fig:Fs_05_cl_f}
	\end{subfigure}
        \caption{Full order VIV results for forward triangle with 1:3 synchronization:
        temporal variation of 
        (a) transverse amplitude, and (c) lift coefficient; 
        normalized power spectrum $P$ versus $f^*$ of: 
        (b) transverse amplitude, (d) lift coefficient at $F_s=0.05$, where  $f^*=f/F_s$ 
        is the frequency of lift and transverse displacement normalized 
        by reduced natural frequency $F_s$. 
        A third-harmonic frequency is evident in $C_l$.}
	\label{fig:3harmonic}
\end{figure}

%%%%%%% figure for VIV vorticity contour at different Fs
 
 \begin{figure}
\begin{subfigure}{0.495\textwidth}
\centering
    \includegraphics[scale=0.3]{fig23a}
    \caption{}
    \label{fig:trian1_fs005}
    \end{subfigure} 
\begin{subfigure}{0.495\textwidth} 
\centering
 \includegraphics[scale=0.3]{fig23b}
	\caption{}
	\label{fig:trian1_fs010}
	\end{subfigure}	
\begin{subfigure}{0.495\textwidth} 
\centering
 \includegraphics[scale=0.3]{fig23c}
	\caption{}
	\label{fig:trian1_fs015}
	\end{subfigure}	
\begin{subfigure}{0.495\textwidth} 
\centering
 \includegraphics[scale=0.3]{fig23d}
	\caption{}
	\label{fig:trian1_fs017}
	\end{subfigure}	
        \caption{Full order results for the forward triangle cylinder at  $(Re, m^*)=(60, 10)$:
                Instantaneous vorticity contours at $F_{s}=$ (a) 0.05, (b) 0.1, (c) 0.15, (d) 0.17. 
Contour levels are from -0.5 to 0.5 in increment of 0.077 and the flow is from left to right.}
	\label{fig:vor_backward}
\end{figure}

The geometry of bluff body alters the flow structures significantly 
in the vicinity of base flow, 
resulting into different root loci and subsequently changing the stability properties. For example, as shown in figure \ref{fig:square_eig}, 
the sharp corner of square stabilizes the flow and the resonance mode dominates 
the entire lock-in for $(Re,m^*)=(60,10)$. 
%
It can be further elucidated by 
looking into the magnitude of leading modal velocity 
$|\widehat{U}|$ for the different geometrical configurations. 
As illustrated in figure \ref{fig:mode_topo}, the stability regions
of ellipse, diamond and triangle geometries shift upstream in comparison to  
circular geometry, indicating the coupling 
effect is enhanced. The stationary configurations of ellipse, diamond and forward triangle have 
the quite similar stability regions.  
%
% this part need to be revised carefully 
%{\color{red} As shown in Table~\ref{tab:Recr_topology}, the ellipse requires the smallest the reduced velocity onset of VIV, 
%followed by diamond, triangle, circular and square.}
The forward triangle is easier to perturb from the flow unsteadiness, 
or by a Hopf bifurcation 
for alternate vortex shedding. However, the sharp corners generally stabilize the 
the fluid-structure system, 
as the separation point are constrained and there is 
lesser freedom for the interaction of flow and structural modes. 
Due to this attribute in the forward triangle configuration, 
there is a delay in lock-in onset $U_r$ with the identical operating 
parameters $(Re,m^*)$ as compared to the elliptical cylinder.  

 %%%%%%% figure for leading mode 
 
\begin{figure}
\centering
\begin{subfigure}{0.495\textwidth}
\centering
	 \includegraphics[scale=0.3]{fig24a}
    \caption{}
    \label{fig:ellipse_re60}
    \end{subfigure} 
\begin{subfigure}{0.495\textwidth} 
\centering
     \includegraphics[scale=0.3]{fig24b}
	\caption{}
	\label{fig:square_re60}
	\end{subfigure}	
\begin{subfigure}{0.495\textwidth} 
\centering
     \includegraphics[scale=0.3]{fig24c}
	\caption{}
	\label{fig:trian1_re60}
	\end{subfigure}	
\begin{subfigure}{0.495\textwidth} 
\centering
     \includegraphics[scale=0.3]{fig24d}
	\caption{}
	\label{fig:trian2_re60}
	\end{subfigure}	
     \caption{Stability regions shown by the spatial distribution of pointwise 
     modal velocity amplitude $|\widehat{U}|$ at $Re=60$: contours of modal velocity for
     (a) circle, (b) ellipse,
      (c) diamond, and (d) forward triangle configuration. 
       The flow is from left to right.}
\label{fig:mode_topo}
\end{figure} 


 
\begin{figure}
\centering
    \includegraphics[scale=0.5]{fig25}
     \caption{Stability phase diagram of VIV lock-in for transversely vibrating two-dimensional bluff bodies with 
     smooth curves and sharp corners for $ 30 \le Re \le 100$, $m^*=10$ and $0.05 \le F_s \le 0.25$. Here the solid curve 
     ({\protect\greensolid}) represents the critical Reynolds number ($Re_{cr}$) of the fixed bluff body and 
     flutter- and resonance-induced regimes are demarcated, where 
     ${\color{blue} \square}$ represents the co-existence of flutter and resonance regimes; 
     ${\color{red} \bigcirc}$ denotes resonance regime; $\triangle$ represents flutter regime.
     For $Re > Re_{cr}$, flutter regime comprises both unstable eigenvalues (Re$(\lambda) > 0$) 
     of WM and SM, whereas resonance regime has only unstable WM.
     }
         \label{fig:top_chart}
\end{figure}

%%% chart for different topology 
Figure \ref{fig:top_chart} summarizes the stability regimes of 
transversely vibrating bluff bodies 
for Reynolds number range $ 30 \le Re \le 100$. In this figure, the solid curve
({\protect\greensolid}) depicts 
the trend for the critical Reynolds number. The following observations can be made from 
the stability phase diagram. The geometries of circle, ellipse and diamond exhibit 
the flutter and mixed resonance-flutter modes. In contrast to the square counterpart, 
the diamond geometry has a movement of asymmetric boundary layers on the front lateral 
edges which allows the co-existence of flutter and resonance modes. For the 
elliptic cylinder, the mixed flutter-resonance regime occurs at lower Reynolds number, 
in comparison to the circular cylinder.
Notably, the forward triangle configuration 
only shows the flutter-induced lock-in regime for this range of Reynolds number and the edges are in 
the leeward side with separated flow. 
Finally, the square-section body shows predominant resonance regime for $ 30 \le Re \le 80$, approximately, 
which turns into the flutter state for $Re > 80$.

%%% 3D discussion
The present ERA-based ROM study has been concerned about two-dimensional bluff body configurations 
for which only two directions in space are resolved. All the notions of ROM, such as 
base flows, eigenvalue realization can be easily extended to full 
three-dimensional settings. Thus the present method does not pose any theoretical limitation 
except there may be numerical one with respect to memory requirements and CPU time 
to solve the generalized eigenvalue problem.  
 


%ccccccccccccccccccccccc
\section{Concluding remarks}
In this study, we presented the ERA-based model reduction 
for the coupled fluid-structure analysis to investigate the stability characteristics 
of the vortex-induced vibrations of bluff bodies. The ERA-based ROM relies on
the singular value decomposition of a Hankel matrix constructed from the impulse response of the Navier-Stokes equations.
The present study has remarkably demonstrated the effectiveness of 
ERA-based ROM for predicting the unstable wake flow behind a stationary circular cylinder.
The critical Reynolds number and the flow dynamics were well predicated and an 
excellent agreement was found with the full order model and the available literature.
%as it does not need an adjoint solver.
We next employed the ROM for a unified description of the lock-in phenomenon 
as a function of Reynolds number $Re$, the mass ratio $m^*$, 
and for the investigation of the effect of rounding  and various geometrical shapes.
%% Development of ERA-based ROM
%
To investigate the VIV mechanism, the ERA-based ROM has been extended to construct 
the fluid ROM and coupled with a linear structure to form a reduced fluid-structure system 
in the state space format. 
%
Two distinct lock-in patterns of flutter- and resonance-induced regimes 
were investigated by the ERA-based ROM 
for a transversely vibrating circular cylinder at the baseline parameters of 
$(Re,m^*)=(60,10)$.
While the resonance state has the unstable WM together with the stable SM  
in the range $0.11 \le F_{s} \le 0.147$, the flutter regime 
has the co-existence of the unstable  SM and WM in the range $0.147 < F_{s} \le 0.179$.
In comparison to the linear ROM used in \cite{Zhang2015},
which is sensitive to the training trajectory, 
the proposed ERA-based ROM is sufficiently accurate and only requires the impulse response of unstable fluid system. 
To generalize the proposed ERA-based ROM for the VIV linear stability analysis, 
the effects of Reynolds number, the rounding of square cylinder 
and the geometry have been systematically examined and compared against the full order simulations.
Based on the systematic parametric study, the following conclusions can be drawn:
% effect of Re 
\begin{enumerate}
\item{
The study on the effect of Reynolds number demonstrates
that the flutter and resonance regimes do not always exist during the lock-in phenomenon. 
For $m^*=10$, it was found that the flutter and resonance regimes 
co-exist for $Re_{cr} < Re < 70$. 
The flutter-induced regime gradually dominates the entire lock-in region 
when $Re \ge 70$. The finding is consistent 
with the theoretical analysis of \cite{DeLangre2006} for high Reynolds number. 
The stability region provides an explanation that it shifts upstream as $Re$ increases,
indicating the coupling between unstable wake and bluff body is enhanced. 
Another observation is that the reduction of rms value of lift force 
is not associated with either resonance or flutter state, but closely related to the 
phase angle jump from $0^0$ to $180^0$, which is also demonstrated during the lock-in of
square cylinder.}

%% Effect of rounding
\item{ 
The analysis on the rounding effect of square cylinder study at $(Re,m^*)=(60,10)$ 
shows that the rounding 
has a remarkable impact on the flutter and resonance regimes.
The flutter regime can be promoted by gradually removing the sharp corners. 
The sharp corners suppress the continuous movement of separation points and 
the WM loop of the square cylinder is smaller than the circular cylinder counterpart.
As the rounding radius $r_s$ increases, the SM trajectory moves gradually towards the 
positive real axis and the flutter region starts to appear. From the comparison of 
leading modes, the stability region shifts downstream as the rounding radius $r_s$ decreases.
There is a reduction in the coupling strength between the fluid and structure due to 
the presence of sharp corners, which inhibit the movement of separation points 
and the susceptibility of inertial coupling. In comparison to the VIV of circular cylinder, 
this study also explains why the lock-in onset $U_r$ is larger and the lock-in region is narrower for a square cylinder.   
}

%% Effect of topology
\item{
The geometry study reveals that the cross-sectional shape significantly alters the VIV and galloping instability.  
The ERA-based ROM can effectively capture the stability properties and the lock-in regimes 
for the elliptical, forward triangle and diamond-shaped configurations.
It is found that root loci of WM and SM coalesce and form coupled modes for these geometries. 
We have provided further 
insights into the phenomena of flow-induced vibration by the ERA-based ROM for these bluff-body geometries.
The elliptical cylinder was found to have the lowest $U_r$ for the lock-in onset 
followed by the diamond and the forward triangle configurations.
Of particular note for the forward triangle VIV, 
the ROM predicts the flutter-dominated VIV persists for $F_s \le 0.184$ or $U_r \ge 5.43$. 
A low-frequency galloping instability and a kink in the amplitude response associated
with 1:3 synchronization were observed  in the forward triangle configuration.
We presented a summary phase diagram to characterize the effects of geometry
on the VIV stability regimes based on the eigenspectrum distribution.  Such phase diagram based 
on the linear dynamics of lock-in process provides insights to develop a unified description of flow-induced vibration.
The phase diagram shows that the resonance mode only exists for a certain range of Reynolds number. 
The VIV lock-in mechanism is eventually dominated by the flutter mode as the Reynolds number increases. 
}
\end{enumerate}

% General impact and way forward
The proposed ERA-based ROM is demonstrated to be accurate and efficient for 
the VIV linear stability analysis of bluff bodies, 
which has a relevance in the development of flow control strategies.
By shifting the unstable eigenvalues of WM and SM 
to the stable left half complex plane, suppression of vortex street and VIV 
can be achieved by the model. The simplicity of the model permits investigation of a range of geometries 
and parameters on VIV mechanism and paves the way for a bottom up approach to develop 
control devices.
%
We want to emphasize that we have considered only linear dynamical systems in this study, 
possible future direction should include an extension of the system identification process
to nonlinear systems.
%
In the future, there is also need for further investigation at high Reynolds number
to expand the proposed model reduction approach for a generalized lock-in description 
with a wider parameter space of mass-damping parameter.

%\section*{\bf{Acknowledgements}}
The first author would like to thank Singapore Maritime Institute Grant (SMI-2014-OF-04) 
for the financial support.

\appendix
\renewcommand\thefigure{B.\arabic{figure}}
\renewcommand{\theequation}{A.\arabic{equation}}
\section*{Appendix A: Derivation of Phase Angle for VIV}
By considering the cylinder motion and fluid forcing as sinusoidal functions, the displacement and lift coefficient can be obtained  for the VIV linear system as
\begin{equation}
\left. \begin{array}{ll}
\displaystyle Y=\hat{Y} e^{\lambda_r t}\cos(\lambda_i t)  \\[8pt]
\displaystyle C_l=\hat{C_l} e^{\lambda_r t}\cos(\lambda_i t + \phi)
\end{array}\right\},
\label{eq:linear_y_cl}
\end{equation}
where $\lambda=\lambda_r+i\lambda_i$ is eigenvalue with real $\lambda_r$ 
and imaginary $\lambda_i$ components, $\hat{Y}$ and $\hat{C_l}$ 
denote the magnitudes of eigenmodes. The phase angle $\phi$ can be derived by plugging Eq. (\ref{eq:linear_y_cl}) 
into the structural Eq. (\ref{eq:structure1}) as
\begin{equation}
\begin{array}{cc}
[\hat{Y}e^{\lambda_r t} (\lambda_r^2 - \lambda_i^2 + 4 \pi \zeta F_s \lambda_r + (2 \pi F_s)^2) 
- \frac{a_s \hat{C_l} e^{\lambda_r t} \cos{\phi}}{m^*}] \cos{\lambda_i t} \\
+ [\hat{Y}e^{\lambda_r t}(-2 \lambda_r \lambda_i - 4 \pi \zeta F_s \lambda_i) + 
 \frac{a_s \hat{C_l} e^{\lambda_r t} \sin{\phi}}{m^*}] \sin{\lambda_i t} = 0.
\end{array}
\label{eq:phase1}
\end{equation}
From equating the coefficients of $\cos(\lambda_i t)$ and $\sin(\lambda_i t)$ to zero, 
we obtain the following relations:
\begin{equation}
\begin{array}{cc}
\hat{Y}e^{\lambda_r t} (\lambda_r^2 - \lambda_i^2 + 4 \pi \zeta F_s \lambda_r + (2 \pi F_s)^2) 
- \frac{a_s \hat{C_l} e^{\lambda_r t} \cos{\phi}}{m^*} = 0, \\
\hat{Y}e^{\lambda_r t}(-2 \lambda_r \lambda_i - 4 \pi \zeta F_s \lambda_i) + 
 \frac{a_s \hat{C_l} e^{\lambda_r t} \sin{\phi}}{m^*} = 0.
\end{array}
\label{eq:phase2}
\end{equation}
By solving Eq. (\ref{eq:phase2}) and setting $\zeta = 0$, 
$\sin{\phi}$ and $\cos{\phi}$ can be obtained as
\begin{equation}
\begin{array}{cc}
\sin{\phi} = \frac{2 \hat{Y} \lambda_i \lambda_r m^*}{a_s \hat{C_l}}, \\
\cos{\phi} = \frac{\hat{Y} m^*(\lambda_r^2 - \lambda_i^2 + (2 \pi F_s)^2)}{a_s \hat{C_l}}.
\end{array}
\end{equation}
Through the trigonometric identity $\cos^2{\phi} + \sin^2{\phi} = 1$, 
the term $\sin{\phi}$ can be further simplified in terms of $(\lambda, F_s)$ as follows:
%\begin{equation}
%\sin{\phi} = \frac{2\lambda_r\lambda_i}{\sqrt{(\lambda_r^2  \\
% - \lambda_i^2 + (2\pi F_s)^2)^2 + (2\lambda_i \lambda_r)^2 }}
%\end{equation}
\begin{equation}
\sin{\phi} = \frac{2\lambda_r\lambda_i}{\sqrt{(\lambda_r^2  \\
+ (2\pi F_s)^2 + \lambda_i^2)^2 - (4\pi\lambda_i F_s)^2 }}.
\end{equation}
%% 
\section*{Appendix B: Effect of mass ratio}
\setcounter{figure}{0}
For the illustration of ERA-based ROM, the effect of mass ratio is shown in
figure \ref{fig:dfms_eig} for $m^*=(5,7.6,20)$ at $Re=60$. 
Figure \ref{fig:dfms_eig23} shows the real and imaginary parts of the root loci as 
a function of reduced frequency $F_s$. 
It indicates the lock-in onset starts to move 
to the lower reduced velocity ($U_r=1/F_s$) regime as 
the mass ratio $m^*$ decreases. 
As expected, owing to weaker fluid-structure coupling for 
larger mass ratio $m^*=20$, the eigenfrequency of WM recovers to 
the frequency of stationary cylinder and the frequency of SM approaches to 
the natural frequency of the cylinder-only system as $F_s$ increases. 
%% branches merging 
As the mass ratio decreases further, 
figure \ref{fig:dfms_eig1} shows that the root loci of SM and WM
gradually \emph{coalesce} and form a coupled mode due to the 
increased strength of fluid-structure coupling.
The approximate threshold mass ratio is $m^{*} \approx 7.6$ for 
the coupled mode, which is very 
close to the predicted $m^{*}=7.3$ in \cite{Zhang2015}.  %The eigenfrequency 
The phenomenon is termed as mixed WM/SM \citep{meliga2011} 
or coupled fluid-elastic mode \citep{mittal2016}. 
As illustrated in figure \ref{fig:dfms_eig23}, for the mass ratio $m^*=5$, 
the coupled wake-structure modes WSMI and WSMII 
resemble to the SM and WM, respectively for $F_s \le 0.175$, 
whereas WSMI and WSMII resemble to the standard
WM and SM, respectively for $F_s > 0.175$. 
%
This finding suggests that 
the stability roles of WM and SM switch at a specific value of $F_s$, 
where the two growth rate Re$(\lambda)$ curves of WSMI and WSMII intersect. 
%
The flutter regime for $m^*=5$ is then defined by 
$0.165 < F_s \le 0.197$ ($5.08 \le U_r < 6.07$),
as shown in figure \ref{fig:dfms_eig23}, which
matches well with the results of \cite{mittal2016} ($5.0 \le U_r < 6.0$) 
at the identical conditions.  
%
In figure \ref{fig:dfms_eig23} (bottom),
the two curves of Im$(\lambda)$ for $m^*=(5,7.6)$ no longer cross and there is 
a characteristic anticrossing with a frequency splitting ($\Delta f$) between the WSMI and WSMII.
This phenomenon of anticrossing is an intrinsic property of strong coupling 
at low mass ratio, as reported for generic coupled mechanical 
oscillators in \cite{novotny2010}. 

\begin{figure}
\centering
\begin{subfigure}{0.495\textwidth}
\centering
    \includegraphics[scale=0.3]{fig26a}
    \caption{}
    \label{fig:dfms_eig1}
    \end{subfigure} 
\begin{subfigure}{0.495\textwidth} 
\centering
 \includegraphics[scale=0.3]{fig26b}
	\caption{}
	\label{fig:dfms_eig23}
	\end{subfigure}	
        \caption{Effect of mass ratio on the eigenspectrum of ERA-based ROM for a circular cylinder at $m^*=(5,7.6,20)$ and $Re=60$: 
        (a) root loci as a function of the reduced natural frequency $F_s$, 
         where the unstable right-half (Re$(\lambda) > 0$) plane is shaded in grey color  
         and the hollow arrow indicates increasing $F_s$, 
         (b) real and imaginary parts of root loci. 
        In (b), growth rate Re$(\lambda)$ curves of WSMI and WSMII intersect at $F_s=0.175$ ({\protect\reddash}) and 
        the flutter regime is defined between $F_s=0.197$ ({\protect\reddashdot})
        and $F_s=0.165$ ({\protect\reddot}) for $m^*=5$.  The frequency anticrossing 
        is shown in the inset of Im$(\lambda)$ plot.
        The WSMI and SM branches are denoted by the filled symbol with the same shape 
        as WSMII and WM in (a,b). }
	\label{fig:dfms_eig}
\end{figure}



\bibliographystyle{jfm}
% Note the spaces between the initials
\bibliography{refs}

\end{document}
