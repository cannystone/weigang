% This is file JFM2esam.tex
% first release v1.0, 20th October 1996
%       release v1.01, 29th October 1996
%       release v1.1, 25th June 1997
%       release v2.0, 27th July 2004
%       release v3.0, 16th July 2014
%   (based on JFMsampl.tex v1.3 for LaTeX2.09)
% Copyright (C) 1996, 1997, 2014 Cambridge University Press

%%%%%%%%%%%%%%%%%%%%%%%%%%%%%%%%%%%%%%%% package include 


\documentclass[standard]{jfm}
\usepackage{graphicx}
\usepackage{gensymb}
\usepackage{epstopdf, epsfig}
\usepackage{color,soul} %highlight with color
\graphicspath{{./figures_num/}} % Specifies the directory where pictures are stored
\newtheorem{lemma}{Lemma}
\newtheorem{corollary}{Corollary}
\usepackage{amssymb}
\usepackage{amsmath}
\usepackage{natbib}
\usepackage{subfloat}
\usepackage{subcaption}
\usepackage{bm}
\usepackage{multirow}
\usepackage{xcolor}
\usepackage{tikz}
%\usepackage{hyperref}
%\usepackage{url}
%\usepackage{amsbsy}
%\usepackage{amsfonts}
%\usepackage{booktabs,array,dcolumn} 
%\usepackage{ulem}

\newsavebox{\largestimage}
%%======================New commands============================================
\newcommand{\todo}[1]{\textcolor{blue}{#1}}
\newcommand{\changes}[1]{\textcolor{magenta}{#1}}     %%defined a new function to trace changes by modfiy font color
\def\vec#1{\mbox{\boldmath $#1$}}
\newcommand{\bx}{\mathbf{\chi}}
\newcommand{\bu}{\mathbf{u}}
\newcommand{\bw}{\mathbf{w}}
\newcommand{\bn}{\mathbf{n}}
\newcommand{\bnx}{\mathbf{n_x}}
\newcommand{\bny}{\mathbf{n_y}}
\newcommand{\bs}{\boldsymbol{\sigma}}
\newcommand{\disnum}[1]{\texttt{#1}}
\def\Otf{\Omega^\mathrm{f}(t)}
\def\bz{\mathbf z}
\newcommand{\xx}{\mbox{$\mathbf{x}$}}
\def\strain{\vec \epsilon}
\def\stress{{\vec \sigma}}
\def\div{\vec \nabla}

\usepackage{environ}
\NewEnviron{myequation}{%
\begin{equation}
\scalebox{0.7}{$\BODY$}
\end{equation}
}
%\usepackage{placeins}
%\usepackage{tabularx}
%\usepackage{epstopdf}
%\usepackage{afterpage}
%%=======================================================================================
\newcommand{\red}{\raisebox{2pt}{\tikz{\draw[red,solid,line width=1.2pt](0,0) -- (5mm,0);}}}
\newcommand{\green}{\raisebox{2pt}{\tikz{\draw[green,solid,line width=1.2pt](0,0) -- (5mm,0);}}}
\newcommand{\blue}{\raisebox{2pt}{\tikz{\draw[blue,solid,line width=1.2pt](0,0) -- (5mm,0);}}}
% For multiletter symbols
%\newcommand\Real{\mbox{Re}} % cf plain TeX's \Re and Reynolds number
%\newcommand\Imag{\mbox{Im}} % cf plain TeX's \Im
%\newcommand\Rey{\mbox{\textit{Re}}}  % Reynolds number
%\newcommand\Pran{\mbox{\textit{Pr}}} % Prandtl number, cf TeX's \Pr product
%\newcommand\Pen{\mbox{\textit{Pe}}}  % Peclet number
%%%%%%%%%%%%%%%%%%%%%%%%%%%%%%%%%%%%%%%%

\shorttitle{Feedback control of vortex-induced vibration}
\shortauthor{W. Yao and R. K. Jaiman}

\title{Feedback Control of Unstable Flow and Vortex-Induced Vibration Using  
       Eigensystem Realization Algorithm}

\author{W. Yao
  \and  R. K. Jaiman
   \corresp{\email{mperkj@nus.edu.sg}}
  }
 
\affiliation{Department of Mechanical Engineering, National University Singapore, Singapore 119077}

%%%%%%%%%%%%%%%%%%%%%%%%%%%%%%%%%%%%%%%%

\begin{document}
\maketitle

%%%%%%%%
\begin{abstract}
% Premise: Key elements of VIV are linear and can be controlled by active blowing/suction.
We present an active feedback blowing and suction (AFBS) procedure 
via model reduction for unsteady wake 
flow and the vortex-induced vibration (VIV) of circular cylinders. 
The reduced-order model (ROM) for the AFBS procedure is developed by 
the eigensystem realization algorithm, which provides a low-order representation of 
the unsteady flow dynamics in the neighbourhood of equilibrium steady state. 
%
The actuation is considered via vertical suction and blowing jet at the porous surface
of circular cylinder with the body mounted force sensor. 
While the optimal gain is obtained using the linear quadratic regulator(LQR),
the Kalman filtering is employed to estimate the approximate state vector. 
%
The feedback control system shifts the unstable eigenvalues of the wake flow and 
the VIV system to the left half complex plane, and subsequently
results into the suppression of the vortex street and the VIV in 
elastically  mounted structures. 
The resulting controller designed by linear low-order approximation is 
able to suppress nonlinear saturated state of wake vortex shedding from the circular cylinder. 
%
A systematic linear ROM-based  stability analysis is performed to understand the eigenvalue 
distribution for the flow past stationary and elastically mounted circular cylinders. 
%
Results from the ROM analysis are consistent with those obtained from full nonlinear 
fluid-structure interaction simulations, thereby confirming the validity of the proposed ROM-based AFBS procedure.
%
A sensitivity study on the number of suction/blowing actuators, 
the angular arrangement of actuators, 
and the combined versus independent control architectures has been performed for the 
flow past a stationary circular cylinder.
%
Overall, the proposed control concept based the ERA-based ROM and LQR algorithm
is found to be effective to suppress the vortex street and 
the vortex-induced vibration for a range of reduced velocity and mass ratio. 

\end{abstract}

\begin{keywords}
feedback control, low-dimensional models, vortex-induced vibration
\end{keywords}

%%%%%%%%
\section{Introduction}\label{sec:intro}
% Passive VIV control (not effective)
Successful control of vortex-induced vibration (VIV) can lead to 
safer and cost-effective structures in offshore, aeronautical and civil engineering. 
In the past several decades, various passive control techniques \citep{owen2001,Choi2008,Baek2009,Yu2015,Law2017} have been explored 
via geometry modification and by adding auxiliary surfaces to alter the flow dynamics without any energy input. 
While the passive VIV control methods offer some simplicity, 
they do not have the ability to work on-demand and may not be effective from 
the perspective of wake stabilization, drag reduction and VIV suppression 
in a wide range of operational conditions.
More importantly, the passive devices e.g. strakes, splitter plate or fairings 
are not easy to implement for certain situations such as  square-shaped 
multicolumn offshore platforms \citep{chakrabarti} and subsea pipelines undergoing VIV in the proximity with seabed floor \citep{sumer}. 
%
The motivation of this study is to develop a feedback active 
control algorithm based on reduced-order model and to demonstrate the algorithm 
to stabilize the wake flow and the VIV for a canonical two-dimensional circular cylinder 
problem. To investigate the proposed control scheme, we consider the flow past a 
stationary circular cylinder at low Reynolds number, $Re=UD/\nu$ 
and the vibrating cylinder as a function of reduced velocity, $U_r=U/(f_{N}D)$  and 
mass ratio, $m^*=4 m/({\rho \pi D^{2}})$ with zero mass-damping parameter, $m^* \zeta = 0.0$. 
Here $U$, $\rho$, $m$, $D$, $f_{N}$, $\zeta$ and $\nu$ are the freestream velocity, 
the density of fluid, the mass per unit length of cylinder, the diameter of cylinder, 
the structural natural frequency, the damping ratio and the kinematic viscosity, respectively.

% Need for active control and blowing/suction
%Different types of flow control strategies have been developed and successfully implemented
%for the flow past a circular cylinder, such as optimal control techniques, full-state feedback
%control, neural networks, and proportional closed-loop control.
%For an active flow control of the cylinder wake, various actuation mechanisms are employed,
%such as cylinder rotation, crossflow motion, and fluidic actuation on the cylinder surface. 
Through external input of small tunable energy into the surrounding flow, active VIV control 
techniques offer a better alternative due to their adaptive and efficient performance.
% Windward suction leeward blowing 
\cite{Kim2005} utilized blowing/suction slots placed on the top and bottom of the circular cylinder to
stabilize the unstable wake. The applied forcing in the slots was sinusoidal along the spanwise direction but kept steady in time.
In another interesting study of \cite{Dong2008} the combined steady windward suction and leeward blowing (WSLB) was found as effective strategy to
eliminate the vortex street and to suppress vortex-induced vibration in cross-flow direction.
%
This WSLB method in \cite{Dong2008} is essentially an active flow control (i.e. external energy 
is required to maintain steady suction and blowing), but can be customized as a passive 
technique by deploying porous surfaces to form connecting channels between the windward
and leeward stagnation points of a circular cylinder.
%
The aforementioned blowing/suction control schemes
provide only open-loop alteration of unsteady flow 
whereby the control input is prescribed and is independent of the flow states.
In another recent study, \cite{Wang2016b} utilized a windward-suction 
and leeward-blowing feedback control to suppress VIV in both inline and cross-flow 
directions at a specific condition $(m^*,U_r)=(2,5)$. 
However, it is not certain whether the system is free of VIV
for a range of reduced velocity $U_r$ and mass ratio $m^*$. 
Moreover the simplified closed-loop control strategy based  on 
the proportional (P), integral (I) and proportional-integral (PI) schemes 
were applied to manipulate the WSLB velocities through the standard 
deviation of surrounding flow velocity. The PI control was found to outperform the 
P and I control schemes with respect to the effectiveness of VIV suppression.
To the best of our knowledge, adaptive feedback control of VIV via vertical suction and blowing 
has not been studied earlier. Furthermore, adaptive feedback control of VIV based on 
the model reduction is not explored in earlier studies, which can be important 
for both numerical and experimental settings. 
% linear ROM 

Various types of active flow control strategies have been explored for the flow past a circular cylinder, such as full-state feedback control, neural networks, and proportional closed-loop 
feedback control. 
In particular, active feedback or closed-loop control of unsteady flow past over a bluff body has 
been recently investigated via numerical simulations \citep{Ahuja2010,parkin2014,Flinois2016}.  
To implement the optimal linear control in an efficient manner, it is imperative to 
develop a low-order linear model by retaining the significant dynamics of 
the original system (\cite{Ahuja2010,Flinois2016}). 
Linear ROM provides a way to trace the eigenspectrum of dynamical system,
while maintaining about an order of magnitude efficiency improvement 
to construct the essential dynamics of the system. Instead of deriving the
system model directly from the linearized governing equations of the fluid flow, 
the system identification method is particularly desirable, because 
it is non-intrusive and can be used directly into an existing Navier-Stokes (NS) solver 
to generate approximate system matrices using only the input-output data sequences. 
Eigensystem realization algorithm (ERA) \citep{Juang1985} is a well-established system identification method to
construct linear ROM from stable system linear impulse response. \cite{Ma2011} proved that the ROM constructed by
ERA is mathematically equivalent to balance truncation. Recently, \cite{Flinois2015} provided a mathematical rigor that 
unmodified balance truncation (designed for stable system) is applicable for unstable system. Based on the work of 
\cite{Ma2011} and \cite{Flinois2015}, the ERA is used for the stabilization of unstable wake flow \citep{Flinois2016}.  
%in which 
%the actuation has been modelled as a body force and not an practical actuation representation. 
%%%% contribution
 
Although the suction/blowing control strategy has been extensively studied for unstable flow and 
VIV past a circular cylinder both numerically \citep{Kim2005,Dong2008,mao2015,Wang2016b} 
and experimentally \citep{Fransson20041031,Chen201325,Chen2015},
the earlier research relies on open-loop control strategy, except the recent study 
of \cite{Wang2016b} via simplified non-adaptive control procedure. \cite{Ahuja2010,Flinois2016} designed a feedback control law 
to stabilize unstable wake flow over a plate and bluff body using model 
reduction method, however the actuation is modelled as a body force and 
may not be implemented as a practical actuator.
%%
The recognition of linear mechanism during the self-sustaining behavior of VIV in our recent study \citep{YaoJFM2016} inspires us to develop a feedback suppression strategy 
based on a linear control theory. 
Motivated by the insight about the frequency lock-in process during VIV, 
it is possible to design a linear controller
to minimize the unsteady vortex-shedding and the VIV effects. For that purpose,
the control input based on vertical blowing and suction at the surface 
can be utilized \citep{Kim2005}. 

%%
The key contribution of the paper is to develop an active feedback blowing and suction (AFBS) procedure based on ERA-based ROM to control the wake instability and the vortex-induced vibration. 
Using the ERA method, a low-order fluid model is constructed, 
and is coupled with the structure via the LQR optimal control scheme. The proposed 
AFBS procedure ensures a VIV free system within a large parameter space of reduced 
velocity ($U_r$) and mass ratio ($m^*$) and it can handle both one degree-of-freedom 
(1-DOF) and two degree-of-freedom (2-DOF) VIV. 
%The designed controller is then demonstrated to stabilize the unsteady wake flow and VIV. 
The results from the ROM solver are compared with those of full-order 
simulations based on the incompressible Navier-Stokes equations. 
We employ the ROM model to predict the performance of the AFBS procedure through 
the eigenvalue distribution in the complex plane.
The present study is based on two questions pertaining to the VIV physics and control: 
(i) can we suppress self-sustaining VIV by assuming linear lock-in mechanism?
(ii) how much additional energy is required for the suppression of VIV 
compared to the stationary nonlinear vortex shedding state? 

The organization of the article is as follows.
The numerical details of full order model based on the Navier-Stokes equations and the ERA-based reduced order model are presented in section \ref{sec:method}. 
Section \ref{sec:results} introduces the active feedback control 
based on the vertical suction and blowing for the unstable wake flow 
of stationary cylinder and vortex induced vibration. 
A sensitivity study on the number of 
suction/blowing actuators, the angular arrangement of actuators 
and the  combined versus independent control architecture 
are also investigated in Section \ref{sec:results}. 
Section \ref{sec:concluding} summarizes the main conclusions. 

\section{Numerical methodology}\label{sec:method}
\subsection{Problem setup}
Figure \ref{fig:cylinder} shows a schematic of problem setup used in our simulation study 
for a flexibly mounted circular cylinder in a flowing stream. 
%The coordinate origin is located at the geometric center of the circular cylinder, 
%the streamwise and transverse directions are denoted $x$ and $y$, respectively.
At the inlet boundary $\mathrm{\Gamma_{in}}$, a stream of incompressible fluid 
enters into the domain at a horizontal velocity $(u,v)=(U,0)$, where $u$ and $v$ 
denote the streamwise and transverse velocities in $x$ and $y$ directions, respectively. 
For the VIV configuration, the circular cylinder with mass $m$ is elastically mounted on a 
linear spring and is allowed to vibrate only in the transverse direction. 
No-slip wall condition is implemented on the surfaces of the bluff body, and a traction-free
boundary condition is implemented along the outlet $\mathrm{\Gamma_{out}}$, 
while the slip wall condition is implemented on the top $\mathrm{\Gamma_{top}}$
and bottom $\mathrm{\Gamma_{bottom}}$ boundaries.
Except stated otherwise, all length scales are normalized by 
the cylinder diameter $D$ and velocities
with the free stream velocity $U$.
%
The numerical domain extends from $-10D$ at the inlet to $30D$ at the outlet, 
and from $-15D$ to $15D$ in the transverse direction.
%For simplicity, $D=1$ is used for all computations below. 
For the flow control, we consider blowing and suction on the porous cylinder 
surface through fluidic actuators. It is known that 
the moderate levels of suction/blowing into the surrounding flow can 
have a great impact on the boundary layer, the separation point and  
wake characteristics \citep{Fransson20041031,Chen201325,Chen2015}. 
Through the active feedback control, the suction 
mechanism can delay the separation of boundary layer 
(i.e., narrower wake width and reduced drag), whereas 
blowing can have an opposite effect. On the other hand, continuous blowing generally 
tends to decrease the Strouhal number for the flow around a porous cylinder, 
continuous suction has the opposite influence on the vortex shedding 
frequency \citep{Fransson20041031}.

In the present study,  we propose a feedback control based on  
a configuration with three pairs of suction/blowing actuators,
 as depicted in figure \ref{fig:blowing_suction}. 
%
In this proposed configuration, termed as $BS0$, there 
are total six suction and blowing slots distributed over the cylinder surface, 
with a pair of slots at the windward $\theta=(135 \degree, 225 \degree$), 
at the midward $\theta=(90 \degree, 270 \degree)$
and at the leeward $\theta=(45 \degree, 315 \degree)$ sides. 
Here  $\theta$ is deviation angle between the centerline of each suction/blowing 
actuation with respect to the base suction point.
%While $BS0$ and $BS1$ form a  symmetric configuration,  
%$BS2$ and $BS3$ are asymmetric with respect to the quadrants of cylinder.
As shown in figure \ref{fig:blowing_suction}, the positive control input is defined as 
suction from the bottom of the cylinder and blowing at the top surface. 
Similar to \cite{pastoor2008}, we consider the actuation slot width to be 
$\sigma_{c}=\pi D/72$, whereby the energy supply from the actuation 
is characterized by the momentum coefficient as
$C_\mu = {2 N \rho V^2_{c} \sigma_c}/{(\rho U^2 D)}$, 
where $N$ is the number of slots, $V_c$ denotes the time dependent 
suction and blowing velocity. 
Owing to the body-conforming Lagrangian-Eulerian coupling for fluid-structure interaction, 
the actuation conditions for the blowing and suction are accurately enforced by 
the Dirichlet boundary condition.
%as compared to previous studies based on body force \citep{Flinois2016} 
%or counter rotating vortices \citep{Ahuja2010} in the immersed boundary framework.
%  
The present full-order fluid-structure model relies on 
a variational finite-element formulation and a semi-discrete time stepping.
While the Navier-Stokes equations are discretized in space 
using $\mathbb{P}_{n}/\mathbb{P}_{n-1}$ iso-parametric finite elements 
for the fluid velocity and pressure, 
the second-order backward scheme is used for the time discretization, where 
$\mathbb{P}_{n}$ denotes the standard $n^\mathrm{th}$
order Lagrange finite element space on the discretized fluid domain. 
%
Details of the numerical techniques, the verification and the mesh convergence study 
based on $\mathbb{P}_{2}/\mathbb{P}_{1}$ 
iso-parametric elements are documented in \cite{YaoJFM2016}.

\begin{figure}
 \centering
  % Store largest image in a box
  \savebox{\largestimage}{\includegraphics[scale=0.6]{fig1a}}%
    
%  \hspace{1em}
%  \hfill
  \begin{subfigure}[b]{0.495\textwidth}
    \centering
    \usebox{\largestimage}
    \caption{}
   \label{fig:cylinder}  
  \end{subfigure}
  \begin{subfigure}[b]{0.495\textwidth}
    \centering
    % Adjust vertical height of smaller image
    \raisebox{\dimexpr.5\ht\largestimage-.5\height}{%
      \includegraphics[scale=0.7]{fig1b}}
    \caption{}
  \label{fig:blowing_suction}  
  \end{subfigure}
  \caption{Problem setup for feedback control of unsteady wake and vortex-induced vibrations: 
           (a) computational domain and boundary conditions for 
           the flow past a freely vibrating cylinder in uniform flow,
           (b) proposed new actuator configuration $BS0$
           through blowing/suction over the porous surface of circular cylinder. 
           The positive control input is defined as 
           suction from bottom and blowing at the top of cylinder. 
%           While $BS0$ and $BS1$ form a  symmetric configuration,  
%          $BS2$ and $BS3$ are asymmetric with respect to the quadrants of cylinder.
}
\label{fig:cylinder_suction}          
\end{figure}

%%

%% actuator arrangement 
%\begin{figure}
%\centering
%\includegraphics[scale=0.5]{suction2}
% \caption{Blowing and suction actuation slots. The positive control input is defined as
% suction from bottom and blowing at the top of the cylinder. $BS0$ and $BS1$ are symmetric; 
% $BS2$ and $BS3$ are asymmetric.}
%\label{fig:blowing_suction}  
%\end{figure}

\subsection{Full-order model}
For the sake of completeness, 
we first present the full-order model (FOM) based on the NS equations
for the moving incompressible viscous fluid domain $\Otf$ as
\begin{align}
\rho \left( \frac{\partial \bu }{\partial t} \bigg\rvert_{{\bx}} + \left(\bu -\bw\right)\cdot \boldsymbol{\nabla} \bu
 \right)= 
 \boldsymbol{\nabla} \cdot \boldsymbol{\sigma} \mbox{ on } \Otf, \label{eq:N-S} \\
\boldsymbol{\nabla} \cdot\bu = 0 \mbox{ on } \Otf, \label{eq:continuity}
\end{align}
where the time derivative is taken 
with the referential coordinate ${\bx}$ held fixed and 
the Cauchy stress tensor for a Newtonian fluid is 
$\boldsymbol{\sigma} = -p\mathbf{I} + \mu \left(\boldsymbol{\nabla} \bu 
+ \left(\boldsymbol{\nabla} \bu \right)^T
\right) $. Here $p$, $\bu$, $\bw$,  $\mu$ and $ \mathbf{I} $  denote the fluid pressure, 
the fluid velocity, the mesh velocity, the dynamic viscosity and 
the identity tensor, respectively. 
%
The mesh nodes on the fluid domain $\Omega^f(t)$ are updated by solving a linear steady pseudo-elastic material model
\begin{align}
\label{MeshEq}
\div \cdot \stress^\mathrm{m} = \vec{0},
\end{align} 
where $\stress^\mathrm{m}$ is the stress experienced by the ALE mesh due to the strain induced by the 
rigid-body movement, which is defined as,
\begin{align}
\label{meshEq}
\qquad \stress^\mathrm{m} = (1+k_\mathrm{m})\left[\left(\div \vec{\eta}^\mathrm{f}+\left(\div
\vec{\eta}^\mathrm{f}\right)^T\right)+\left(\div \cdot \vec{\eta}^\mathrm{f}\right)\vec{\mathrm{I}} \right],
\end{align}
where $\vec{\eta}^\mathrm{f}$ represent the ALE mesh node displacement and 
$k_\mathrm{m}$ is a mesh stiffness variable chosen as a function of the element area to 
limit the distortion of small elements located in the immediate vicinity of the fluid-body 
interface \cite{Liu2014}.

Given a base flow $\bu_0$, the corresponding linearized NS equations can be written in a semi-discrete form as 
\begin{equation}
\mathsfbi{E}\frac{d \mathsfbi{Q} }{d t} = \mathsfbi{F} \mathsfbi{Q} + \mathsfbi{G} \mathsfbi{V}_m,
\label{eq:NS_pr}
\end{equation}
where the matrices and vectors in Eq.~(\ref{eq:NS_pr}) are
\begin{subequations}
\begin{equation}
 \mathsfbi{F}=\left( \begin{array}{cc}
  -\left(  \right)\cdot \boldsymbol{\nabla} \bu_0 - \bu_0 \cdot \boldsymbol{\nabla} \left(  \right)+ \mu \left(\boldsymbol{\nabla} \left(  \right)
   + \boldsymbol{\nabla}^T\left( \right) \right)
     & -\boldsymbol{\nabla}\left(  \right) \\
 \boldsymbol{\nabla} \cdot \left(  \right) & \mathsfbi{0}
\end{array}  \right),
\label{eq:NS_state_space}
\end{equation} 

%\begin{align}
%\rho^\mathrm{f} \left( \frac{\partial \bu^\mathrm{f}}{\partial t} \bigg\rvert_{\widehat{x}} + \left(\bu^\mathrm{f}-\bw\right)\cdot \boldsymbol{\nabla} \bu^
%\mathrm{f} \right )= \boldsymbol{\nabla} \cdot \boldsymbol{\sigma}^\mathrm{f} + \mathbf{b}^\mathrm{f} \mbox{ on } \Otf \label{eq:N-S} \\
%\boldsymbol{\nabla} \cdot\bu^\mathrm{f} = 0 \mbox{ on } \Otf \label{eq:continuity}
%\end{align}


%
\begin{equation}
\mathsfbi{G} = \left( \begin{array}{cc}
  \left(  \right)\cdot \boldsymbol{\nabla} \bu_0 & \mathsfbi{0} \\
 \mathsfbi{0}  & \mathsfbi{0}
\end{array}  \right),
\mathsfbi{E} = \left( \begin{array}{cc}
  \mathsfbi{I} & \mathsfbi{0} \\
 \mathsfbi{0}  & \mathsfbi{0}
\end{array}  \right),
\mathsfbi{Q} = \left( \begin{array}{cc}
  \bu \\
  p
\end{array}  \right),
\mathsfbi{V}_m = \left( \begin{array}{cc}
  \bw \\
 \mathsfbi{0}
\end{array}  \right).
\label{eq:NS_EQ}
\end{equation} 
\end{subequations}

The linearized NS equation for stationary cylinder is obtained 
by setting mesh velocity $\bw=\mathsfbi{0}$. 
After the discretization, the generalized eigenvalue problem of the linearized 
NS equation can be written as
$(\mathsfbi{A_f} + \sigma \mathsfbi{B_f}) \mathsfbi{p} =\mathsfbi{0}$, 
where the non-symmetric matrices $\mathsfbi{A_f}$ and $\mathsfbi{B_f}$ 
results from the spatial and temporal discretizations, 
$\sigma$ denotes the eigenvalue of the discretized system,
$\mathsfbi{p}$ is the right eigenvectors (forward modes). 
The corresponding discrete adjoint problem can be obtained as
$\mathsfbi{q} (\mathsfbi{A_f} + \sigma \mathsfbi{B_f})  =\mathsfbi{0}$, 
where $\mathsfbi{q}$ is the left eigenvector of the discrete system 
and represents the approximation of the adjoint modes \citep{Luchini2007}. 
%and its corresponding adjoint solver can be written immediately as
%\begin{equation}
%\mathsfbi{E}^*\frac{d \mathsfbi{Q}^* }{d t} = \mathsfbi{L}^* \mathsfbi{Q}^* 
%\label{eq:NS_ad}
%\end{equation}
%where asterisks $^*$ denotes conjugate transpose. 
%%How to compute base flow
%%The base flow is computed by fixed point iteration without the time dependent term
%%in the nonlinear incompressible NS equation. 
% 
The linearized NS equations are solved by the semi-discrete 
variational procedure,  which is employed for the nonlinear 
fluid-structure equations in \citep{Liu2014,Jaiman2015}.  
The fluid-structure coupling is achieved 
through a partitioned staggered procedure \citep{Jaiman2011}. 
We next present the ERA-based ROM via input-output dynamics of FOM. 

%%%%%%%%%%%%%%%%%%%%%%%%%%%%%%%%%%%%%%%%%%%%%%%%%%%%%%%%%%%%%%%%%%%%%%%%%%%%%%%%%%%%%%%%%
\subsection{Feedback control via reduced-order model}
%% VIV-ROM 
The ERA-based ROM is constructed by linear input/output dynamics of the NS equations given in Eq.~(\ref{eq:N-S}) and Eq.~(\ref{eq:continuity}). 
The input vector  $\mathsfbi{u}=[Y,V_c]^T$ for the fluid system are transverse displacement
$Y$ and the suction and blowing vertical velocity $V_c$, while the output  
is the total lift coefficient $C_l$ over the structural body.
The fluid ERA-based ROM formulated in state-space form at discrete times $t=k\Delta t$, 
$k=0,1,2,...,$ with a constant sampling time $\Delta t$ reads 
 %
\begin{equation}
\left. \begin{array}{ll}
\displaystyle \mathsfbi{x_f}(k+1)=\mathsfbi{Ax_f}(k)+\mathsfbi{Bu}(k)  \\[8pt]
\displaystyle C_l(k)=\mathsfbi{Cx_f}(k)+\mathsfbi{Du}(k)
\end{array}\right\}
 \label{eq:fluid}
\end{equation}  
where $\mathsfbi{x_f}$ is an $n_r$-dimensional state vector, and the integer $k$ is a sample index for the time stepping.
%
The system matrices are $\left(\mathsfbi{A},\mathsfbi{B},\mathsfbi{C},\mathsfbi{D}\right)$, 
which are obtained by ERA method. To construct the ERA-based ROM, 
the impulse response of NS equation is first defined as $\mathsfbi{y}$, based on which
the generalized block Hankel matrix $r \times s$ can be constructed as
\begin{equation}
\mathsfbi{H}(k-1) = 
\left[ {\begin{array}{*{20}c}
    \mathsfbi{y}_{k+1}&     \mathsfbi{y}_{k+2}& ...& \mathsfbi{y}_{k+s}        \\    
    \mathsfbi{y}_{k+2}& \mathsfbi{y}_{k+3}& ...& \mathsfbi{y}_{k+s+1}    \\
    \vdots& \vdots& \ddots& \vdots        \\
    \mathsfbi{y}_{k+r}& \mathsfbi{y}_{k+r+1}& ...&\mathsfbi{y}_{k+(s+r-1)}
 \end{array} } \right],
\label{eq:Hankel}
\end{equation}  
and by applying singular value decomposition (SVD) of Hankel matrix $\mathsfbi{H}(0)$ as
\begin{equation}
\mathsfbi{H}(0) = \mathsfbi{U}\mathsfbi{\Sigma}\mathsfbi{V}^*=
\left[\mathsfbi{U}_1 \quad \mathsfbi{U}_2\right]
\left[{\begin{array}{*{20}c}
   \mathsfbi{\Sigma}_1 \quad \mathsfbi{0} \\
   \mathsfbi{0} \quad \mathsfbi{\Sigma}_2
 \end{array} }
\right]
\left[{\begin{array}{*{20}c}
  \mathsfbi{V}_1^* \\
  \mathsfbi{V}_2^*
 \end{array} }
\right]
\end{equation}
where the diagonal matrix $\Sigma$ are the Hankel singular values (HSVs). 
The block matrix $\Sigma_2$ contains the zeros or 
negligible elements. By truncating the dynamically less significant states, we estimate 
$\mathsfbi{H}(0) \approx \mathsfbi{U}_1\mathsfbi{\Sigma}_1\mathsfbi{V}_1^*$. 
The state space matrices $(\mathsfbi{A}, \mathsfbi{B}, \mathsfbi{C}, \mathsfbi{D})$ are obtained by
\begin{equation}
\left. \begin{array}{ll}

\displaystyle \mathsfbi{A}=\Sigma^{-1/2}_{1}\mathsfbi{U}^{*}_{1}\mathsfbi{H}(1)\mathsfbi{V}_{1}\Sigma^{-1/2}_{1} \\
\displaystyle \mathsfbi{B}=\Sigma^{1/2}_{1}\mathsfbi{V}^{*}_{1}\mathsfbi{E_m}   \\
\displaystyle \mathsfbi{C}=\mathsfbi{E_t}^{*}\mathsfbi{U}_{1}\Sigma^{1/2}_{1}   \\    
\displaystyle \mathsfbi{D}=\mathsfbi{y}_{1}
\end{array}\right\}
 \label{eq:ERA_ROM_matrices}
\end{equation}  
Here, $\mathsfbi{E_m}^*=\left[\mathsfbi{I}_q \quad \mathsfbi{0} \right]_{q\times N}$, 
$\mathsfbi{E_t}^{*}=\left[\mathsfbi{I}_p \quad \mathsfbi{0} \right]_{p \times M}$, where
$N =s \times q$, $M =r \times p$, and $\mathsfbi{I}_{p,q}$ are the identity matrices. 
%$\mathsfbi{x_f}$ is the $n_r$-dimensional state vector. 
The matrices 
$\mathsfbi{B}$ and $\mathsfbi{D}$ can be rewritten as $\mathsfbi{B}=[\mathsfbi{B}_Y,\mathsfbi{B}_{V_c}]$ and  
$\mathsfbi{D}=[D_Y,D_{V_c}]$, where the subscripts denote the input components defined in the  vector $\mathsfbi{u}$. 
{\color{red} The ERA-based ROM is constructed in the vicinity of a given base flow at $t=0$ ($k=0$), 
and the impulse signal starts from $t= \Delta t$ ($k=1$).}

The VIV system is simplified to 
a transversely vibration circular cylinder with one degree-of-freedom (1-DOF), and the nondimensional structural equation in the state-space form is written as \citep{Yao2016_JFS}
\begin{equation}
 \dot{\mathsfbi{x_{s}}}=\mathsfbi{A_{s}}\mathsfbi{x_{s}}+\mathsfbi{B_{s}}C_l.
\label{eq:structure_state}
\end{equation} 
The state matrices and vectors are
\[ 
	\mathsfbi{A_{s}}=\left[ \begin{array}{cc}
	0 & 1\\
	 -(2\pi F_{s})^2 &-4\zeta\pi F_{s}      
 	\end{array} \right],
 	\mathsfbi{B_{s}}=\left[ \begin{array}{c}
 	0 \\
 	\frac{2}{\pi m^{*}}
	\end{array}  \right],
	\mathsfbi{x_{s}}=\left[ \begin{array}{c}
 	Y \\
 	\dot{Y}
	\end{array}  \right], 
 \] 
 %
where $Y$ is the transverse displacement, $C_{l}$ is the lift coefficient, 
$F_{s}$ is nondimensional reduced structural frequency 
defined as $F_{s}=f_{N}D/U = 1/U_r$. 
Based on the ERA-based ROM, the resulting closed-loop system for VIV 
can be formulated by coupling  of the structure Eq.~(\ref{eq:structure_state}) 
and the fluid system Eq.~(\ref{eq:fluid}) as  
\begin{equation}
\left. \begin{array}{ll}
\displaystyle \mathsfbi{x_{sf}}(k+1)=\mathsfbi{A_{sf}}\mathsfbi{x_{sf}}(k) 
+ \mathsfbi{B_{sf}} \mathsfbi{u}_c(k) \\[8pt]
\displaystyle \mathsfbi{y_{sf}}(k+1)=\mathsfbi{C_{sf}}\mathsfbi{x_{sf}}(k) 
+ \mathsfbi{D_{sf}} \mathsfbi{u}_c(k)
\end{array}\right\}
 \label{eq:vivrom}
\end{equation}  
where
\[
\underset{(n_r+2) \times (n_r+2)}{\mathsfbi{A_{sf}}}=\left[ \begin{array}{cc}
\mathsfbi{A_{sd}}+\mathsfbi{B_{sd}}D_Y\mathsfbi{C_{sd}} & \mathsfbi{B_{sd}}\mathsfbi{C} \\
\mathsfbi{B}_Y\mathsfbi{C_{sd}} & \mathsfbi{A}
\end{array}  \right], \quad
\underset{(n_r+2) \times 2}{\mathsfbi{B_{sf}}}=\left[ \begin{array}{cc}
\mathsfbi{0} & \mathsfbi{B_{sd}}D_{V_c} \\
\mathsfbi{0} & \mathsfbi{B}_{V_c}
\end{array}  \right] \]
\[
\underset{3 \times (n_r+2)}{\mathsfbi{C_{sf}}}=\left[ \begin{array}{cc}
\mathsfbi{I} & \mathsfbi{0} \\
D_Y\mathsfbi{C_{sd}} & \mathsfbi{C}
\end{array}  \right], \quad
\underset{3 \times 2}{\mathsfbi{D_{sf}}}=\left[ \begin{array}{cc}
\mathsfbi{0} & \mathsfbi{0} \\
0 & D_{V_c}
\end{array}  \right].
\]
%
The structural time discrete matrices are defined as 
$\mathsfbi{A_{sd}}=e^{\mathsfbi{A_{s}}\Delta t}$, 
$\mathsfbi{B_{sd}}={\mathsfbi{A_{s}}}^{-1}(e^{\mathsfbi{A_{s}}\Delta t}-\boldsymbol{I})\mathsfbi{B_{s}}$ 
and $\mathsfbi{C_{sd}}=[1,0]$, 
%
where the state vector $\mathsfbi{x_{sf}}=[\mathsfbi{x_{s}},\mathsfbi{x_{f}}]^T$ is a $(n_r+2)$-dimensional vector, and the output
vector is defined as $\mathsfbi{y_{sf}}=[\mathsfbi{x_{s}},C_l]^T$. 
The present ERA-based ROM allows to provide the access to most controllable 
and observable modes of the coupled fluid
and structure system. With regard to control design, the ROM 
is valid in the neighbourhood of the unstable steady state.

%% LQR control algorithm 
As shown in figure \ref{fig:controlplant}, the optimal feedback control based 
on the linear quadratic regulator (LQR) is utilized to 
calculate the optimal gain matrix $\mathsfbi{K}$ such that the state-feedback law 
$\mathsfbi{u}_c(k)=-\mathsfbi{Kx_{sf}}(k)$ minimizes the quadratic cost function 
for the discrete system as
%%
\begin{equation}
J=\sum_{k=1}^{\infty}\left\lbrace \mathsfbi{x^*_{sf}}\mathsfbi{Q}\mathsfbi{x_{sf}} + \\
\mathsfbi{u}_c^*\mathsfbi{R}\mathsfbi{u}_c \right\rbrace 
\label{eq:lqr}
\end{equation}  
where $\mathsfbi{Q}$ and $\mathsfbi{R}$ set the relative weights of state deviation and input usage, respectively, 
and the asterisk $^*$ denotes the transpose of matrix. 
%
The $\mathsfbi{Q}$ is chosen as identity matrix 
$\mathsfbi{I}$ for simplicity, whereas $\mathsfbi{R}=c\mathsfbi{I} > 0$  provides input to the cost function $J$.
%
The coefficient $c>0$  gives a relative weighing of output and input norms and can be tuned 
for an optimal tradeoff between the efficiency of VIV regulation and the energy input effort.
%and to avoid any overshot or aggressive control input. 
The control input is defined as the blowing and suction 
through vertical velocity magnitude $\mathsfbi{u}_c=(0, V_c)$ over 
the surface of two-dimensional cylinder. 
%This will provide a single scalar value of the control input $V_c$ thus the optimal gain matrix $\mathsfbi{K}$ recovers to a single row vector of diagonal entries. 
For the LQR algorithm, the full state vector $\mathsfbi{x_{sf}}(k)$ should be observable to calculate the control input $\mathsfbi{u}_c$, while it is not always possible in practice. 
By assuming the measurements of force, velocity or acceleration on the body mounted sensors, 
we can estimate  the full state vector via the Kalman filter as 
%%
\begin{equation}
\mathsfbi{\hat{x}_{sf}}(k+1)=\mathsfbi{A_{sf}}\mathsfbi{\hat{x}_{sf}}(k) + \mathsfbi{B_{sf}} \mathsfbi{u}_c(k) \\
+\mathsfbi{L}[\mathsfbi{y_{sf}}(k)-\mathsfbi{C_{sf}}\mathsfbi{\hat{x}_{sf}}(k)-\mathsfbi{D_{sf}} \mathsfbi{u}_c(k)]
\label{eq:Kalman}
\end{equation}  
where $\mathsfbi{\hat{x}_{sf}}$ is the minimum mean-square estimator of the state vector  and $\mathsfbi{L}$ is 
the filter gain matrix. After that the control input vector $\mathsfbi{u}_c$ can be determined 
as $\mathsfbi{u}_c(k)=-\mathsfbi{K}\mathsfbi{\hat{x}_{sf}}(k)$, resulting a closed-loop VIV system as given in 
Eq.~(\ref{eq:vivrom}).
Although the formulation presented here is general for any vibrating 
structure, results are presented for a canonical bluff body of circular cylinder.
%
We next demonstrate the designed controller for the unstable 
flow past a stationary circular cylinder and the feedback control of freely vibrating cylinder.
%%
\begin{figure}
\centering
 \includegraphics[scale=0.6]{fig2}
	\caption{Feedback control of VIV using reduced-order model:
	         schematic of closed-loop control $G_{CL}$ with ERA-based ROM, 
             where $\delta$ and $\mathsfbi{K}$ represent  
             impulse input and gain matrix, respectively,
              Kalman denotes the filter defined in Eq.~(\ref{eq:Kalman}). }
	\label{fig:controlplant}
\end{figure}


%% stability, receptivity and wavemaker 
\begin{figure}
\centering
\begin{minipage}{.4\textwidth}
\begin{subfigure}[b]{0.4\textwidth} 
\centering
  \includegraphics[scale=0.4]{fig3a}
	\caption{}
	\label{fig:re60_stability}
	\end{subfigure}	\\
\begin{subfigure}[b]{0.4\textwidth} 
\centering
  \includegraphics[scale=0.4]{fig3b} 
	\caption{}
	\label{fig:re60_receptivity}
	\end{subfigure}	\\			
\end{minipage}%
\begin{minipage}{.4\textwidth}
  \begin{subfigure}[c]{0.45\textwidth}
\centering
  \includegraphics[scale=0.4]{fig3c} 
    \caption{}
    \label{fig:wavemaker}
    \end{subfigure}  
\end{minipage} 

\caption{Spatial distribution of flow field: 
(a) forward, (b) adjoint velocity amplitudes, and 
(c) wavemker region for circular cylinder at $Re=60$. 
         In (c), actuation slots as triangles and body-mounted force sensor 
         as red line over the cylinder are shown for $BS0$ configuration.
         In (a) and (b), Contour levels are from $0.002$ to $0.018$ in increment of $0.002$.
         In (c), contour levels are from $0.02$ to $0.2$ in increment of $0.01$.
         }
\end{figure}


%%%%%%%%%%%%%%%%%%%%%%%%%%%%%%%%%%%%%%%%%%%%%%%%%%%%%
\section{Results}\label{sec:results}
\subsection{Active feedback control of unsteady wake flow}\label{sec:stationary}
We first demonstrate the feedback control scheme for the  flow past a stationary circular cylinder at $Re=60$ with $BS0$. The system
input is the blowing and suction vertical velocity $V_c$, and the output is the fluctuating lift coefficient $C_l$. The corresponding 
fluid ROM can be obtained by setting $\mathsfbi{B_Y}=\mathsfbi{0}$ and $D_Y=0$ 
in Eq.~(\ref{eq:fluid}). 
%% ROM contruction procedure 
%
To start with the ERA-based ROM construction, the unstable steady state (i.e. the base flow) is first computed by a fixed point iteration without the time dependent term in Eq. (\ref{eq:N-S}). 
For that purpose, 800 impulse response outputs ($C_l$) are stacked 
at each time step $\Delta t=0.05$ by imposing an impulse of $\delta(t)=10^{-4}$ 
to the blowing and suction vertical velocity $V_c$. 
Subsequently, the ERA-based ROM is obtained by performing a
singular value decomposition of a $500 \times 200$ Hankel matrix. 
The order of ERA-based ROM is determined by examining the singular values 
of the Hankel matrix.
The linearity of the impulse response outputs ($C_l$) is confirmed 
by comparing the impulse response of two different 
values $\delta(t)=10^{-4}$ and $10^{-3}$. To visualize the most excited flow structures, 
the leading POD mode shown in figure \ref{fig:modes}, is extracted via proper orthogonal decomposition (POD) method. For this purpose, the snapshots of flow field is stacked at
each time step during the impulse response modeling.
%
The aforementioned ERA-based ROM construction procedure can be considered as 
open-loop identification process.
% wavemaker  
%
Based on the wavemaker region \citep{Luchini2007}, we determine the locations 
with high sensitivity and strong response, which are computed by taking the pointwise 
product of the forward and adjoint global modes as shown in figure \ref{fig:re60_stability} and \ref{fig:re60_receptivity}. 
The modes are obtained directly by solving 
a generalized eigenvalue problem of the linearized NS Eq. (\ref{eq:NS_pr}) in the neighbourhood of the base flow. 
%
%
As shown in figure \ref{fig:wavemaker}, 
the body mounted force sensor and the suction and blowing actuators 
are within the wavemaker region, which allows the feedback control 
to produce the largest drift of unstable observable and controllable modes.

%% mode 
\begin{figure}
\centering
\begin{subfigure}[c]{0.495\textwidth} 
\centering
  \includegraphics[scale=0.45]{fig4a}
	\caption{}
	\label{fig:mode1ux}
	\end{subfigure}	
\begin{subfigure}[c]{0.495\textwidth} 
\centering
  \includegraphics[scale=0.45]{fig4b}
	\caption{}
	\label{fig:mode1uy}
	\end{subfigure}		
        \caption{Leading POD mode at $Re=60$:  
        (a) streamwise velocity, and (b) the cross-stream velocity. 
        Contour levels are from $-0.01$ to $0.01$ in increments of $0.0025$.}
        \label{fig:modes}
\end{figure}
%

%% ROM validation and impulse response suppression 
\begin{figure}
\centering
\begin{subfigure}[b]{\columnwidth}
\centering
 \includegraphics[scale=0.5]{fig5a}
    \caption{}
    \label{fig:val_cl}
    \end{subfigure} \\
\begin{subfigure}[b]{\columnwidth} 
\centering
  \includegraphics[scale=0.5]{fig5b}
	\caption{}
	\label{fig:val_uc}
	\end{subfigure}	
        \caption{Impulse response of stationary cylinder with $BS0$ configuration: temporal variation of
        (a) lift coefficient $C_l$, 
        (b) control input $V_c$ predicted by ROM and 
        compared with the FOM at $Re=60$ and $c=10^2$.
         The controller is switched on after $tU/D=50$ convective time units.  }
        \label{fig:linear_val}
\end{figure}

%Figure \ref{fig:linear_val} confirms that the ROM with $N_{modes}=33$ overlaps the FOM almost 
%perfectly. 
% wavemaker  
%Based on the wavemaker region \citep{Luchini2007}, we determine the locations 
%with high sensitivity and strong response, which are computed by taking the pointwise 
%product of the forward and adjoint global modes. 
%%
%%
%As shown in figure \ref{fig:wavemaker}, 
%the body mounted force sensor and the suction and blowing actuators 
%are within the wavemaker region, which allows the feedback control 
%to produce the largest drift of unstable observable and controllable modes.             
%%
Once the ERA-based ROM is constructed, the LQR method is employed to design the optimal feedback gain $\mathsfbi{K}$ 
for the different values of weighting parameters $c=10^2, 10^3, 10^4$. 
As shown in figure \ref{fig:controlplant}, the closed-loop model 
can be directly constructed through the ERA-based ROM. 
The effectiveness of closed-loop ROM with $c=10^2$ is examined  
by comparing temporal variations of $C_l$ and $V_c$ against the FOM counterpart, 
as illustrated in figure \ref{fig:linear_val}. 
An excellent match is found between the ROM with number of modes $n_r = 33$ and 
the FOM, thereby confirming the accuracy and convergence of 
the present ERA-based ROM. The impulse response of the open-loop system is rapidly attenuated 
once the controller is switched on at $t \geqslant 50$. 

The eigenvalues are calculated by $\lambda=$log(eig($\mathsfbi{A_{sf}-B_{sf}K}))/\Delta t$, where $\Delta t=0.05$ is used for all the computations. Figure \ref{fig:sta_eig} shows 
the  comparison of eigenvalue distribution between the open-loop (OL) and closed-loop (CL)
systems. As expected, a pair of eigenvalue of open-loop system is at the unstable
right half-plane as $Re=60 > Re_{cr}$ (where the critical Reynolds number is
$Re_{cr} \approx 46.8$). 
On the other hand, the eigenvalues of closed-loop system are all in the stable 
left half-plane, which further confirms that the closed-loop system is stable 
via the process of placing the poles in the stably-damped locations in the complex plane. 
%%%
As $c$ decreases, figure \ref{fig:sta_eig} also indicates that 
the eigenvalue moves further leftward and the closed-loop system becomes more stable. 
 However, a smaller value of $c$ introduces a larger control input $V_c$ thus   
 there is a tradeoff between the aggressive control input and the 
 rapid suppression of wake instability. 

%% lco suppression for stationary
\begin{figure}
\centering
\begin{minipage}{.4\textwidth}
  \begin{subfigure}[c]{0.45\textwidth}
\centering
  \includegraphics[scale=0.45]{fig6a} 
    \caption{}
    \label{fig:sta_eig}
    \end{subfigure}  \\
\end{minipage} 
\begin{minipage}{.4\textwidth}
\begin{subfigure}[b]{0.4\textwidth} 
\centering
  \includegraphics[scale=0.4]{fig6b}
	\caption{}
	\label{fig:lco_cl}
	\end{subfigure}	\\
\begin{subfigure}[b]{0.4\textwidth} 
\centering
  \includegraphics[scale=0.4]{fig6c} 
	\caption{}
	\label{fig:lco_cd}
	\end{subfigure}	\\
\begin{subfigure}[b]{0.4\textwidth} 
\centering
  \includegraphics[scale=0.4]{fig6d}
	\caption{}
	\label{fig:lco_uc}
	\end{subfigure}				
\end{minipage}%
\caption{Feedback control of stationary cylinder at $Re=60$ with $BS0$: 
      (a) eigenspectrum for open and closed-loop with different values of $c$, 
         time variation of (b) $C_l$, (c) $C_d$ (base flow drag subtraction),
         and (d) control input $V_c$ corresponding to closed-loop response of  
         full nonlinear system.  While the system has an impulse at $t=0$, 
         the controller is switched on after $tU/D=175.5$.
         }
\end{figure}
%% 
\begin{figure}
\centering
\begin{subfigure}[b]{0.495\textwidth} 
\centering
  \includegraphics[scale=0.45]{fig7a}
	\caption{}
	\label{fig:iso_vor1}
	\end{subfigure}	
\begin{subfigure}[b]{0.495\textwidth} 
\centering
  \includegraphics[scale=0.45]{fig7b}
	\caption{}
	\label{fig:iso_vor2}
	\end{subfigure}	
\begin{subfigure}[b]{0.495\textwidth} 
\centering
  \includegraphics[scale=0.45]{fig7c}
	\caption{}
	\label{fig:iso_vor3}
	\end{subfigure}	
\begin{subfigure}[b]{0.495\textwidth} 
\centering
  \includegraphics[scale=0.45]{fig7d}
	\caption{}
	\label{fig:iso_vor4}
	\end{subfigure}	
        \caption{Stabilization effect on the vortex shedding of stationary circular cylinder at $Re=60$ with $BS0$ and $c=10^2$. 
        Snapshots of spanwise vorticity contours at: 
        $tU/D=$ (a) 175, (b) 225, (c) 250, (d) 325. 
        Contour levels are from $-1$ to $1$ in increments of $0.1$.}
        \label{fig:iso_vor}
\end{figure}


%%%saturated state 
%The designed controller is used to stabilize the saturated state. 
To demonstrate the AFBS controller to stabilize the saturated vortex street, 
figures \ref{fig:lco_cl} and \ref{fig:lco_cd} show the effective attenuation of 
the fluctuating lift $C_l$ and the drag $C_d$ 
with the control turned on at $tU/D=175.5$. 
It is expected that $c=10^4$ requires the smallest control input $V_c$ and 
the longest time scale to reach the target state,
followed by $c=10^3$ and $c=10^2$. 
The corresponding values of $C_\mu$ are $0.086$, $0.22$ and $0.52$ 
for $c=10^4, 10^3, 10^2$, respectively. The flow field evolution during suppression 
process is illustrated in figure \ref{fig:iso_vor}. 
%In the next section, the effect of 
%blowing and suction configuration is studied. 
%%%%%%%%%%%%%%%%%%%%%%%%%%%%%%%%%%%%%%%%%%%%%%%%%%%%%%%%%%%%%%%%
\subsection{Sensitivity study for unsteady wake flow control}\label{sec:sensitivity}
%
Before proceeding to the application of proposed ERA-based active jet control 
for VIV system,
we present a parametric study of a set of representative
actuator configurations based on suction/blowing pairs,
the effect of angular arrangement of actuators, and 
the combined versus independent control system architectures.
%
\subsubsection{Effect of actuator configurations}\label{sec:configuration}
To compare the effectiveness of the reference case $BS0$, we consider three 
additional configurations as depicted in figure \ref{fig:suctionbs123} to analyze the active feedback control based on ERA-based ROM.
%
The configuration $BS1$ has only midward actuation slots, whereby the configuration $BS2$ has the 
slots in the leeward side. By removing windward slots from $BS0$ 
at $\theta=(135 \degree, 225 \degree$),  the new configuration of $BS3$ is recovered.
%
Figure \ref{fig:dfbs_eig} shows the comparison of the eigenvalues 
of closed-loop  between $BS0$ and other three actuation configurations with $c=10^2$. 
While $BS0$, $BS1$ and $BS3$ provide similar 
damped eigenvalues, the configuration $BS2$ is the least effective actuator, which is further confirmed 
by the closed-loop response of full nonlinear system in 
figures \ref{fig:dfbs_lco_cl} and \ref{fig:dfbs_lco_cd}.
%
The figures also show that the fastest suppression of vortex street  
is achieved by $BS0$, while the $BS2$ actuator takes the longest time to eliminate vortex shedding. 
On the other hand, the actuation configurations of $BS1$ and $BS3$ 
behave similarly with respect to the suppression of the vortex street. 
The baseline $BS0$ with $\theta=45 \degree$ is considered 
to be the most effective actuator configuration. 
%
By comparing $BS1$ and $BS2$ configurations, the midward $BS1$ actuation slots 
have an improved control performance than the leeward $BS2$ configuration. 
This implies that the active suction/blowing control in the boundary layer (before 
the separation) is relatively efficient.
%
Based on the above study, the configuration $BS0$ is found to be more 
effective in reducing mean drag and suppressing fluctuating forces.
Next we investigate the effect of suction/blowing angle $\theta$ 
in the baseline $BS0$ configuration.
%while keeping the midward 
%slots $\theta=(135 \degree, 225 \degree$) fixed.

%Therefore, the configuration $BS0$ is employed for the VIV control in the next section. 
%While the $BS2$ actuator takes 
%the longest time to suppress $C_l$ and $C_d$, the fastest suppression of vortex street  
%is achieved by $BS0$. 
%
%% primary and adjoint mode; wavemaker region. 
%\begin{figure}
%\centering
%\begin{minipage}{.4\textwidth}
%  \begin{subfigure}[c]{0.4\textwidth}
%\centering
%  \includegraphics[scale=0.4]{primary_mode} 
%    \caption{}
%    \label{fig:primary_mode}
%    \end{subfigure}  \\
%\begin{subfigure}[c]{0.4\textwidth} 
%\centering
%  \includegraphics[scale=0.4]{adjoint_mode}
%	\caption{}
%	\label{fig:adjoint_mode}
%	\end{subfigure}	
%\end{minipage} 
%\begin{minipage}{.4\textwidth}
%\begin{subfigure}[b]{0.4\textwidth} 
%\centering
%  \includegraphics[scale=0.4]{wavemaker}
%	\caption{}
%	\label{fig:wavemaker}
%	\end{subfigure}		
%\end{minipage}%
%\caption{(a) Primary, (b) adjoint mode and (c) Wavemaker region for circular cylinder at $Re=60$. 
%The $BS0$ actuation slots are shown as triangles and body-mounted force sensor is shown as a red line in (c). 
%Contour levels are from $-0.2$ to $0.20$ in increment of $0.025$ for (a), 
%from $-1$ to $1$ in increment of $0.1$ for (b),
%from $0.02$ to $0.2$ in increment of $0.12$ for (c).}
%\end{figure}
%% different blowing and suction configuration

\begin{figure}
\centering
\includegraphics[scale=0.6]{fig8} 
	\caption{ Additional three ($BS1,BS2,BS3$) configurations 
           of actuators as blowing and suction slots over 
           the surface of circular cylinder. 
           The positive control input is defined as 
           suction from bottom and blowing at the top of the cylinder. 
           While $BS1$ configuration forms a  symmetric configuration of 
           suction-blowing pairs,  
           $BS2$ and $BS3$ are asymmetric with respect to the quadrants of cylinder.}
	\label{fig:suctionbs123}
\end{figure}


\begin{figure}
\centering
\begin{minipage}{.4\textwidth}
  \begin{subfigure}[c]{0.47\textwidth}
\centering
  \includegraphics[scale=0.45]{fig9a} 
    \caption{}
    \label{fig:dfbs_eig}
    \end{subfigure}  \\
\end{minipage} 
\begin{minipage}{.4\textwidth}
\begin{subfigure}[b]{0.4\textwidth} 
\centering
  \includegraphics[scale=0.4]{fig9b}
	\caption{}
	\label{fig:dfbs_lco_cl}
	\end{subfigure}	\\
\begin{subfigure}[b]{0.4\textwidth} 
\centering
  \includegraphics[scale=0.4]{fig9c} 
	\caption{}
	\label{fig:dfbs_lco_cd}
	\end{subfigure}	\\
\begin{subfigure}[b]{0.4\textwidth} 
\centering
  \includegraphics[scale=0.4]{fig9d}
	\caption{}
	\label{fig:dfbs_lco_uc}
	\end{subfigure}				
\end{minipage}%
\caption{Effect of the placement of fluidic actuators based suction-blowing pairs 
on the cylinder surface: (a) distribution of eigenspectrum of closed-loop system for different actuation slots placements ($BS0-BS3$),
		temporal variations of fluctuating (b) $C_l$, (c) $C_d$, 
                and (d) control input $V_c$ 
		for the closed-loop response of full nonlinear 
                system at $(Re,c)=(60,10^2)$. 
		%The system is subject to an impulse at $t=0$ with 
                %the controller switched on after $t=175.5$.
                }
\end{figure}
 
%%%%%%%%%%%%%%%%%
\subsubsection{Effect of angle for suction-blowing pairs}\label{sec:angle}
In this section, the sensitivity of actuation angle for $BS0$ is investigated 
by varying it from the baseline $\theta=45 \degree$ to $\theta=(30 \degree, 60 \degree)$. 
%
We keep the midward pair of suction/blowing actuation at 
$\theta=(90 \degree, 270 \degree)$, but we only 
change other two diametrically pairs of suction and blowing actuations.
%
The distribution of eigenspectrum, as shown in figure \ref{fig:angle_eigvalue}, suggests 
that $BS0$ configuration with $\theta=(30 \degree,45 \degree,60 \degree)$
provides a similar effectiveness with regard to the least damped eigenvalues. 
Next, the configuration $BS0$ with different angle $\theta$ is applied 
to stabilize the vortex 
shedding at nonlinear saturated state. 
The results in figures \ref{fig:angle_cl}, \ref{fig:angle_cd} and \ref{fig:angle_uc} 
show similar suppression trends for the lift $C_l$ and the drag $C_d$ signals and 
the control input $V_c$ obtained for the three angles. 
Figures \ref{fig:angle_cl} and \ref{fig:angle_cd} also show that $BS0$ with $\theta=60 \degree$ 
has slightly larger overshot when the controller is switched on 
followed by $\theta=30 \degree$ and $\theta=45 \degree$. 
%
In all our previous studies, 
the combined controller is designed as the macro-manipulator for controlling 
the global dynamics. In other words, the actuators are not allowed to vary their 
control input speed $V_c$ independently, therefore the same control input $V_c$ 
in all suction-blowing pairs is generated. 
%
It is interesting to study the effect of control system architecture, whereby 
there is no coupling between the suction/blowing subsystems and the actuators 
can generate independent control input.
%
\begin{figure}
\centering
\begin{minipage}{.4\textwidth}
  \begin{subfigure}[c]{0.47\textwidth}
\centering
  \includegraphics[scale=0.45]{fig10a} 
    \caption{}
    \label{fig:angle_eigvalue}
    \end{subfigure}  \\
\end{minipage} 
\begin{minipage}{.4\textwidth}
\begin{subfigure}[b]{0.4\textwidth} 
\centering
  \includegraphics[scale=0.4]{fig10b}
	\caption{}
	\label{fig:angle_cl}
	\end{subfigure}	\\
\begin{subfigure}[b]{0.4\textwidth} 
\centering
  \includegraphics[scale=0.4]{fig10c} 
	\caption{}
	\label{fig:angle_cd}
	\end{subfigure}	\\
\begin{subfigure}[b]{0.4\textwidth} 
\centering
  \includegraphics[scale=0.4]{fig10d}
	\caption{}
	\label{fig:angle_uc}
	\end{subfigure}				
\end{minipage}%
\caption{Effect of angle of fluidic actuators on the cylinder surface: 
         (a) distribution of eigenspectrum of closed-loop system for different actuation slots placements, 
		 time variations of fluctuating (b) $C_l$, (c) $C_d$, and (d) control input $V_c$ 
		for the closed-loop response of full nonlinear system at $$(Re,c)=(60,10^2)$$. 
		%The system is subject to an impulse at $t=0$ with 
                %the controller switched on after $t=175.5$.
                }
\end{figure} 

%% different velocity 
\begin{figure}
\centering
  \includegraphics[scale=0.8]{fig11}
    \caption{Schematics of combined and independent (decoupled) suction/blowing 
    control system 
    architectures. In contrast to the 
    combined controller $BS0$ (left), $BS0D$ (right) 
    partitions the controller pairs into different pieces
    with different blowing and suction velocity denoted 
    by $V_{c1}$, $V_{c2}$, and $V_{c3}$, respectively.}
    \label{fig:absf5}
\end{figure}
%% 
\begin{figure}
\centering
\begin{minipage}{.4\textwidth}
  \begin{subfigure}[c]{0.47\textwidth}
\centering
  \includegraphics[scale=0.45]{fig12a} 
    \caption{}
    \label{fig:absf5_eig}
    \end{subfigure}  \\
\end{minipage} 
\begin{minipage}{.4\textwidth}
\begin{subfigure}[b]{0.4\textwidth} 
\centering
  \includegraphics[scale=0.4]{fig12b}
	\caption{}
	\label{fig:absf5_cl}
	\end{subfigure}	\\
\begin{subfigure}[b]{0.4\textwidth} 
\centering
  \includegraphics[scale=0.4]{fig12c} 
	\caption{}
	\label{fig:absf5_cd} 
	\end{subfigure}	    \\
\begin{subfigure}[b]{0.4\textwidth} 
\centering
  \includegraphics[scale=0.4]{fig12d} 
	\caption{}
	\label{fig:absf5_Vc}
	\end{subfigure}				
\end{minipage}%
\caption{Comparison between combined versus decoupled control system architectures: 
         (a) distribution of eigenspectrum. $c=10^2$ for $BS0$, and $c=(10,10^2,10^3)$ for $BS0D$,
         time variations of fluctuating (b) $C_l$, (c) $C_d$, and (d) control input $V_c$ 
		 for the closed-loop response of full nonlinear system at $c=10^2$ and $Re=60$.   
		 In (c),  the vector ($V_{c1},V_{c2},V_{c3}$) is the control input of the $BS0D$ configuration. 
          }
\end{figure}

\subsubsection{Combined versus independent controller}\label{sec:independent_control}
To understand the effect of control architecture, 
we decouple the pairs of suction/blowing actuators and compare the performance 
against the combined controller counterpart.
The decoupled controller configuration with different blowing and suction velocity, 
termed as $BS0D$, is shown in figure \ref{fig:absf5}. 
As compared with the combined controller configuration 
$BS0$ with single DOF control input velocity $V_c$, 
the controller of $BS0D$ has three independent control inputs $V_{c1}$, $V_{c2}$ and $V_{c3}$. 
As demonstrated in figure \ref{fig:absf5_eig}, the combined configuration $BS0$ 
is more effective to damp the unstable eigenvalues as compared to the decoupled $BS0D$ 
counterpart. Moreover the configuration $BS0D$ is found to be lesser sensitive for the 
least damped eigenvalues for the identical range of $c$ values. 
%
To further assess the performance of $BS0$ and $BS0D$ controllers, 
figures \ref{fig:absf5_cl} and \ref{fig:absf5_cd}
illustrate the nonlinear saturated state suppression through the force time histories. 
After removing the same matching control input constraint in the LQR algorithm 
for $BS0D$ controller, the force trends suggest that the controller becomes less
effective when using different blowing and suction velocity at the same value of $c$.
It is further confirmed in figure \ref{fig:absf5_Vc}, which shows 
that the control velocity inputs
($V_{c1}$, $V_{c2}$, $V_{c3}$) become smaller at the same $c=10^2$, 
resulting into larger time for the suppression of vortex street. 

As shown in the previous section, the baseline $BS0$ configuration with $\theta=45 \degree$ is 
the most effective suction/blowing controller configuration. 
Therefore, we will employ this configuration for the active feedback control of VIV.
%
As reported in \cite{owen2001}, passive control 
techniques that work well for suppressing loads for fixed cylinders may 
not be effective for elastically mounted configurations.
%
We next demonstrate our proposed active feedback control scheme for elastically 
mounted circular cylinders, which are free to vibrate 
in transverse only (1-DOF) and in coupled streamwise/transverse directions (2-DOF).

%%%%%%%%%%%%%%%%%%%%%%%%%%%%%%%%%%%%%%%%%%%%%%%%%%%%%%%%%%%%%%%%%%%%%%%
\begin{figure}
\centering
\begin{subfigure}[b]{\columnwidth}
\centering
  \includegraphics[scale=0.5]{fig13a}
    \caption{}
    \label{fig:viv_val_cl} 
    \end{subfigure}  \\
\begin{subfigure}[b]{\columnwidth}
\centering
  \includegraphics[scale=0.5]{fig13b}
    \caption{}
    \label{fig:viv_val_y} 
    \end{subfigure}   \\
\begin{subfigure}[b]{\columnwidth} 
\centering
  \includegraphics[scale=0.5]{fig13c}
	\caption{}
	\label{fig:viv_val_uc}
	\end{subfigure}	
        \caption{Impulse response of closed-loop VIV system based on AFBS control: temporal evolution of 
        (a) lift coefficient $C_l$, 
        (b) transverse displacement $Y$, and 
        (c) control input $V_c$ predicted by ROM and compared with the FOM 
        for $(Re,m^*)=(60,10)$, $F_s=0.176$, and $c=10^2$ with controller switched on at $tU/D=0$.}
        \label{fig:viv_val}
\end{figure}

\subsection{Feedback control of vortex-induced vibration}\label{sec:VIV}
%%
%% VIV mechanism 
In \cite{YaoJFM2016}, we discussed that the onset of VIV lock-in is related to 
the instability exchange between the structural mode (SM) and the fluid mode (WM),
where the SM and WM represent two distinct eigenvalue branches of ERA-based VIV ROM system.
The critical reduced natural frequency $F_s$ or onset reduced velocity $U_r=1/F_s$ can be pinpointed 
by the SM instability branch. The objective of the present active feedback control 
is to drive all unstable eigenvalues  of the VIV system to the stable left
half complex plane. 
%
While the open loop VIV system (without feedback control) can be obtained 
by simply setting $\mathsfbi{u}_c=0$ in Eq.~(\ref{eq:vivrom}), 
the optimal gain is computed for the VIV system 
at $(Re,m^*)=(60,10)$ using the similar procedure as discussed for the stationary case. 
%$F_s=0.176$ is used 
%for structural equation in Eq.~(\ref{eq:structure}), which is in the vicinity 
%of VIV onset $F_s=0.179$ as shown in figure \ref{fig:eig_viv}.


As shown in figure \ref{fig:viv_val}, the closed-loop VIV system 
response subject to an impulse is completely stabilized and the ROM results nearly 
overlap with the FOM counterpart. The linear stability is then determined 
by the eigenvalue analysis for both open- and closed-loop systems,
as illustrated in figure \ref{fig:eig_viv}. Both the WM and a part of SM 
$(0.147 < F_{s} \le 0.179)$ are unstable ($Re(\lambda) > 0)$ for the open-loop system, 
while the eigenvalue of the closed-loop system are all in the stable left half-plane 
for $0.005 \leq F_s \leq 0.5$ ($2 \leq U_r \leq 200$) with $c=10^2, 10^3$. 
The results show that the unstable eigenvalues 
are driven to the lower stable left half-plane as the coefficient $c$ 
decreases, as indicated by the dash-dot arrow in figure \ref{fig:eig_viv}. It is 
important to note that the closed-loop VIV system with $c=10^4$ 
only delays the VIV onset and remains unstable for $F_s \leq 0.117$ ($U_r \geq 8.55$), 
which indicates that the amplitude may grow continually as $F_s$ decreases. 
%Therefore, the $c \in [10^2,10^3]$ should be satisfied to ensure VIV free system. 
%Therefore, the closed-loop
%VIV system might induce even more dangerous galloping if the controller is not properly designed. 
%% mass ratio 

\begin{figure}
\centering
\begin{subfigure}[b]{0.495\textwidth}
\centering
  \includegraphics[scale=0.45]{fig14a}
    \caption{}
    \label{fig:eig_viv}
    \end{subfigure} 
\begin{subfigure}[b]{0.495\textwidth} 
\centering
  \includegraphics[scale=0.45]{fig14b}
	\caption{}
	\label{fig:eig_viv_dm}
	\end{subfigure}	
        \caption{Root loci of open and closed-loop VIV systems ($0.005 \leq F_s \leq 0.5$): 
        (a) sensitivity of $c$ parameter on eigenspectrum at $(Re,m^*)=(60,10)$, 
        (b) effect of mass ratio $m^*$ on the control performance at $(Re,c)=(60,10^3)$. 
        The unstable right half-plane is shaded in gray. The parameter 
        $c \in [10^2,10^3]$ ensures the VIV free system for 
        the whole range of reduced frequency. 
        }
        \label{fig:viv_linear}
\end{figure}

%% viv suppression 
\begin{figure}
\centering
\begin{subfigure}[b]{0.495\textwidth} 
\centering
  \includegraphics[scale=0.45]{fig15a}
	\caption{}
	\label{fig:viv_lco_viv}
	\end{subfigure}	
\begin{subfigure}[b]{0.495\textwidth}
\centering
  \includegraphics[scale=0.45]{fig15b}
    \caption{}
    \label{fig:viv_lco_y}
    \end{subfigure} 
\begin{subfigure}[b]{0.495\textwidth} 
\centering
  \includegraphics[scale=0.45]{fig15c}
	\caption{}
	\label{fig:viv_vor1}
	\end{subfigure}	
\begin{subfigure}[b]{0.495\textwidth} 
\centering
  \includegraphics[scale=0.45]{fig15d}
	\caption{}
	\label{fig:viv_vor2}
	\end{subfigure}	
\begin{subfigure}[b]{0.495\textwidth} 
\centering
  \includegraphics[scale=0.45]{fig15e}
	\caption{}
	\label{fig:viv_vor3}
	\end{subfigure}	
\begin{subfigure}[b]{0.495\textwidth} 
\centering
  \includegraphics[scale=0.45]{fig15f}
	\caption{}
	\label{fig:viv_vor4}
	\end{subfigure}	
        \caption{Suppression of VIV response of circular cylinder using AFBS procedure: 
        time variation of 
        (a) $C_l$ for saturated (thin-dash) and suppressed (thick-solid) states, 
        (b) transverse displacement $Y$ and actuator input $V_c$ 
        with controller switched on at $tU/D=150$, $c=10^3$ and $F_s=0.176$,
        snapshots of spanwise vorticity contours at: 
        $tU/D=$ (c) 155, (d) 400, (e) 600, (f) 800. 
        Contour levels are from $-1$ to $1$ in increments of $0.1$.}
        \label{fig:viv_cl_y_uc}
\end{figure}

In figure \ref{fig:eig_viv_dm}, 
the root loci of different mass ratios are plotted indicating the 
unstable eigenvalues of lower mass ratio damped more effectively with $c=10^3$. 
%
The controller with $c=10^3$ is also able to suppress the saturated VIV response 
as shown in figure \ref{fig:viv_cl_y_uc}. The lift coefficient $C_l$ and the transverse 
displacement $Y$ are suppressed once the controller is switched on after $tU/D=150$, when
the VIV response reaches saturated state. 
As compared to the saturated vortex shedding state
of stationary  cylinder, which requires the maximum $V_c \approx 0.65$, the VIV case requires approximately $4.5$ times larger 
maximum control input $V_c \approx 2.91$ with $C_\mu \approx 4.40$. 
%For $c=10^3$, the maximum $V_c$ for stationary and VIV saturated state suppression is approximately $0.65$ and $2.91$, respectively.
%
Figures \ref{fig:viv_vor1} and \ref{fig:viv_vor2} show snapshots of the spanwise vorticity contours to illustrate the attenuation of vortex shedding when the controller is 
switched on $t\geqslant 150$. 
%
We can infer from the success of the proposed linear AFBS for VIV that there exists 
a key linear mechanism during the self-sustained 
VIV oscillation that can be controlled effectively at low Reynolds number.
Nonlinear effects of the fluid flow, which 
attempt to saturate vortex shedding and to form a limit cycle in VIV, 
become dominant in the later stage.
It is worth pointing that identifying and controlling the linear mechanism
in nonlinear VIV is not equivalent to predict nonlinear VIV with a
linear model. 

 
%% viv suppression 
\begin{figure}
\centering
  \begin{subfigure}[c]{0.47\textwidth}
\centering
  \includegraphics[scale=0.45]{fig16a} 
    \caption{}
    \label{fig:viv2dof_saturated}
    \end{subfigure}  
    \begin{subfigure}[c]{0.47\textwidth}
\centering
  \includegraphics[scale=0.45]{fig16b} 
    \caption{}
    \label{fig:viv2dof_f}
    \end{subfigure}    \\
\begin{subfigure}[b]{\columnwidth} 
\centering
  \includegraphics[scale=0.6]{fig16c}
	\caption{}
	\label{fig:viv2dof_xy}
	\end{subfigure}	\\
\begin{subfigure}[b]{\columnwidth} 
\centering
  \includegraphics[scale=0.6]{fig16d} 
	\caption{}
	\label{fig:viv2dof_Vc}
	\end{subfigure}				
\caption{ Open loop saturated state of 2-DOF VIV system at $(Re,m^*)=(60,10)$: (a) figure-8 trajectory of $X$ and $Y$ displacements,
         (b) normalized power spectrum $P$ versus frequency $f^*$ of lift coefficient $C_l$ and transverse displacement $Y$, 
         where $f^*=f/F_s$. 
         Suppression of 2-DOF VIV response: (c) transverse $Y$ and inline $X$ response trends,
         (d) lift coefficient $C_l$ and actuator input $V_c$. 
          The controller is switched on after $tU/D=150$ with $c=10^3$ and $F_s=0.176$.  }
\end{figure}

%% viv contour 
\begin{figure}
\centering
  \begin{subfigure}[c]{0.47\textwidth}
\centering
  \includegraphics[scale=0.45]{fig17a} 
    \caption{}
    \label{fig:viv2dof_175}
    \end{subfigure}  
    \begin{subfigure}[c]{0.47\textwidth}
\centering
  \includegraphics[scale=0.45]{fig17b} 
    \caption{}
    \label{fig:viv2dof_200}
    \end{subfigure}    
\begin{subfigure}[b]{0.47\textwidth} 
\centering
  \includegraphics[scale=0.45]{fig17c}
	\caption{}
	\label{fig:viv2dof_300}
	\end{subfigure}	
\begin{subfigure}[b]{0.47\textwidth} 
\centering
  \includegraphics[scale=0.45]{fig17d} 
	\caption{}
	\label{fig:viv2dof_400}
	\end{subfigure}				
\caption{Adaptive feedback control of 2-DOF VIV of circular cylinder at $(Re, m^*)=(60,10)$. Snapshots of spanwise vorticity contours at: 
        $tU/D=$ (a) 175, (b) 200, (c) 400, (d) 500. 
        Contour levels are from $-1$ to $1$ in increments of $0.1$.}
\label{fig:viv2dof_vor} 
\end{figure}

%% table for summary 
\begin{table}
  \begin{center}
  \begin{tabular}{c c c c }
       control input   & stationary & 1-DOF &  2-DOF \\
       $(V_{c})_{max}$   & 0.65 & 2.91 & 3.00 \\
       $(C_{\mu})_{max}$   & 0.22 & 4.40 & 4.70 \\
      $(V_{c})_{rms}$   & 0.10 & 0.77 & 0.79 

  \end{tabular}
  \caption{ Comparison of control inputs among stationary cylinder, 1-DOF VIV system and 2-DOF VIV
  with $BS0$ at $c=10^3$. The root mean square ($rms$) and maximum ($max$) are computed for $t \in [t_0,t_0+200]$, and $t_0$ is when 
  the controller is switched on after $t_0=175.5$ and $150$ for stationary and VIV systems, respectively.}
  \label{tab:summary}
  \end{center}
\end{table}
 
% 2 DOF 
To demonstrate whether the designed controller for the 1-DOF VIV system can be utilized for 
2-DOF VIV system feedback control, we introduce free vibration in the streamwise ($X$) direction along with the transverse ($Y$) direction.
%with the same structural parameter setting ($F_s, \zeta, m^*$) as the transverse 1-DOF described in Eq. (\ref{eq:structure_state}). 
A typical figure-8 VIV saturated trajectory is shown in figure \ref{fig:viv2dof_saturated} at $F_s=0.176$. 
The frequency plot shown in figure \ref{fig:viv2dof_f} confirms the frequency lock-in occurrence. 
As shown in figure \ref{fig:viv2dof_xy} and \ref{fig:viv2dof_Vc}, 
similar suppression is achieved for the 2-DOF VIV
system with the same controller designed for 1-DOF VIV 
system (figure \ref{fig:viv_cl_y_uc}). As compared with 1-DOF, which requires maximum $V_c \approx 2.91$, 
the same controller requires maximum $V_c \approx 3.0$ for 2-DOF VIV saturated response suppression or approximately
$3 \%$ larger than its 1-DOF VIV counterpart. 
Figure \ref{fig:viv2dof_vor} further illustrates the similar flow field evolution during the suppression process for 
the 2-DOF system. Overall, the results suggest the proposed feedback control scheme based on vertical blowing/suction 
is able to suppress 2-DOF VIV system effectively with approximately the same amount of control energy as 1-DOF system. 
%

The control input is summarized in table \ref{tab:summary} for a stationary cylinder and VIV configurations with both 1-DOF and 2-DOF  response at $c=10^3$. 
As shown in the table, 2-DOF VIV system only requires slightly larger control energy to suppress the saturated VIV state than its 1-DOF counterpart. 
The results suggest that the transverse response is the most dynamically significant for the isolated circular cylinder VIV system. 
Therefore, the proposed controller designed for 1-DOF VIV system can be applied to reduce 2-DOF VIV response. 

\section{Concluding remarks}\label{sec:concluding}
Through the ERA-based dimensionality reduction, an active feedback blowing 
and suction concept is proposed to suppress the vortex street and VIV for 
flexibly mounted structures.
A variational finite element formulation has been considered for the full order 
fluid-structure model and the generalized eigenvalue problem of 
linearized Navier-Stokes system.
The control scheme relies on the optimal feedback gain by the LQR synthesis and the state estimation by 
the Kalman filter.
% AFBS based on linear optimal control theory works (VIV has linear elements)
Based on the combined vertical blowing and suction,
we employed the feedback control to obtain suitable gains that stabilize the wake instability 
and the vortex-induced vibration. 
%We discussed the canonical vortex shedding problem of moving the poles
%to the left-half of the complex plane and computed the system response for each case.  
We found that the essential elements in the self-sustaining VIV process are linear, 
and are subject to the active feedback control.
From the ERA-based computations and feedback control of unstable eigenmodes, 
it can be deduced that the increase in the VIV amplitude of cylinder occurs primarily through the linear instability process. 
%  comparison between different BS configuration 

By applying the AFBS procedure to the vortex shedding of stationary cylinder, we
observed a remarkable reduction in the time-dependent fluctuating
components of both  lift and drag forces. 
By means of eigenspectrum distributions via ERA-based simulations, 
we have explored several configurations to 
confirm the generality and sensitivity of the AFBS procedure 
with respect to a range of parameters.
While the configuration BS0 with six actuation slots 
performed efficiently, followed by $BS1$, $BS3$, the configuration 
$BS2$ exhibited a poor performance as having the actuators in the leeward side of cylinder.
% angle effect 
In general, the performance of $BS0$ configurations is not much sensitive to the jet angle $\theta$. When applying for 
nonlinear saturated state, the similar suppression is achieved with $\theta=(30 \degree, 45 \degree, 60 \degree)$,
while a slightly larger overshot is found when the controller is switched on for the angle $\theta=60 \degree$. 
% independent control velocity 
The performance of $BS0$ becomes less effective when removing the same control 
input velocity constraint, thereby suggesting an improved 
performance of combined control architecture than 
the decoupled independently designed control system. 
% VIV suppression 

With respect to the VIV suppression, the designed controller has performed 
well for a range of mass ratio $m^* \in [5,100]$  and 
the reduced natural structural frequency $F_s \in [0.005,0.5]$ at $Re=60$.
% Quantify energy input for VIV as compared to stationary
In contrast to the stationary vortex shedding, the suppression of VIV requires 
about four times large control input for the same Reynolds number $Re=60$ 
to suppress both the fluctuating lift and the VIV amplitude. 
The controller designed for transversely VIV 
system can be also adapted to 2-DOF VIV suppression 
with only approximately $3 \%$ larger control input than
the transverse 1-DOF VIV system. 
%
Since the present reduced-order model does not require any adjoint solver, 
the proposed ERA-based feedback control can be directly used for actual 
physical problem and experimental setting.
The ERA-based ROM approach is relatively straightforward and computationally efficient.
Furthermore the model offers a reasonably accuracy for the development of stabilized feedback controller design for unstable flows and VIV.
% Future
Three-dimensional and high $Re$ flows will be of interest for future study.

The first author would like to thank Singapore Maritime Institute Grant (SMI-2014-OF-04) 
for the financial support.
\bibliographystyle{jfm}
\bibliography{refs}

\end{document}
